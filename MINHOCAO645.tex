\chapterspecial{{Minhocão}}{}{}
 

\letra{A}{ história das minhocas.} Quando a floresta existia, mesmo que a terra
existia: 
--- Vou cavar minhocas! --- ninguém dizia isso. 
Não existia minhoca, como as minhocas não saiam, ninguém saia, ninguém
pescava depois de tirar minhocas. Era assim. Nós não as deixaremos cair
na água quando estamos com fome, nós cavamos onde há minhocas, nós as
tiramos, muitas surgirão; para que nós fizéssemos assim, ele morou com a
filha. Lá onde surgiu aquela mulher, a filha de Pokoraritawë ensinará
aos Yanomami a não gostar do marido, elas não gostam dos maridos às
vezes. Ensinando"-nos, a filha de Pokoraritawë se zangava demais, pois
estava com medo, não queria seu marido, apesar de ele ser muito bonito,
a mulher não o queria, a filha de Pokoraritawë fez as minhocas surgirem.
A mulher chegou lá com os dois Minhocões, que comiam o esperma deles
mesmos. Aquele que ela desposou, apesar de ser bonito, foi embora caçar,
até que afinal a mãe falou com a filha: 

--- Filhinha querida, teu marido foi de novo! Vai atrás dele! Vai!---
ela disse. 

Ela foi bem devagarzinho atrás dele. 

Ele foi, soprou veneno em cuxiús, matou, ele era muito bom caçador,
Paricá. Ela não gostava dele, de Paricá, era o nome do genro de
Pokoraritawë. Minhocão fez os filhotes se multiplicarem com a esposa de
Paricá. Quando seu marido passou, os dois chegaram aonde Paricá estava.
Ele estava longe à frente quando a mulher passou perto dos dois
Minhocões, o mais velho e o mais novo. 

Os dois moravam em terra plana e viviam na condição de Yanomami, pois
não existiam minhocas à época. Os pais das minhocas moravam lá no
início. Eles farão os filhos se multiplicar. Passando nesse caminho, lá
em baixo bem longe, Paricá matava cuxiús. As frutas de Minhocão estavam
grudadas. Naquele caminho, as frutas eram numerosas para atrair a
mulher. Toso, toso, toso, toso! Faziam os restos. Hõti, hõti, hõti!
Faziam assim também. 

Osdois eram muito bonitos, os pais das minhocas, tinham a testa
enfeitada de rabo de cuxiú, guardava na testa, o rosto dos dois era
enfeitado e bonito. Assim era o rosto dos dois. Os dois Minhocões tinham
barbabonita para parecer o rosto de Paricá e enganar a mulher. Ele
olhou: 

--- Krai! Rae!--- disseram assim. 

Os dois eram brancosos: 

--- Hɨ̃ɨ! Olhe! Olhe! É você?--- disse a mulher bem bonita com seios
bonitos.

--- Ô! De quem é essa voz?

Como tinha uma clareira, a mulher ficou em pé no limpo. 

--- Não pergunte quem sou! Sou eu! Você! É você mesmo! --- disse a
mulher.

--- Não, não sou aquele que você pensa, eu sou outro! 

--- É você, é seu rosto mesmo, assim que é seu rosto!--- disse ela. Ela
perguntou. 

Ele pronunciou seu nome: 

--- Eu sou Minhocão mesmo!

--- Não, você não é outro, é você! 

Enquanto ela insistiu em dizer isso, os dois Minhocões falaram para ela
logo. 

Um deles olhou e disse: 

--- Se você falar assim, tire essa folha nova de arumã, aí, aquela folha
enrolada, você a arranca e a desenrola, e você senta em cima, sente"-se
em cima. Coloque sua bunda em cima--- disseram os dois de um jeito
cantado. 

Rindo, ela correu para arrancar a folha. Pensando que era Paricá, pois
tinha o mesmo rosto, quando ele disse isso, ela arrancou a folha. Depois
de arrancá"-la e desenrolá"-la em um lugar bonito da clareira, onde não
havia nada, ela se sentou em cima onde estava limpo. Os dois desceram,
os dois desceram rapidamente e copularam com ela uma vez, não várias
vezes, somente uma vez. Apesar de copular somente uma vez, cada um com
ela, os dois copulavam enquanto o marido estava matando todos os cuxiús,
pois era muito bom caçador, acumulando as presas. 

Ela não alcançou, andava devagar. 

Depois de ter copulado, não foi nos dias seguintes, mas no mesmo dia,
apareceu o ventre que, apesar de uma vez só, já estava crescendo. 

--- Vai! Vai logo! --- disseram os dois Minhocões que voltaram para a
moradia deles. 

O ventre daquela que estava andando sozinha crescia e crescia. 

--- Vai lá onde teu marido está matando os cuxiús, ouça os gritos!---
disse o Minhocão. 

--- Hõhaaa!--- ela ficou pensando. 

Depois de falar isso, ela foi bem devagar à direção onde estava seu
marido. Indo lá, o ventre sempre crescia, porque não havia só um filho.
Apesar de serem pequenos, eles estavam acabando com a carne dela. Ela
ficou em pé, enquanto Paricá estava amarrando os cuxiús, ela ficou em pé
lá distante. 

Ele estava voltando. Ele havia matado todos os cuxiús e estava voltando,
depois de carregá"-los, ele estava voltando. Quando voltava, ele viu o
ventre dela enorme de gravidez. 

--- Nunca mexi essa mulher, e tem filho nesse ventre!--- ele pensou. 

Ele pensou simplesmente. Ele nunca tinha copulado com ela, pois ela não
gostava dele. Ele voltou passando. Ela voltou sozinha. Ela estava
voltando rindo. Ela estava voltando atrás, seu ventre cresceu
rapidamente. Ela voltava com esse ventre enorme. 

Depois de um dia, o ventre dela estava gigantesco. Ele olhou atrás e viu
a mulher com o ventre enorme. 

--- Hõãaa! É o ventre com criança--- ele pensou e continuou andando 

--- Hɨ̃ɨɨ! Será que eu a sujei uma vez?--- disse isso.

Xiri! Anoiteceu muito rápido. A noite caiu depressa. O ventre estava
cheio. Olha só o suporte dos bichos. Não havia só um! O ventre estava se
mexendo. 

--- Õa, õa, õa, õa!--- diziam dentro. 

A mulher sofria, sofria passando mal, sofria por causa do que acontecia
dentro dela. Doía muito o ventre dela. O marido dela estava deitado na
sua rede sem olhar para ela, enquanto o ventre dela doía, pois doía
muito, acariciando sua barba, enquanto a noite se tornou logo densa e
grossa, as minhocas saíram.

Weo! Weo! A placenta se derramava como se fosse água e saiam filhotes de
minhocão:

--- Ũa! Ũa! Ũa!--- diziam assim primeiro. 

Como parecia voz de criança, ele olhou para as crianças no chão, apesar
de estar deitado na rede, ele olhou. Não havia criança. Ele olhou de
soslaio. Não dava para ver. Em baixo dele:

--- Ũa, ũa, ũa, ũa!--- diziam assim sem parar. 

Eles nasciam, nasciam, nasciam, nasciam, nasciam, nasciam, nasciam. 

Hɨ̃ɨɨ! Havia tantos montes de minhoca que o fundo da casa sumiu, a vagina
dela estava cheia de minhocas. Depois do trabalho de parto, ela
olhou;ela fez desse jeito. Eles choravam como crianças, chorando de sede
já, eles acusavam sede: 

--- Sede! Sede!--- diziam com uma voz de criança --- Estou com sede! ---
diziam rapidamente. 

--- A criança cresceu tão rápido! --- ela pensou assim. 

Como estavam sempre com sede, ela deu o seio. 

--- Tusu! Suku! Tusu! Suku! --- faziam assim enquanto amamentavam. Ela
fez assim. Como as minhocas faziam sempre isso, ele ficou esperto. Ele
entendeu: 

--- Hɨ̃ɨ!--- ele pensou. 

A mãe dela chegou correndo. Apesar de olhar, ela não as viu
imediatamente. Apesar de escutar o choro de criança, ela olhou e voltou
a deitar. 

Deitada, a mãe das minhocas as cobriu, cobriu, cobriu, cobriu, cobriu.
Amanheceu. Como a filha estava indo de manhã cedo, ela falou para sua
mãe enquanto o marido ficava pensativo.

--- Mãe! Não descubra o que eu cobri no fundo da casa. Não fique olhando
o fundo da minha casa!--- disse. 

Havia tantas minhocas! Elas se embolavam, zoando, porque estava cheio 

--- Não olhe o fundo da minha casa. Não descubra o que eu cobri!--- ela
disse e saiu. 

Xiriririri! E sumiu. Enquanto isso, a mãe levantou da rede. 

--- Por que? Onde está a criança que deveria estar no colo,
recém"-nascida? Vai chorar muito assim!--- ela pensou, e correu até a
casa. 

Ela foi logo. Ela correu e descobriu o que estava onde a filha morava,
aquelas minhocas, todas mexiam a cabeça ao mesmo tempo. 

--- Xiririririri! Sede! Sede! Sede! Avó! Sede!--- eles a chamavam de
avó.--- Avó! Sede! Avó! Sede! Avó! Sede!--- todos diziam. 

--- Hɨ̃ãaaaaë!--- ela gritou logo--- Hɨ̃ãaaaë! Só para você fazer surgir
aqueles! Por isso! Você não trata bem seu marido! É por causa desses
bichos estranhos que você não conseguiu dormir!--- ela disse--- Vai! Meu
genro! Enquanto se mexem assim, derrube logo essa lenha, faça um fogo
grande para ela!--- disse a mãe. 

Ela mandava cremar a filha viva! Depois de ela dizer isso, ele desceu da
rede. Ele não demorou: derrubou aquele carapanã-uba.

Kraxi! Kraxi! Kraxi! Krao! Torou! Fazia lenha para cremá"-la. Enquanto
fazia lenha, ela voltou. Ela tinha ido tomar banho à toa, ela passou
onde estava partindo lenha. Ele virou as costas onde ele estava fazendo
lenha. Ele nem olhou. Ela deitou encolhida. 

Pou! Pou! Pou! Ele amontoou monte de lenha. Pou! Pou! Pou! Ele pegou
brasas para acender o montão de lenha, ele fez aumentar o fogo. Como a
lenha era seca, o fogo pegou logo. 

Weee! Ele fez uma cerca, fez para ela. Depois, ele correu atrás dela.
Ela nem se levantou, gritou para pegá"-la, pois queria a cremar viva. 

Weeeee! Ela estava deitada reta. Ela nem reagiu, ele correu a carregando
à direção do fogo, ela chorava: 

--- Ëaë! Ëaë! Mãe! Pai!--- disse ela. 

As pernas dela estavam balançando, dando impulsos. Ele a jogou no meio
do fogo. 

Pou! Ele pegou outra lenha que estava no chão e amassou, amassou com
força. 

Ëëëaaaëëë! Proto! O fogo queimou, enquanto cremava, a sogra dele correu
em cima dos minhocões para queimar os feios. Ela correu para pegá"-los.
Ela já tinha colocado água em cima do fogo em uma panela de barro para
cozinhá"-los. Ela correu com uma vasilha d´água quente à direção das
minhocas cobertas. Ela jogou a palha de coruá que as cobravam: 

Weeeo! Os minhocões gritavam:

--- Õiii, õiii, õiii!--- gritavam. 

--- Avó! Couro encolhido! Couro encolhido! Couro encolhido! Couro
encolhido!--- diziam inebriados, chamando"-se de pele encolhida. 

--- Avó! Couro encolhido! Avó! Couro encolhido! Avó! Couro encolhido!---
diziam os pedaços arrebentados. 

Olha só os montões de pedaços! Os pedaços estavam correndo logo, e
ocuparam toda a floresta, os minhocões. Ficaram ocupando a floresta, os
arrebentados correndo logo em todas as beiras de rio, entraram depressa
no fundo da terra. 

Depois de acontecer isso: são minhocões! Dizemos. Assim que aconteceu.
Não existiam minhocas. Foi com ela que se multiplicaram. Nós as faremos
cair na água para nós comermos peixes. A minhoca não apareceu do nada. 

Depois de os dois Minhocões copularem com ela e multiplicarem seus
filhotes, que foram embora com os pais. Os filhotes não moraram onde
foram cremados, nem ficaram perto. Os dois foram logo. Assim foi. Desde
que aconteceu, quando cai a chuva:

--- Tëɨ, tëɨ, tëɨ, tëɨ tëɨ!--- Dizem os pais onde estão. 

Assim foi a história.

 

 
