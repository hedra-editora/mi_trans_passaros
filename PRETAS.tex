\textbf{O surgimento dos pássaros} \lipsum[1] reúne narrativas que abordam o surgimento de outros elementos
do mundo natural e social dos Yanomami. Os quatis, as cutias e as antas, as queixadas, os cuatás, os beija"-flores, os passarinhos --- que hoje são animais -- viviam todos em grupos como nós, o que segundo o olhar yanomami significa dizer que eles eram humanos. Também é apresentada a famosa narrativa da queda do céu, retomada pelo yanomami Davi Kopenawa em seu discurso, e a história da ploriferação do fogo. 

<<<<<<< HEAD
\textbf{Anne Ballester} nasceu em 1955 na França e viveu por 24 anos com os Yanomami. Enquanto ativista, trabalhou como agente de saúde no combate à malária, além de alfabetizadora em língua yanomami e professora de português para jovens e adultos em posições de liderança indígena. É cofundadora da \textsc{ong} Rios Profundos. 
=======
\textbf{Anne Ballester} nasceu em 1955 na França viveu por vinte e quatro anos com os Yanomami. 
Enquanto ativista, trabalhou como agente de saúde no combate à malária, foi alfabetizadora em língua 
yanomami e professora de português para jovens e adultos em posições de liderança indígena. É cofundadora da \textsc{ong} Rios Profundos. Atuou como tradutora e organizadora dos livros \textit{A árvore dos cantos}, \textit{O surgimento dos pássaros}, \textit{O surgimento da noite} e \textit{Os comedores de terra}, todos incluídos na Coleção Mundo Indígenas.
>>>>>>> 0e69953810ab86a0d6106d43abca1f69a5149325

\textbf{Coleção Mundo Indígena} reúne materiais produzidos com pensadores de diferentes povos indígenas e pessoas que pesquisam, trabalham ou lutam pela garantia de seus direitos. Os livros foram feitos para serem utilizados pelas comunidades envolvidas na sua produção, e por isso uma parte significativa das obras é bilíngue. Esperamos divulgar a imensa diversidade linguística dos povos indígenas no Brasil, que compreende mais de 150 línguas pertencentes a mais de trinta famílias linguísticas.



