\part[O surgimento dos pássaros]{O surgimento dos\break pássaros}

\chapter[O surgimento dos pássaros]{O surgimento dos\break pássaros}
 
\letra{O}{grande} \textit{tuxaua} que dividiu a terra onde nós moramos era um verdadeiro
líder, e morava além da parte central da terra dos
Këpropë.\footnote{Trata-se de outro nome dos próprios Parahiteri, narradores da história.} Esse nome, um nome importante, também é o nome
da região onde ele nos ensinou a morar; é o nome da região onde eles
moravam antigamente.

Nessa terra, ele fazia o \textit{kawaamou} e transmitiu esse ritual; ele os
mandava se reunirem e nos ensinou, assim, a nos
reunirmos em nosso xapono.\footnote{Os xaponos são as casas coletivas circulares onde moram os Yanomami. Cada casa corresponde a uma comunidade; em geral não se fazem duas casas numa mesma localidade.} 

Como se chamava aquele que dividiu a terra, quando, no início, os
Yanomami ainda não praticavam esse ritual; que, em seguida, se tornou
líder e deu a terra? Ele tem nome. Aquele que mostrou
o \textit{kawaamou} se chamava Gavião.

Ele existiu desde o início, esse verdadeiro líder que, depois, chegou
para morar com os poucos líderes que sobreviveram ao dilúvio. Gavião
morava em um xapono e o cunhado dele era o líder do grupo dos
Wãhaawëteri em outro xapono vizinho. Wãhaawë era o nome
dele. Os dois sobreviveram.

A região que ocupou se chamava Wãha, por isso o nome dele era
Wãhaawë. Foram os ancestrais mesmo. Foi ele que dividiu e deu a terra
para nossos antepassados morarem. Aquele que deu a terra aos nossos
antepassados se chamava Wãhaawë. Os moços iguais aos nossos, que moravam
com ele, se chamavam também Wãhaawë. 

Moravam os dois, Gavião e Irara. Chamava-se Irara aquele que ensinou os
Yanomami a atirar com zarabatana.

Nessa terra, quando não havia Yanomami, moravam somente três pessoas.
Era o xapono de Gavião, que bebia a água do rio Porena. O xapono dele se
localizava na beira do rio Porena e bebia sua água. É o nome do rio onde
os dois moravam. Tomavam banho nessa água. A região de Porena era a região
deles. Moravam lá, perto daquele rio. 

Nós, Yanomami, quase que não aprendemos a soprar veneno nos outros. Já
mencionei esses nomes, Irara e Gavião, nomes terríveis que vocês já
ouviram. 

Agora é a história daquele que experimentou soprar. Foram esses dois que
ensinaram àquele feio; Irara já conhecia o veneno, pois ele o possuía. 

Depois de ter observado Irara soprar veneno, o feio aprendeu e logo
soprou em Mel. Experimentou nele; aquele feio experimentou em Mel pela
primeira vez, quando o ato de matar não existia e quando ninguém
morria. 

Aquele antepassado chamado Irara recebeu o nome de Nokahorateri, o que
entende de veneno; o grupo dele também passou a se chamar assim. 

O feio quis atirar logo em Mel, pois pediu veneno a Irara. Ele o pegou,
quebrou um pedaço de semente para experimentá-lo imediatamente. Esta é a
história do feio que soprou pela primeira vez o veneno. Não são os
nossos antepassados, são outros! 

Havia um líder muito bonito chamado Mel. Era nobre e muito cheiroso:
onde ele passava, deixava seu perfume, pois o cheiro dele era como o dos
cabelos das mulheres que os Yanomami acham tão bonitas que desejam
ir aos xaponos delas para se casarem.

Duas mulheres chegaram e viram Mel roçando, as duas ficaram sentadas. Ele
mostrou como roçar, ele abriu a roça. Saíram do xapono delas e andaram uma
atrás da outra, até o xapono dele, só para vê-lo. 

--- Olha só, como esse homem é bonito! 

As duas pensavam isso por causa dos seus cabelos bem macios e
cheirosos. Hɨ̃a!, fizeram as duas cheirando. Toda a floresta
estava cheirosa, o perfume dele se espalhava. 

No mesmo xapono, morava Mucura, o feio, que não cheirava tão bem.\footnote{Mucura é um nome amazônico para o gambá.} Ao avistar as duas mulheres, cheias de desejo e se
dirigindo apressadas à casa de Mel, Mucura se zangou. Ele era mesquinho
e, assim, ensinará os feios a matar os bonitos. Indiretamente, as
duas mulheres entregaram Mel à morte. 

Quando apareceram as duas mulheres, passando na frente da casa do feio,
ele estava deitado no chão. \textit{Xiri}! \textit{Kuxuha}! \textit{Krɨhɨ}! Mucura bateu o chão com o pé para chamar a atenção das duas e, quando elas olharam, o feio fingiu cair no chão para elas cuidarem dele.

Apesar de ser fedorento, de estar sempre com olhos purgando e cheio de
feridas, quando ele viu as duas mulheres se abraçando, ele bestamente
pensou:

--- Elas chegaram para mim, minhas duas mulheres apareceram, gosto
delas. 

Ao perceber que o olhar das duas mulheres se dirigia à rede bonita de
Mel, ele se zangou mais ainda. 

--- Estou aqui! --- disse ele, mas sem efeito, pois elas não vieram por
sua causa.

A rede estava amarrada lá do outro lado, em um espaço livre, e as
franjas de sua rede vermelha, cor de sangue, se balançavam levemente, a
rede estava logo ali. 

--- É essa daí, lá mesmo, lá está a rede dele! --- disse uma das duas
mulheres, em pé de frente para a casa de Mucura. 

Enquanto isso, Mucura fingia fazer o \textit{wayamou}, apesar de não saber
fazê-lo:\footnote{O \textit{wayamou} é um diálogo cerimonial realizado à noite por um hóspede e um anfitrião por ocasião de uma visita, destinado reforçar ou restabelecer relações pacíficas entre dois xaponos. Mucura não está fazendo o \textit{wayamou} do modo correto.} 

--- \textit{Aë, aë, aë}! Minha mãe, a comida que eu guardo no moqueador,
pegue a carne e dê às minhas duas esposas que chegaram, dê a elas
agora. 

Apesar de não serem esposas dele, ele falou assim. 

Ele deu um pedaço de carne da bunda dele às duas mulheres, que olharam
de soslaio e fugiram, por causa do fedor. 

\textit{Kuxu}!, elas fizeram, cuspindo, \textit{kuxu}!

O fedor afastou as duas, que foram em direção da rede de Mel, que estava um pouco mais adiante. Elas foram lá, deitar-se juntas na rede
dele. Ao chegar em casa, Mel pensou, \textit{hɨ̃ɨ}!, e perguntou:

--- O que vocês duas estão fazendo? Não havia ninguém como vocês antes,
como vocês duas chegaram?

--- Viemos por sua causa! Venha! Venha! 

Agarraram-no e o deitaram em cima delas. Deitaram uma de cada lado para
cheirá-lo. O feio, que fazia o \textit{wayamou}, viu isso e
chorou. 

Ele se abaixou e as moscas seguiam seu fedor. Ele estava cheio de
ura,\footnote{Isto é, \textit{bernes}.} as pálpebras estavam cheias de ovos de mosca.
Quase cego pela ira, ele rapidamente saiu, pois aqueles que guardavam o
veneno, Irara e Gavião, moravam ali perto. Ele pediu: 

--- Me deem o veneno! Vou experimentar! Quero magoar as duas mulheres
que chegaram e me fizeram sofrer. É verdade o que estou dizendo! Vou usar agora! Dêem um pedaço! --- disse ele. 

Eles logo responderam:

--- Então pegue o meu! Experimente com este! Experimente! Tente! 

--- Mas com quem? 

--- Não digam que não sabem! Vocês já conhecem aquele feio, Mel! Sabem
como ele não presta! É o nome daquele que não presta! 

Apesar de ser menos que ordinário, de ser careca, apesar de ter um rosto
feíssimo, ele disse isso! O nariz dele era horrível, e mesmo assim ele
disse isso. O olhar dele era horroroso, e mesmo assim, ele disse isso. 

--- Verdade, mesmo? 

--- Sim! Verdade! Vou experimentar hoje!

--- Se você diz isso, então tome! 

Quando voltou, ele rapidamente se preparou. 

Querendo impedi-lo, a mãe de Mucura falou: 

--- Querido! O que você está querendo fazer? Deixe para lá! Deixe para
lá! --- disse ela. --- Cuidado, querido! O que você está querendo? O que
você está pretendendo fazer? Cuidado! Não vai, não, não vai! Fique
quieto! --- disse ela. --- Não vá lá de novo, com Irara e Gavião! As
mulheres são de Mel! São mulheres dele! Não pense que elas vieram para você!
--- disse a mãe. --- Elas fecharam o nariz por causa do seu fedor!
--- disse.

Pois seu fedor chegava longe.

--- Não me pergunte o porquê! Eu vou! 

Mel havia voltado para derrubar as árvores grandes. Mucura já queria matar
Mel. 

Mucura correu e logo foi, enquanto zoava a queda das grandes
árvores. \textit{Kou}! \textit{Kou}! \textit{Kou}! \textit{Kou}! 

--- \textit{Ãaaaaaõoooo}! \textit{Aë, aë, aë, aëëëëëëëëëë}! --- estrondava o eco da voz bonita de Mel. 

Para experimentar a força do veneno, Mucura o experimentou primeiro em
Lagartixa. As lagartixas, que ficam grudadas aos jutaís, aquelas que
sempre sobem. Mucura foi em direção de Lagartixa, enquanto Mel
derrubava. 

Onde se erguia um jutaí, Lagartixa subia, fazendo \textit{tararararara}! 

--- \textit{Kuxu}! \textit{Kuxu}! \textit{Kuxu}! Vou soprar naquele mesmo! Eu vou experimentar a força do veneno! --- disse Mucura. 

Primeiro ele soprou Lagartixa. Soprou. Pegou de raspão na garganta de
Lagartixa. \textit{Paha}! E uma última vez, \textit{paha}! \textit{Paiii}!

Com isso, ele o feriu, e onde arrancou um pedaço da garganta, deixou
vermelhas as gargantas das lagartixas. Mucura ficou olhando, para ver
Lagartixa passar. 

O veneno não agiu. O veneno era fraco. Somente as folhas de jutaí,
tontas de veneno, caíram. Lagartixa, que Mucura havia soprado, subiu
mais acima. Mesmo tonto, ele não caiu. Ele se recuperou. 

--- Puxa! Não faz isso comigo! 

Mucura não gostou de ele ter resistido ao veneno. Lagartixa sumiu, já
estava em outro lugar. Mucura, depois de soprar o veneno que tocou
levemente a garganta de Lagartixa, foi em direção ao bonito Mel, que
trabalhava e não aguentaria a força do veneno: o fôlego dele não seria
tão forte. 

Mel estava virado em cima do andaime. \textit{Paha}! \textit{Paha}! Ele se
desequilibrou. Aquele que foi soprado antes pegou um atalho e foi falar
a Mel: 

--- Tome cuidado! Ele me soprou! Aquele feio fedorento quis primeiro
experimentar a força do veneno comigo! Cuidado, fique atento! Ele vem em
sua direção! --- disse ele. 

--- \textit{Hɨ̃ɨɨ}! A força do veneno não me destruirá! O feio não me
pegará! --- Mel disse. 

O feio já estava perto.

--- Ele está me desafiando! --- disse Mucura. 

Mel virou, enquanto derrubava com o machado. \textit{Paha}! \textit{Paha}! \textit{Paha}!

Mucura simplesmente soprou; você não aguentaria muito tempo a força
desse veneno, não dá para aguentar. 

Mucura foi logo embora e Mel caiu, tonto. Ele caiu. Corre, corre! Aquele
que tinha caído voltou para suas duas mulheres. Quando ele chegou, não
demorou…

 --- Estou com muito frio! Estou morrendo de frio! --- chegou em casa,
delirando, tonto. --- Faça fogo para mim! --- disse ele. 

Ele já estava morrendo. Enquanto as duas cuidavam do marido,
esfregando-o, ele caiu morto. 

Ensinando o choro aos Yanomami, elas duas logo choraram, choraram,
pensando nele. As duas se abraçavam, chorando, se segurando pela mão,
chorando. Onde estava o marido morto, as duas mulheres avançavam e
recuavam.\footnote{Trata-se de um movimento da dança feita em ocasiões fúnebres.} Aí o feio se juntou, aquele que havia
soprado fingia chorar, de medo fingia chorar, e disse assim: 

--- Meu grande e querido sobrinho! Meu sobrinho, mesmo! Mataram meu
sobrinho! Mataram meu irmão! Puxa vida! Reapareça para elas! ---
fingia. 

Enquanto falava assim, as duas mulheres fugiram, não queriam escutar.
Ele as seguiu. Enquanto ambas as mulheres rodavam ao redor do xapono,
chorando, ele falava, fingindo chorar; rodeava ao redor do
xapono, dissimulando, e as duas mulheres choravam, fugiam dele de novo,
pois estavam bem zangadas. Apesar de ele chorar, as lágrimas não saíam,
ele simplesmente fingia. 

Ele fugiu. As duas mulheres cremaram o corpo de Mel e, enquanto o
cremavam, o feio fugiu para se esconder, como se fosse escapar: fugiu
pensando poder se esconder. 

Depois da fuga dele, e ensinando os Yanomami como se mata, como se segue os
rastros, eles seguiram logo os rastros. Procuraram os rastros.
Procuraram os rastros. Procuraram. Fizeram isso logo, e não perderam os
rastros. 

--- Foi o feio que o matou! --- diziam. --- Foi mesmo o feio, o feio
fugiu! Agora é a vez dele! --- esbravejavam. --- Nunca sobreviverá,
criminoso! Vamos dar o troco! Acabou com nosso líder, então vamos
matá-lo! --- disseram, e seguiram os rastros. 

As formigas Tokonari, as formigas Xĩriana, as
formigas Mamisipreima e as saúvas se mexeram, seguindo os
rastros. Os passarinhos também seguiram os rastros. 

Eles chamaram {Resimaritawë}, que seguiu os rastros como cachorro,
pelo cheiro; eles passaram levando Resimaritawë.\footnote{No dicionário de Lizot consta tratar-se do ancestral do minhocuçu, um anelídeo grande, \textit{heresima}. Como o termo não coincide exatamente, mantemos o nome original.} Não conseguiam encontrar os rastros onde havia pedras, onde havia a
pedra \textit{maharixitoma}, porque os rastros já tinham sumido. 

Naquele lugar, onde Mucura havia matado Mel, os seus rastros estavam
dando voltas, sumindo no meio de um pântano. Ele fugiu e estava a uma
distância igual àquela que nos separa do rio Maupuuwei, aonde
vamos {caçar em grupo}. 

Mucura se escondia na montanha, ele foi lá em cima, porque queria
escapar. Ele subiu em uma árvore. A montanha era redonda como um jutaí,
ele entrou lá, onde a montanha tinha uma fenda. Pretendia se trancar
ali. Eles derrubariam a montanha para pegá-lo. Mucura não entrou no
jutaí. Ele entrou nessa montanha, porque queria fugir. 

Como criminoso, ele se arranhava, mas não com as
unhas.\footnote{A pessoa que assassina alguém costuma se arranhar com as unhas,
castigando-se. Mucura é mau, então não chega a se machucar.}  Adormeceu. Criminoso. Ele se admirava, porque
matou. Ele ensinou a ser criminoso. Ele perpetuou o crime. Ele estava
dormindo. 

Resimaritawë seguiu o fio de Mucura, e ficou escutando. 

--- Aqui! Ele está dormindo --- pensou.

Como criminoso, Mucura guardava um tipo de canudo no braço, guardava um
em cima da orelha. \textit{Õooo, õooo, õooo}! Ouvia-se a respiração de
Mucura e Resimaritawë escutava:

--- \textit{Õooo, õooo, õooo}! Mel! Mel! Você viu como é bom? Foi isso que
fiz para você! --- roncava Mucura. 

Apesar de estar roncando, ele se gabava. 

--- Está vendo? Foi o que fiz pra você! Mel, foi o que fiz para você!
--- dizia, roncando.

Ouvindo isso, Resimaritawë logo se assustou e gritou: 

--- \textit{Hɨ̃aaaaaaaaaëë, aaaaaaëë}, o Mucura fedorento, aqui, o feio em
pessoa está se gabando bem na minha frente, \textit{aaaaaë}! --- disse ele. 

Esse abrigo estava na base da montanha, que vocês não conseguiriam
derrubar rapidamente, mas eles trabalharam como loucos e causaram um
impacto incrível no cume da montanha. Não tinham terçados, mas mesmo
assim conseguiram matar Mucura. Apesar de não possuírem terçados como os
dos \textit{napë,} eles conseguiram matá-lo. 

Chamaram os do grupo dos tucanos Parawari, porque o grupo das
Maitacas não conseguiam. Os Araris estavam tendo dificuldades com seus
machados de pedra, que se destruíam. Apesar dos machadinhos
dos Tokorari, todos sofriam por causa das ferramentas, que se
quebravam em pedaços e não entravam na pedra. Chamaram os do grupo
Parawari, que viviam agrupados na baixada. 

Devido ao que fez o feio ao seu sogro, o bonito Mel, seu genro Parawari
disse:

--- Vamos! Vão chamar meus pais, que moram com meus avós: eles moram bem
perto e têm verdadeiros facões! 

--- Vai, querido, corra! Vai você! Vai buscar! Vai buscar! --- diziam
assim. 

Eles os buscaram. Eles chegaram e atacaram a montanha. Não queriam
destruir o facão deles. Como outros facões haviam sido destruídos, os
pedaços das lâminas estavam espalhados no chão. Para poupar esforços
inúteis, eles amoleceram a parte interna da montanha, como se fosse uma
árvore, com a força do pensamento. 

\textit{Kraxi}! \textit{Kraxi}! \textit{Kraxi}! Depois de amolecer a pedra, derrubaram uma
parte. \textit{Kraxi}, \textit{kraxi}, \textit{kraxi}! Atacaram logo de todos os lados para
fazerem voar lascas de pedra. A montanha era do tamanho de uma sumaúma.
Fizeram outro buraco grande, para poderem continuar com a destruição da
pedra. O buraco no tronco se aprofundava; fizeram uma espécie de cratera
na pedra. 

--- Vejam essa montanha! --- disseram os Parawari. 

Conseguiram fazer esse buraco porque os Parawari têm esse bico mais
comprido, e que será mais comprido para sempre. Os terçados, sendo
mais compridos, começaram a derrubar a montanha. O bico do tucano
empoleirado é, na verdade, seu terçado. Ele não nos morde, quando olhamos
para ele? \textit{Kraxi, kraxi, kraxi}! Foi nesse momento que começaram a
fazer a montanha balançar. Não demorou. \textit{Kru tu tu tu tu tu}!
Pedacinhos de pedra voavam e pulavam; os pedaços se espalharam no
chão. \textit{Huãaaaaaa}! Eram muitos, e estavam tristes por estragar seus
terçados, pois o bico deles é curto. Era isso mesmo: o trabalho começava
a ser realizado. \textit{Kru tu tu tu tu}! Continuavam trabalhando.

Só faltava o coração da montanha. Os terçados sendo curtos, esse
pedacinho ainda resistia. Apesar de pequeno, o tronco da montanha não
quebrou rapidamente.

--- \textit{Pei kë}! \textit{Aaaaaaaaooooo}! Vamos! 

Apesar de o tronco estar quase torado, a montanha não estava se
mexendo. 

--- \textit{Aaaaaëëëë}! Vai você! A ponta da montanha vai nessa direção
--- disseram a Preguiça. 

Queriam que ele puxasse a ponta da montanha, pois a montanha não caía.
Preguiça esticou um fio flexível e puxava a montanha. Os terçados dos
Parawari, cortando a pedra, não atingiam o núcleo. 

Preguiça esticou uma espécie de fio. Ele nem pensou: --- Essa montanha vai
me machucar! --- Ele não tem costume de ficar com medo. Com o fio,
parecido com linha de pesca, ele puxou o cume da montanha. Não havia
nada para conter a pedra. Preguiça estava sozinho, sem apoio, mas também
sem medo. \textit{Ku tu tu tu tu tuuuuuuu}! A pedra começou a estourar,
fazendo um barulho enorme; parecia cair um pedaço grande de céu. 

Mucura estava preocupado e chorava, olhando para baixo com lágrimas
escorrendo. Ele estava desesperado. 

\textit{Tuuuuuuuuuuuu, tẽẽẽrërërërë}! Preguiça caiu mais à frente, para
não acabar embaixo da pedra, que caía. \textit{Ãaaaaaa}! O fio o
impulsionou e ele foi cair bem longe, ele se engatou lá longe com suas
garras. A pedra, quebrando e levando árvores, não alcançou a árvore onde
Preguiça se agarrou; se ficasse pendurado mais perto, ele se
machucaria. 

Eles destruíram Mucura. 

Todos os animais, as araras, os tucanos, os urus, os inambus, os mutuns,
os jacus, os urumutuns, os mutuns-de-traseiro-vermelho e os jacamins
eram gente. 

Todos juntos, pegaram aquele que foi destruído. 

Chegaram até o sangue de Mucura derramado no chão para se pintarem. As
pequenas maitacas amarelas se pintaram com seu sangue cinzento. Todos os
passarinhos são diferentes: uns são vermelhos, outros cinzentos, outros
têm pálpebras cinzentas. As cores dos pássaros vêm daquele momento,
quando se transformaram em animais, naquele mesmo lugar, com o sangue
derramado. Transformaram-se onde havia o sangue derramado. 

Aqueles que foram destruir Mucura não voltaram para seus xaponos, onde
havia roças. Não voltaram para recuperá-las. De tanto se pintarem com o
sangue derramado, ele acabou e, depois de terminarem, logo voaram. Logo
se transformaram. Eles ocuparam a floresta toda, não restou nenhum
espaço. 

Outros pintaram de preto o seu peito, passavam um pouco de sangue na
garganta, outros pintaram as pálpebras, outros, os cabelos, outros
pintaram o cume da cabeça; outros se pintaram de cor cinzenta,
derramaram os miolos e os excrementos com os quais se pintaram. Assim
fizeram. 

Preguiça, que puxou a pedra, passou no seu corpo os excrementos, por
isso ele tem cor cinzenta, é a cor dos excrementos de Mucura. Ele também
passou sangue na bunda, levemente. 

Assim, depois de esgotarem o sangue de Mucura, eles logo voaram, e todos
sumiram. 

Termina assim essa história, porque eles voaram, se transformaram e
partiram. Onde aconteceu essa história, outra segue. Contamos o que
aconteceu a Preguiça e Mucura, cujo esconderijo foi derrubado.

A região onde eles moravam tem um nome. Aquele homem bonito, que foi
morto, morava à frente da serra Moyenapɨwei. Ele nasceu, morava à frente
da serra Moyenapɨwei.  Essa serra se chamava Moyena.
Ao pé dessa serra, Mel abriu roças. Por isso, se chamaram assim, pois
ocupavam a região desse nome. 

Apesar de ser uma serra, chamava-se assim. Tinha esse nome, pois era uma
região bonita; quando as palmeiras \textit{moyena} floresciam, as exalações
tomavam toda a floresta. Ainda existe o perfume onde ficava sua moradia,
e seus descendentes ainda moram lá. Chamam-se Moyenapɨweiteri. Ficaram
morando lá, pois os antepassados se chamavam assim. 

Nessa região central, morava também Jacaré. Jacaré morava nessa região
central, onde Mucura matou Mel.

\chapter{Naroriwë}

\letra{Ɨ}{hɨ} të pë rë kuaanowei, hei pata a yai rë kui a urihi rë wëyënowei,
urihi a rë kui, urihi kamiyë pëma kɨ rë përɨaɨwei a rë wëyënowei, pata
përɨamɨ yai, përɨamɨ a rë kuonowei, Këpropë kɨ urihi mɨ amo hamɨ he tore
ha a kuoma. Pei a wãha rë kui, pata yai, pata pë rë përɨaɨ rë hira
hërɨɨwei, pë rë përɨanowei, a wãha urihi, a wãha kua. 

Urihi a rë kui, a rë hẽreamonowei, kawãa të rë takenowei, kawaamou pëma
të tapë kurenaha, a hirapë, a rë wëkenowei, hapa Yanomamɨ të pënɨ hẽrea
a pou maopehe tëhë, a pou mao ma makuhei, a patamoa rë notirayonowei,
ɨ̃hɨnɨ pata urihi a rë hipënowei, weti naha a wãha kuoma? A wãha kua:
Yanomamɨ të pë hẽrea rë hipëkenowei Koparikesi wãha kuoma. 

Hapa Koparisi yai përɨkema, përɨamɨ yai, ɨ̃hɨnɨ të pë kãi notikema. Pata
të pë rë pakakumanowei, të pë wãisiapɨ rë hëprarionowei, ɨ̃hɨ Koparisinɨ
të pë kãi përɨkema. Hei Koparisi rë kui hei a përɨa, hei Koparisi, hei
heri e rë kui përɨamɨ Wãhaawëteri e kuoma. Ɨhɨ a wãha rii kuoma
përɨamɨ. 

Pe heri xo kɨ rë hëpɨprarionowei, kama urihi Wãha rë yukenowei Wãhaawë a
wãha kuoma. Pëma kɨ no patama rë përɨamaɨwei a urihi rë hipënowei, hapa
pata kɨ yai. Kamiyë yama kɨ no patapɨ yai, hei kamiyë yama kɨ no patapɨ
iha a urihi rë hipëke hërɨnowei, Wãhaawë a wãha kuoma. Hei ipa huya pë rë
kurenaha pë kãi rë përɨonowei, kama Wãhaawëteri e pë wãha kuoma. 

Hei kama Kopari, Hoari xo kɨ rii përɨpɨoma. Hoariwë a wãha kuoma.
Yanomamɨ të pë horaɨ rë hiranowei ɨ̃ha a kuoma. 

Urihi a hamɨ Yanomamɨ të pë përɨaɨ mao tëhë, hei pë përɨoma, hei, hei,
hei. Ɨnaha pë kuoma. Hei Koparisi yahipɨ, hei kee xapono. Koparisinɨ
Porena u koama. Porena u ha xaponopɨ praoma. Ɨhɨ u koama. Kɨ rë
përɨpɨonowei u wãha. Ɨhɨ u yaruama. Porena urihi, kama urihipɨ Porena a
wãha rë kuonowei a yupoma. Ɨha pë përɨoma, ɨ̃hɨ u ha, hei pe heri pei a
urihi yai rë huwëponowei, mɨ amo yai hamɨ. 

Yanomamɨ të pë horayopë, ɨ̃naha të mori kuaanomi makui, hei ya wãha
yuprarema. Hoarinɨ, Koparisinɨ a kãi rë përɨkenowei, wãritiwë pëma të
wãha rë hiripouwei. 

Hapa totihi të mixiã rë wapanowei, ɨ̃hɨ të wãti yaia, ei kɨpɨnɨ të ha
hirakɨnɨ, a hiraɨ, a hiraa xoakema, ɨ̃hɨnɨ. Të xi wãrihiwë rë taɨwehei,
të xi wãrihiwë xomi rë taɨwehei, hei Hoarinɨ të yai tama, hɨrɨ a yai rë
pore. 

Ɨhɨ iha të kɨ taɨ ha tararɨnɨ, hei wãriti të rë hiranowei, ai a horaɨ
xoaoma, Yamonamowãro a rë horanowei, a rë wapanowei, wãriti tënɨ a
wapama, hapa, ai xëpraɨ kuo mao tëhë, ai të nomaɨkuo mao tëhë. 

Pata Hoari a wãha rë kuonowei, pata Nokahorateri a wãha yai rë
yukenowehei, Nokahorateri pë kuprou xoarayoma. 

Ɨhɨ tënɨ a horaɨ xoaoma, ɨ̃hɨ iha e kɨ nakarema. E yuaɨ xoarema, e mo ko
wai ha huyetirënɨ e wapaɨ xoaoma. Wãriti tënɨ a rë horanowei të rii. Hei
kamiyë yama kɨ no patama rë kui mai! Ai rii. Ai të rii. 

Ɨhɨ katehe a rë përɨkenowei, përɨamɨ a kuoma, Yamonamowãro. A no xi
hiraarema. Hei kamiyë kurenaha mai, mɨ amo hamɨ amoamo a yai kuoma. A rĩ
yai kẽterio totihioma, wa henakɨ hɨprɨo, wa hɨprɨo, wa huɨ ha, wa rĩ
warou rë kurenaha, a rĩ kuoma. Pei henakɨ rĩ, kuwë yaro, ɨ̃ha suwë
Yanomamɨ të pë ha riëhëaprahenɨ, ɨ̃naha të pë napë kõo kuo puhiohe yaro.
Ɨhɨ iha Yanomamo wãro iha të pë hirakemahe. Hirano të taprapë. 

Suwë kɨpɨ waropɨkema, a ha tapɨrarɨnɨ. Ɨhɨ a rë hikarimore ha, ɨ̃ha kɨ rë
ropɨawei, kama kɨ kupɨopë ha, a kuaaɨ yaro. Të pë hikarimou hiraɨ ha,
hikari a tama. Ɨhɨ kama kɨ xaponopɨ prapɨopë ha, kɨ ropɨopë ha,
kɨpɨkororoa xoakema. A tapɨrarema. A mɨpɨma. 

--- Kurahë wãro a riëhëwë rë totihi! 

Kɨpɨ puhi kupɨtarioma, henakɨ si yai ëpëhëoma, ɨ̃hɨ pei henakɨ rĩ ha.
Hĩa! Kɨ puhi kupɨɨ xoaoma, urihi a rĩ pata kẽteri haikiwë ha, a rĩ
hirakaama. 

Ɨnaha a rĩ kuwë ha, kama a rĩ kuwëmi makui. 

Naroriwë a rĩ kuwëmi makui, a ha huxutarunɨ, suwë kɨ no wayuawë kõpei
yaro, kɨ no xi harirapɨwë totihiwë ha, a huxutarioma. Të pë huxutou
hiraɨ ha, a nohi ohotatanomi, wãriti të pë makui Yanomamɨ totihitawë të
pë xëpehe, huxuo hiraɨ ha. A napë rë rurupɨkenowei të rii. 

Suwë kɨpɨ pëpɨtario ha, hei a rë kurenaha wãriti të rëpraoma. Të praopë
ha, suwë kɨpɨ hapɨa katitirayoma. \textit{Xiri! Kuxuha! Krɨhɨ!} Mamikɨ të marë
tamaɨwei, ɨ̃hɨ të ha krɨhɨmonɨ, suwë kɨpɨ mamo xatiprakema. 

Të rë wãritii, të kãi xomi kea nokarayoma. Kama ihamɨ kɨpɨmi makui,
mohoti katehe a yai ihamɨ kɨpɨ kõpɨoma kure ha. 

--- Kamiyë iha kɨ kõpɨoma, ipa suwë rë kɨ pëpɨtario ë, ya kɨ no xi
hirapɨaɨ! 

A puhi xomi ha kunɨ, a rĩ hĩtari makui, a kãi krĩhipɨ, a kãi warapisi
makui, a kuaama, a huxutou puhio yaro, suwë kɨpɨ hãkɨkopɨma, ɨ̃hɨ naha
kɨpɨ kupɨaɨ ha, e kea nokarayoma. 

--- Hëyëmɨ, hei kë ya! --- e xomi kupɨma, kama ihamɨ mi makui, a nohi rë
yaipɨpore pëkɨ no aihiwë yawëtëtao horayoma, yĩɨ e kɨ kasi kɨ ĩyë wakë
xurixurimoma, kuopë hamɨ: 

--- Kihi yai, kihi pëkɨ yai! --- e kɨ kupɨɨ kãi upraoma. 

Kuɨ tëhë, e xomi wãyamoma, wãyamou taomi makui, e xomi wãyamou kãi
kepɨrayoma: 

--- \textit{Ae, ae, ae}, napemi, hei nii ipa ya, ya e kɨ, kɨ rë ma, makepo, pore
a wa yãhi ha yurɨnɨ, hei ipa suwë kɨ, kɨ rë kõpɨpohe ha yãhi ta hipëaɨ
ayaonɨ a --- Kama ihamɨ mi makui a xomi kuma. 

Kama posi moko kero, kero. Ɨnaha të kuɨ ha, kɨpɨ mamo ha axëpraikunɨ pei
të rĩ ha kɨ rupɨrayo hërɨma, kɨpɨ mamo ha axëpɨpraikunɨ:

--- \textit{Kuxu!} --- kahi u këxëɨprao hërɨma. --- \textit{Kuxu!} --- kɨ kupɨ hërɨm., 

Kɨ yaxupɨrema, kihi Yamonamowãro pëkɨ hamɨ kɨpɨ katitipɨke herayoma.
Kama e nahi yawëtëa yaro, kɨ ha ukupii kuhurunɨ kɨ yakapɨpari hirayoma.
Kama a rë kui a kõpema. 

--- \textit{Hɨ̃ɨɨ!} --- a puhi kutarioma. --- Exi wahë të taɨ? Ɨnaha kuwë të kɨ
kupɨonomi! Weti ihamɨ wahë kɨ huimaɨ kuhe?

--- Kahë ihamɨ! Kahë pë napë kõo. Pë tararema yaro, a ta hapo! 

A hurihipɨa mɨ parema. Pei kɨpɨ parɨkɨ ha a makepɨpoma. Pei a rĩ ha a mɨ
hetutupɨrema. Kuaaɨ ha, wãriti të rë wayamore, të no preaama, a ɨ̃kɨma. 

A mohe poarioma, moo pë pata makui kama a rĩ hĩtari hamɨ, a kãi
motepɨoma, mamo kasi pë motepɨmou haikioma. Kuwë yaro a huxutarioma,
rope a hurayo hërɨma, a no yai rë hërɨpɨ hiraɨwei a ahetea yaro, a rërëa
xoakema, a pehi yuaɨ ha, e rope ha nakarënɨ a ha rërëikunɨ:

--- Pei, hërɨ a ta hio! Ya ta wapa, kamiyënɨ! Suwë kɨ hore kõpɨpohe ya
kɨ a no ta prepɨmapo! Ware ã no prepɨma kɨhë! Peheti re ya wã! Hei kuikë
rë ya wapaɨ xoao! Ãtahu kɨ ta hio! --- e kuma. 

Si ihehewë a wã kãi hupɨrema. 

--- Pei, kutaenɨ ipa a ta yurë! Hei anɨ, a ta wapa! A ta wapa! Wapëpraa!
Weti kë? e kɨ kupɨma.

--- Weti mai! Wahë a nohi taɨ waikire! Yamonamowãro! Wahë pë xami taɨ
waikire! Yamonamowãro pë wãha xami rë kure! 

Kama a kuhupɨmi makui të kuma, a maxixipɨ yaro mohekɨ wãritiwë makui, a
kuma. Hei pei he rë kui ha, pëma kɨ henakɨ rë kurenaha henakɨ kuwëmi
makui, he tapramou wãritiwë makui, a kuma, hũxipë kãi wãritiwë makui, a
kuma, mamo xatio kãi wãritiwë he parohowë makui, a kuma. 

--- Peheti rë kë?

--- Awei, peheti hei kuikë ya wapaɨ! 

--- Wã haɨ yaio, pei yurë hërɨ! 

E ha kõponɨ a xurukou xoa peheroma. 

Naroriwë nɨɨpɨ e ã hama, pë nɨɨnɨ a hore wasima: 

--- Xei! Exi të ha wa xurukou kure? Kuo pëtao! Kuo pëtao! --- e hore
kuma --- Mihamɨ, xei, exi wa të taɨ? Exi wa të taapraraɨ ayaa kure?
Mihamɨ! A hurɨhë, a huɨ ta yanɨkɨ taru! --- e hore kuma --- A hua
kõrɨhë. Kama të pë hesiopɨ! Kama të pë! Kamiyë ihamɨ të pë tao puhi kuɨ
ayao tihë! --- e kuma, pë nɨɨ --- E kɨpɨ hũkakɨ kahuaɨ ha! --- e kuma. 

Praha makui a rĩ warou yaro. 

--- Ɨhɨ exi të ha mai! Ma! Ya huɨ! 

Kihi a rë kayapamo rahari ha, kiha a haɨpraɨ puhiopë yaro. A haɨprapë! E
rërëkema. E hua xoarayo hërɨma, kayapa hi kɨ hõra pë ramapou tëhë. \textit{Kou! Kou! Kou! Kou!} 

--- \textit{Ãaaaaaõoooo! Aë, aë, aë, aëëëëëëëëëë!} Katehe e a
karëtoma. Të kɨ wayu wapaɨ ha, reha a wapama. 

Ãroko hi pë hamɨ reha pë marë sutiowei, reha pë marë tuo xĩroaɨwei, 
ɨ̃hɨ a ha ukuikunɨ, a hõra përao tëhë. 

Ãroko hi pata rë upraawei iha:

--- \textit{Tarararararara!} --- reha e kupe hërɨma. 

--- \textit{Kuxu! Kuxu! Kuxu!} Ya wapaɨ ta yaio kë. Ya hamiwë ta wapaxo! --- e
kuma. Ɨhɨ reha rë a horaɨ parɨoma. A horama. Ɨhɨ rehanɨ: 

--- \textit{Paha!} Hɨtɨtɨ anɨ! \textit{Paha! Paiii!} 

Õramisi yoamarema. Ɨhɨ re e uno. Ɨhɨ rë õramisi rë tɨhɨyëprarenowei
hamɨ, ɨ̃hɨ rë pë õramisi kuprawë, a yuo mɨma. Ɨhɨnɨ e kɨ hamianomi. E kɨ
okearema. Ãroko hi rë kui, e ko hi henakɨ porepɨ kea tahiarayoma. Ɨhɨ
kama reha a rë kui a rë horaɨwei e torere pe hërɨma. E nomaɨ makui, e
kenomi. Ɨha kama rë a harorayoma. 

--- A no huxuaɨ tikooma ta yaitanɨ --- e matarioma, reha. 

Yai ha e kua yaro. Ɨhɨ a ha taakɨnɨ, õramisi ha yoamarënɨ, katehe a yai
hikarimopë ha e ukua piyëkema. E kɨ no horawëapraɨ rë mai! Mixiã tiremaɨ
rë mai. 

A rerekeaprarou tëhë, \textit{Paha! Paha!} Ɨha rë e rë yutuprore, hei reha a rë
horare, e ã hama. Ɨha rë hei rë e rë kui a rë horaɨwei a hamirayou yaro,
e he tiherikema. 

--- Pei, a ta moyawëiku! Ware a hore horahe! Naro a rĩ wãriti rë
hĩtariweinɨ, ware a hore wapaɨ parɨohe, miha kahë a ta moyawëiku! Kahë
wa napë hore huimahe --- e kuma. 

--- \textit{Hɨ̃ɨɨ!} Ya yuo mai kë të! Wãriti anɨ ware a waɨ mai kë të! ---
Yamonamowãro e kuma. 

A wãti kua ma waikirayoi ha. 

--- A wã no hore huxuoma ta yai tanɨ ɨɨɨ!

A kuu tëhë, a rerekeaprarou tëhë, hãyokoma a tuyë taɨ tëhë: \textit{Paha! Paha!
Paha!} 

A ɨnaharë, ɨnaha e kɨ horaa takema, wa no no tetepɨ ma rë mai, ɨhɨ
naxomi të kɨnɨ. 

Të pë no tetemaɨ tao marë mahei, ɨha e xi wãrihiprou xoarayoma,
wãrihiprarunɨ, e porepɨ kerayoma, a ha kerɨnɨ: Sarai! Hesiopɨ kɨpɨ
ihamɨ, ɨha rë a keheropë, a no tetenomi, ɨ̃hɨ e ha kõtaponɨ:

--- Pei! Hĩa ya rë saihia tikorɨhe ë, ya rë saihirɨhe ë! --- e kuuroma,
porepɨ. --- Kaɨyë wakë ta yëpɨpa xë ɨ̃ɨ --- e kuma. 

A maɨ waikia hërɨa yaro. A hore huripɨɨ mɨ paa, huripɨɨ tëhë e
kepɨrayoma. 

Ɨhɨ Yanomamɨ të pë ɨ̃kɨɨ hiraɨ ha, e kɨɨ̃kɨpɨma, kɨɨ̃kɨpɨ xoaoma, puhi
wayuyopɨma yaro, puhi mɨrapɨɨ yaro, kɨ hãkɨkɨpɨaprou, kɨ mɨa kãi
ɨpɨhɨpɨaprarou, hẽaropɨ a nomawë përɨopë hamɨ, kɨ tikukupɨaprarou, kɨ
tirurupɨaprarou, ɨ̃ha wãriti e të xomi nikeropɨma, a ma horanowei e xomi
ɨ̃kɨma, e xomi kiriri ɨ̃kɨma, e ã piyëkoma:

--- Õasiwë të pata yaio! Xëtëwë të pata yaio! Xëtëwë a pata xamia përaru
haikë! Asi yai! Ipa e pata pëpɨtou ta kõro! --- a xomi uhuti kuma. 

E kuɨ tëhë kihamɨ e kɨpɨ rurayoma, të ã hiripɨɨ puhiomi yaro. Ɨhamɨ e
xomi tiporepɨa ha kõrɨnɨ, kihamɨ e xomi kuaaɨ kõoi maɨ tëhë, kɨ rupɨa
kõrayoma kihamɨ kɨ rii ɨ̃kɨpɨma, ɨnaha kɨ kupɨama, kɨ huxupɨtarioma yaro,
a ɨ̃kɨɨ makui, mapuu kɨ kãi hanomi, xomi kuu pëoma, a xomi ɨ̃kɨɨ pëoma. 

Kuo tëhë, e ha kuokuopo hërɨnɨ, hesiopɨ kɨpɨnɨ a yapɨpema, yaaɨ tëhë,
wãriti të rë kui a hãtoprario hërɨma, a no tokupɨma marë! A xomi
hãtoprario hërɨma. 

A ha tokurɨnɨ, Yanomamɨ të pë xëpraɨ hiraɨhe ha, të pë mayo nosi pou
hiraɨhe ha, të pë nosi yaupraa xoakema. 

--- Mayo taeiwei, mayo taeiwei, taeiwei! --- kuaaɨ xoatarioma. Mayo nosi
hikepou xoatariomahe --- Ɨhɨ rë a wãritinɨ a xamimararei kuhe. A wãriti
kua marë mai, a wãriti hore ma rë tokurɨhe. --- Pei! Kama rë ã! --- a
noã taɨ xoaomahe. --- Pë unokãi përɨkei ha yamuku! Ɨhɨ pëma a no payeri
tapraɨ xoao. Kamiyë pëma kɨ përɨamɨpɨ hore xëprarɨhe! --- të pë kuma. 

A nosi yure hërɨmahe, a nosi ha yurë hërɨhenɨ, ɨ̃ha a napë kõkaprouhe,
tokonari pënɨ, xĩrianari pënɨ, mamisipreimari pënɨ, koyeri pë kãi, pë
kuaama. Kiritari pë kãi pë nosi yauama. 

Ɨhɨ pei mayo nosi yai rë hikepore, hiima kurenaha hũka kɨ yakëɨ rë mai!,
Resimaretawë a nakaremahe. Resimaritawë a hayuremahe, mayo nosi
katitipraimihe yaro. 

Maa ma pë hamɨ, maa ma pë kurenaha, maharixitoma pë rë kure hamɨ, mayo
tokua waikirayou yaro. Hei a rë xëprare, kihamɨ mayo rë tɨhɨyëmatiiiiii,
wawëri pë pata ha mayo tokurayoma. A rë tokurɨhe, kihi ai të pë rë
heniyomo pohori. Maupuuwei u rë papohori naha, kihi Naroriwë a hitëo
kupohori, pei makɨ pata ha, a heakaprou piyërayoma. 

A tokuu puhiopë yaro. Hii hi ha Naroriwë a tua piyëkema. Ɨnaha
komorekomore ãroko hi kurenaha ɨ̃hɨ pei makɨ pata komore xatiopë ha, a
rukërayo tayoma. A he ruketayoma. Pei makɨ tuyëmapë. Ãroko hi yai hamɨ,
Naroriwë a rukënomi. Ɨhɨ kɨ ha, a he rĩya ha ruonɨ, a rukërayoma. 

Unokãi hixehixemou xoaoma. A mikema. Unokãi. A mɨprou yaro, a xëprarema
yaro. Të pë unokãimou hiraɨ ha, a unokãimoma. Ɨha a mitaoma. 

Ɨha Resimaritawënɨ, kama mananaepɨ hamɨ, mayo nosi rë hikepouwei yɨmɨka
takema. 

--- Hëyëha! --- e puhi kuma, e yɨmɨka takema, a mitakema. 

Kana ã hãkipoma, ai ã huuporanɨ. Õoooo, õooo, õooo! E mixiã kɨ kãi
horeheoma, e yɨmɨka takema. 

--- \textit{Õooo, õooo, õooo!} Yamonamowãro! Yamonamowãro! Wa të oarei kuhe,
ɨ̃naha pë rĩya tapraɨ kuhe! --- e hũhũrua kuma, pei e hũhũrua makui, a
noka hekaɨ ha. --- Ma rë kui! --- Ɨnaha pë rĩya tapraɨ ɨɨɨ, Yamonamowãro
ɨ̃naha pë rĩya tapraɨ ɨɨɨ! 

Hũhũrua kuma, kuɨ ha hĩri tarënɨ, e rarɨa xoarayoma ɨ̃ha, e kirirarioma
pei a wãha, e rarɨa xoarayoma. 

--- \textit{Hɨ̃aaaaaaaaaëë, aaaaaaëë}, hëyëha Naro a rĩ wãti rë hĩtariwei hëyëha
wãriti a wãti makui ware a no ka hore rë heka yahi aaaaaë! --- e kuma
--- Hei yahi a rë kupe, ɨ̃hɨ kihi rë të ma pata koro kumopë ha, wa të no
tupraaɨ haɨo taopɨ rë mai! 

Ɨhɨ pei rë ma pë nanoka rë kuinapë pënɨ ma pë ha karomaɨhe ha, të pë
ãtahu nanoka pata rë taamaɨwehei naha, makɨ nanoka pata taamamahe, pë
siparapɨ kua ma mai rë, a xëpraɨ he yatirayomahe. Ɨhɨ të rë kui, a ha
xëprarɨhenɨ, a rë xëprarenowehei, napë pë mi makui, pë siparapɨ rë
kure. 

Parawari pë ha nakarehenɨ, ãrimari pë no preaaɨ yaro, arari pë makui pë
siparapɨ yahekioprou no preaoma yaro, pë poopɨ yahatomou no preoma,
pokorari pë makui hapa të pë waɨ tipuramaɨ no preomahe, rukëimi ha, pë
nakaremahe. Parawari pënɨ, pë rë hiraawei, pë hiraa rë pepiawei, pë
nakaremahe. 

Ɨhɨ hei wãriti tënɨ, katehe Yamonamowãro iha a sioha rë kuaawei, ɨ̃hɨ e ã
harayomahe: 

--- Pei! Hei wama kɨ rë kui, hayë pë yai ta nakatarɨhe, xoayë pë yai rë
hirare, miha hayë pë yai hiraa kupe, hëyëha, ahete ha pë hiraa kupe, ɨ̃hɨ
rë pë siparapɨ yai kua kure --- e kumahe. 

--- Pei! Pei! Pei! Oxei, rërëiku, ɨ̃hɨ kahë rë wa! Pë ta kõa! Pë ta
kõaxë! --- e pë kuma, a noã tamahe: sioha ai hei, tëëpɨ yupoma: 

Pë kõremahe. Kõrehenɨ, pë ha waroikunɨ, ɨ̃hɨ pënɨ e ma kɨ napë kea
xoakemahe. Kama pë puhinɨ, pë siparapɨ wãriao puhiomi yaro, pë no preaaɨ
maopë, ɨ̃hɨ ai pë siparapɨ rë wãriaonowei, të pë hemata no wai preprawë
ha, kama pë puhinɨ, ma kɨ huxomi pata ëpëhëprare hërɨmahe, hii hi
kurenaha, kama pë puhinɨ, makɨ huxomi pata ha ëpëhëpramarɨhenɨ, të kɨ no
pata tuyëwëapranomihe. 

--- \textit{Kraxi, kraxi, kraxi} --- të kɨ atahu pata yutuhamaɨ piyëkopehe, e
makɨ napë pata kea mɨ hetutua xoakemahe. Ɨhɨ wãrimahi pë pata rë
kurenaha, të makɨ pata rii kuwë, kihi kë të kɨ posika pata yawakaɨ
haɨtao huimati, të kɨ posika pata yawëtou kuimati, hawë makayo kɨ ka
pata taamamahe. 

--- Kɨ mamo no ta yërehe --- e pë kumahe Parawari. 

Ɨhɨ të pë husi ma rë tirehetai, ɨ̃hɨ kama xoati pë siparapɨ hĩiprawë,
sipara rë pënɨ, ɨ̃hɨrë pënɨ pehi rë tuyënowehei, pë siparapɨ hĩihiwë. 

Hei kamiyë pëma kɨ yaro, Yanomamɨ pëma kɨ mamo yëo tëhë, mayẽpɨ a pao
ha, pei pëma pë husi wãha rë hiraɨ, sipara e pë, sipara rë a kepou kure.
Wa si wërema, wa si nohi no tapɨ rë mai! Të pë nohi ma rë kure. Kraxi,
kraxi, kraxi! --- ɨ̃hɨ hei rë të rë kutaruhe naha, të pehi wëkëama
hërɨmahe. No tetenomihe. \textit{Kru tu tu tu tu tu!} Të kɨ ãtahu pata xɨrɨkamaɨ
piyëkomahe, të kɨ pata rë tihatomouwei heinaha kuwë, të kɨ ãtahu pata
ĩtatarayo hërɨma. \textit{Hũaaaaaaa!} Hei siparapɨ wãriaopë teri të pë husi
onohowë no rë preprai, kihi të kɨ mɨ pata puruwë. 

--- Ɨnaha kë yai ë, kɨ wã kãi rë haɨma piyehei ë --- e pë kuma. --- \textit{Kru
tu tu tu tu!} të pë pata kutaama. Ĩsitoripɨ heinaha kuwë të amo hõro wai
tikëa hërɨma, pë siparapɨ hawëɨ he yatioma, hei të wainɨ të kɨ pata
huwëpoma. Kuwë makui, këprou haɨtaonomi, kurenaha rë të pata:

--- \textit{Pei kë!} \textit{Aaaaaaaaooooo!} 

Pei të kɨ koro pata reiwë totihiwë rë a makui! Të kɨ pata hãweteo rë mai
kë! 

--- \textit{Aaaaaëëëë}, e pë kuma. --- Pei kë kahë wanɨ, kihamɨ të pata ora këɨ
kuaa hërɨpë! 

Ihamariwë a noã tamahe. Ei të kɨ ora pata ɨpɨamaɨ puhimahe, të pata
hãweteproimi ha, të pata këɨ haɨami ha, ɨ̃hɨ hei rë të kɨ ha yõriyõri rë
a rë taare, a rë ututuare anɨ, Ihamariwënɨ të kɨ ora pata ɨpɨretayoma.
Pë siparapɨ hawëa he yatirayoma. Kihi të kɨ pata ma rë tuyëɨ
hërakɨrɨhei, kiha e të kɨ ora pata ututuretayoma. Ihamariwënɨ: 

--- Ware a xëprarɨ! --- të kuɨ ha maonɨ, të pë kirihou marë mai!

Ɨhɨnɨ kɨ yai ɨpɨrema. Kama yõriyõri enɨ, ihiya masita rë kurenaha, e të
no owawëmi makui kɨ ora pata ɨpɨre kirioma, xĩroxĩro të ha yami
yaiikutunɨ, xi kãi ha kirirɨnɨ mai! \textit{Ku tu tu tu tu tuuuuuuuuuuuu!} Hawë
kihi të parɨkɨ pata ketayou të kɨ pata tẽreremotayoma të kɨ pata kë
hërɨɨ yaro. 

Naroriwë a mɨa no preoma, a ɨ̃kɨma, mapuu kɨ kãi kei mɨ tëapraroma, a
puhi õkii yaro. 

--- \textit{Tuuuuuuuuu, tẽẽẽrererere!} 

Kihi Ihamariwë pehi rë yokëɨ nokarakiri, hëyëmɨ të keo maopë. 

--- \textit{Ãaaaa!} --- A pehi rë yokëɨwei, kihamɨ të no preaaɨ a pehi keo
\textit{kurakiriiiiiiiiiiiiii}, a yaupraye kirioma, pei imisinɨ, të wãti yaupraye
kirioma, kuprao tëhë, pei e makɨ pata rë këre, ai te hi kɨ pata rë
hayuyare, e te hi kɨ hawërayoma, ahehe hamɨ moi pei a rë yaukenowei
kirionowei, a xëkei. 

Ɨhɨ tëhë a rë wãriarɨhahei. 

Ɨhɨ hei yarori pënɨ, ara pënɨ, mayẽpɨ pënɨ, pokorari pënɨ, hororomɨ
pënɨ, paruri pënɨ, kuremɨ pënɨ, katara pënɨ, katauri pënɨ, yãpi pënɨ,
Yanomamɨ kurenaha të pë kuoma yaro, hei kurenaha a nohi warokemahe yaro.
Ɨhɨ rë ĩyë pë rë hɨprɨarahei të pë warokema. Ɨhɨ rë ĩyë pë wai praaɨ rë
marahei, ĩyë pë napë warokemahe. Ãrima pë wai makui, të pë wai rë hãrei,
axiaxi ĩyë pë wai rë hɨprɨanowehei ɨ̃hɨ rë të pë. Kiritamɨ pë makui të pë
yaitawë, ai të pë he wakë, ai të pë he axi, pë mamokasi kɨ kãi axi,
ɨnaha pë rë kure, ĩyë pë hɨprɨamahe, të pë rë kuonowei hamɨ, ɨ̃ha rë pë
xi wãrihopë ĩyë pë hɨprɨaɨ kurahei, ai kihi pë rë itorɨhe hamɨ, pë
hãtopɨ nahi mɨonomi. Ɨharë. 

Ai pë kõa mɨ ha yaparɨnɨ, pë ni kuopë ha, pë haropë mai! Ɨhɨ hei rë ĩëpë
rë yãarahei, hɨprɨarahei, haikorahei, ɨ̃ha rë pë ha waikiprarunɨ, pë yëo
xoaokema. Pë xi wãrihiprou xoarayoma. Urihi a haikia xoaremahe. Urihi a
hëprou rë mai! 

Ai të pë parɨkɨ ĩxi yahetiataroma. Të pë õramisi hãhɨataroma, mamo kasi
kɨ hɨprɨataroma, të pë henakɨ kãi hɨprɨataroma, ai të pë yaro he marë
rohorei, pei hẽoxipë wai hɨprɨamahe, pei xĩhipë makui, xĩhipë kãi hɨprɨa
haikiaremahe, xĩhi pë praonomi, ɨ̃naha të tamahe. 

Pei makɨ ɨpɨarewënɨ xĩhipë hɨprɨama, ɨ̃hɨ ũ kɨ marë rohorei, ɨ̃hɨ pei rë
xĩhipë, pei hẽoxipë hɨprɨamahei, pei kõhesi ha, ĩyëĩyë e të wai
hɨrɨkɨkema. 

Kuwë yaro ɨ̃ha ĩyë pë rë waikarahei, pë yëokema. Pë marayoma. Ɨhɨ të rë
kui, të maprarioma, ɨ̃ha kama pë rii huokema yaro. Ɨha kama pë xi rii
wãrihiprarioma yaro, pë rë huoi kuhe, ɨ̃hɨ tëhë të wãisipɨ maxi rii hamɨ,
ai të rii kua notia, ai. Ai të ã. Ei të rë kui, të rë tanowehei, të ã
kuprarioma. Ihamariwë. Naroriwë pehi rë tuyënowehei të ã. 

Ɨhɨ rë te he rii tikëa kure. Ɨhɨ rë a urihi ha, urihi a wãha pei pë rë
përɨonowei a wãha. A wãha kãi kua. A wãha urihi rë taponowehei. Pëma kɨ
urihipɨ wãha marë kuprai kurenaha a wãha kuopë ha, të pë përɨhɨoma. Ɨhɨ
kama katehe a rë xënowei, ɨ̃hɨ Moyenapɨwei a parɨkɨ ha, a përɨoma. Katehe
a yai rë takenowei, Moyenapɨwei a parɨkɨ ha a përɨoma. Ɨhɨ Moyena pei e
makɨ wãha kuoma. Ɨhɨ kɨ të pëpoma, Yamonamowãronɨ. Ɨhɨ kama pënɨ, kamanɨ
Moyenapɨwei të wãha yupomahe. Moyena a urihi poma. 

Hehu makui ɨ̃hɨ Moyenapɨwei e kɨ kuoma. Katehe a urihi yaro a tapoma. Ɨhɨ
naxomi a himo rarou tëhë, urihi a rĩ pata hirakaatima. A rĩ pata hirakaa
xoaa, kama a përɨo no kuopë ha. Ɨha kama a no hekama përɨa xoaa,
Moyenapɨweiteri pë wãha rii kua. Ɨha Moyenapɨwei teri pë përɨa hëa, kama
pë wãha kuoma yaro, pata pë wãha kuoma yaro. 

Ɨhɨ të mɨ amo ha, te he tikë ha, hëyëha Iwariwë a rii përɨoma. Ɨhɨ
Yamonamowãro a xamiano rë kure ha a urihi mɨ amo ha he tikë ha a rii
përɨoma. 

 
\chapter[A transformação dos quatis]{A transformação\break dos quatis}

\letra{O}{s animais} moravam em xapono; os quatis, as cutias e as antas, as
queixadas, os cuatás, os beija-flores, os passarinhos moravam em grupos
como nós.

Naquela época, a mesma transformação ocorreu com todos esses animais,
exceto o caititu. Ele não andava como nós; não morava como nós, ele
sempre andou como ele ainda anda hoje. É um animal, e sempre foi. Sempre
andou como animal, assim como os cuxiús, os inambus, as cutias
vermelhas, os veados roxos.

Havia cinco espécies de animais. 

Já existiam todos os animais que há hoje na floresta? Não, somente
esses. Os jabutis não existiam, não andavam, não existiam como animais e
nem como gente. Nem os tatus-galinha. 

Vocês comem o quati, apesar de ele ser Yanomami. Os Yanomami se
transformaram em quatis. Tornaram-se animais no tempo de Horonamɨ. 

Como eram Waika, eles se transformaram. Eram os ancestrais dos Waika.\footnote{O par \textit{waika}/\,\textit{xamatari} parece ter sido usado originalmente para designar outros grupos yanomami vivendo em região geográfica diversa de quem fala, os primeiros ao norte e oeste, e os segundos ao sul, reconhecendo-se neles conjuntos de características que os particularizam. Os termos foram atribuídos em diferentes momentos pelos brancos para designar grupos específicos de forma estável e, no caso de \textit{xamatari}, para designar a própria língua do tronco yanomami usada pelos Parahiteri que fizeram este livro.} Quando saíram de \textit{wayumɨ}, todos se transformaram.\footnote{\textit{Wayumɨ.} são longas estadias coletivas na floresta. Em geral são motivadas pela falta de comida no xapono. A comunidade pode se dividir em vários grupos quando se trata de um xapono populoso, e se desloca num vasto círculo, fazendo acampamentos sucessivos.} 

--- Querido! Meu nariz se rasgou! 

--- Meu nariz se rasgou assim também!

--- \textit{Õãaa}! \textit{Xiri}! Meu nariz arrebitou! 

Foi assim com todas as crianças. 

--- \textit{Õãaa}! \textit{Xiri}! Avô! Meu nariz também arrebitou! 

--- \textit{Õãa}! \textit{Xiri}! --- disseram todas as crianças ao avô delas. 

Transformaram-se enquanto estavam de \textit{wayumɨ}. Os ancestrais Waika
se metamorfosearam. São os primeiros moradores; eles se transformaram.
Ocuparam toda a floresta. Não sobrou nenhum xapono em torno do qual não
vivam quatis. 

Esse rio grande, rio abaixo, do qual vocês comem muitos peixes, na sua
parte média, apesar de ser rio abaixo, dá para avistar a pelagem muito
vermelha dos quatis. Eles andam por lá. 

--- \textit{Fĩfĩfĩ}! --- eles dizem. 

Os quatis são Waika. Foram os antepassados dos Waika, que se
transformaram indo de \textit{wayumɨ}. O nariz quebrou e se arrebitou.
Transformaram-se no meio da floresta. Nunca mais voltaram a morar em
xapono. Eles se transformaram. A imagem deles se alastrou por toda a
floresta, como também a imagem dos jabutis.

\chapter{Të pë rë yaruxeprarionowei}

\letra{Y}{arori} pë rë kui, hei yaro pë hirapramoma, yaruxe, tomɨ, xama, Yanomamɨ
kurenaha të pë rë hiraonowei, warë, paxo, të pë rë hiraonowei, tẽxo,
kiritamɨ pë kãi hiraoma, hei kurenaha. 

Ɨhɨ tëhë kama të rë wãrihore, poxe pë rë kui, Yanomamɨ të pë rë kurenaha
a huɨ taonomi, a përɨaɨ taonomi, kama xoati a huɨ rë xoaonowei. Poxe
yaro kë a rë hunowei ɨ̃hɨ a wai mahu a xĩro huma. Poxe, wɨxa, hõrama,
xĩhɨna, prẽari. 

Ai, ai, ai, ai, ai, ɨ̃naha të pë huɨ kutaoma. Hei kurenaha urihi a
hɨtɨtɨoma, yaro pënɨ? Ma, hei të pë xĩro kuoma. Totori pë kãi kuonomi,
pë kãi huɨ taonomi, totori Yanomamɨ pë kuonomi, opo pë kãi hunomi. 

Hei yaruxe, Yanomamɨ a makui wama a waɨ, ei të pë rë kui Horonamɨ a kuo
tëhë, pë xi wãrihoma. 

Yanomamɨ të pë yaruxeprarioma, Waika kë pë. Kama Waika pë wãha kuoma,
Yanomamɨ. Yaruxe pë rë kui Waika pata pë përɨoma. Waika pata të pë
hiraoma. Ɨhɨ Waika pë rë hiraonowei, ɨ̃hɨ kama Waika rë pë rii yaro pë xi
wãrihoma. Ɨhɨ rë pëma pë Waika yaruxe waɨ, pë xi rii rë wãrihonowei.
Wayumɨ pë xi wãrihou haikoma. 

--- Õasi ya hũka kɨ nohi hëtɨa yairëhe! Ɨnaha ya hũka kɨ rii hëtɨa
kurayou kuhe. Ihiru hɨtɨtɨwë të pë kãi rë kui. 

--- \textit{Õaaa! Xiri!} Xoape ya hũka kɨ rii hëtia kurayou kuhe. 

--- \textit{Õaaa! Xiri!} --- e pë kuma. 

Ɨhɨ wayumɨ rë pë xi rii wãrihiprarioma. Waika pata kë pë xi rii
wãrihoma. Xomaomɨ të pë, pë xi wãrihoma. Urihi kë a haikia xoaremahe. Ai
të kɨ yahipɨ xee hëama rë mare! 

Hei të u pata koro hamɨ, hei wama të u pëɨxokɨ no yuripɨ waɨ rë xoape,
të u koro pata makui hamɨ, yaruxe të kɨ wakë pata xopoi kunomai të kɨ
pata ma rë huɨ korowë piyëkei. 

--- \textit{Fĩfĩfĩ!} --- pë kutoma

Ɨhɨ Waika kama nohi patama pë kuoma kutaenɨ pë wayumɨ huɨ xi ha wãrionɨ
pë xi wãrioma. Pë hũkakɨ ha kërarunɨ, pë hũkakɨ mɨ yaprekewë. Pë hũkakɨ
hëtɨtoma.

Hëyëha, Wãikayoma pë kãi hiraomahe. Wãikayoma pë kãi rë hiraonowehei.
Koteahiteri pë yahipɨ he tikëre hamɨ, hei kë pë yahipɨ.
Õramisitarematari pata kë pë yahipɨ. Hëyëha pë rii hiraoma. Ɨhɨ pë rë
hirare pë yai naiki he ropao totihioma. Yaro a waɨ no teteonomihe. Pë
yai naikioma. Ɨhɨ kutaenɨ Õramisitarematari pë wãha kuoma. Ɨhɨ
Õramisitarematari pënɨ Koteahiteri pë nohipomahe. Kama nohi e pë kuoma.
Pe he rë waroyouhe rë tikëkonowei Õramisitarematari pë wãha kuoma. 

Ɨhɨ rë pë iha amoa hi nohi kõapraɨ piyëkomahe. Kama pë yahipɨ he
tipëtëmoma. Pata të pë wãha yai. Oramisitarematari pë yahipɨ he tikë ha,
Waika pë xi rë wãrihonouwei pë rii hiraoma. 

Urihi ha, pë xi wãrihia xoararioma. Pë përɨaɨ kõonomi. Ɨhɨ pë xi rë
wãrihiraruhe, pë no uhutipɨ huomopotayoma, urihi a haikiremahe. Urihi a
he tatohowë rë kurenaha, a urihi yaitawë makui, a urihi haikiremahe,
yaruxe pënɨ, totori pënɨ. 

\chapter{A proliferação do fogo}

\letra{E}{u vou} contar a história de Jacaré. Seus conterrâneos sofriam por causa
da escuridão à noite, porque não conheciam o fogo e, então, comiam cru.
Depois de apanharem frutas \textit{kaxa}, eles as comiam cruas, pois não
havia fogo.\footnote{Lizot identifica \textit{kasha} como um tipo de lagarta que vive em uma variedade de ingazeiro chamado \textit{kasha nahi}. Na versão por ele recolhida desta história, o detentor do fogo come as lagartas cozidas, não as frutas do ingá.} A região chamada Kaxana era a região de Jacaré. Ele ocupava
essa região. Ele bebia a água do rio Kaxana. Era o nome dessa região.
Ele comia escondido as frutas \textit{kaxa} cozidas, aquele que
detinha o fogo na sua boca. Essa região Kaxana fica no centro da
floresta. No meio dessa região, há o rio Kaxana. 

Os jovens que moravam com Jacaré estavam sofrendo e delirando por causa
da comida crua. Ele comia o cozido sozinho. Ele não oferecia aos outros. Até o
paladar das mulheres sofria com a comida crua. A história dos
companheiros de Jacaré, que, depois de pensarem, encontraram o fogo
guardado por Jacaré, ocorre no meio da nossa história. 

Alguém acordado à noite ouviu o som baixinho daquele que mastigava
a comida frita escondido. \textit{Kãrɨ, kãrɨ, kãrɨ}! Por causa de sua boca
que fazia esse som baixinho, um deles percebeu o que estava acontecendo. 

--- Será que ele está comendo algo frito?

Ele pensou assim, apesar de não ver lenha queimada no chão, como
acontece quando a gente acorda e, ainda deitado, olha para o chão. 

Depois de torrar os alimentos na sua boca, ele os comia com sua esposa à
noite. \textit{Kãrɨ, kãrɨ, kãrɨ}! Os dentes dos dois faziam esse som
baixinho. 

Depois de seus companheiros acordarem, falavam de Jacaré, baixinho. 

--- \textit{Hoaaaaaaaa}! Vamos, vamos procurar! 

Não foram nossos antepassados que descobriram o fogo. Nós não conhecemos
o fogo por ele ter aparecido de repente. Nossos antepassados sofriam,
eles endoideciam por causa da comida e pareciam doentes. Sofriam e
endoideciam por comer carne crua. 

Salvaram-se com o fogo de Jacaré, que pegaram e espalharam. 

Eles viviam sempre tristes; o fogo quase não saía, quase não existia.
Viveriam sempre tristes se o fogo não tivesse existido e sido
repartido. 

Jacaré ensinou os \textit{napë} a fazer fogo; eles o usam porque Jacaré
lhes ensinou, pois ele guardava o fogo na sua boca.

Depois de acender as brasas em um tipo de forninho, ele colocava a
comida nas brasas. A comida cozinhava escondida em cima das brasas; ele
fritava a caça e as frutas escondido, ele fazia assim. Ele
guardava o fogo com ciúme, quase não revelava o fogo. 

Todos se juntaram para descobri-lo. Reuniram-se, ali onde Jacaré comia
queimado e guardava as folhas dos embrulhos. Ele as enterrava, cobria-as
com terra; cavava um pouco o chão e colocava as folhas queimadas; fazia assim. Sozinho, saboreava a comida cozida. É um jacaré, como a gente
diz.

Mas ele cedeu o fogo? Não, ele sovinava o fogo. Vocês só veem um pedaço
da língua dele por causa do fogo. É um jacaré. A língua não queimou à
toa, o fogo acabou com a língua, ficou somente um pedaço no fundo, pois
ele guardava o fogo na boca. 

Avisaram todos os xaponos, convidando-se a se reunirem. Chegaram e se
reuniram. Jacaré não ficou sabendo, e saiu à procura da
fruta \textit{kaxa} enquanto eles se reuniam. 

Ali onde se revelará o fogo, eles queriam se alegrar rapidamente com a comida
cozida; conseguiram encontrar lascas queimadas de comida. O fogo foi revelado com Jacaré; os \textit{napë} e os Yanomami terão o usufruto do fogo,
por isso comemos cozido.

No início, quase não conhecíamos a comida cozida. Onde nossos ancestrais
comeram cozido? Onde comiam cru com sangue? Quase todos nós comíamos
cru! Você escuta as minhocas e os minhocões, eles suspiram
aliviados, \textit{tɨ̃ɨɨ}! \textit{Tɨ̃ɨɨ}! Quase você comia esse tipo de
minhocões, quase você sofria, assim como os antepassados sofriam. 

Reuniram-se, como nós fazemos agora. Na reunião, proibiram Jacaré de sair. 

--- Vamos, avô! --- seus netos disseram. 

Um de seus netos, aquele tipo de calango que fala
assim --- \textit{Serororo}! --- aqueles calangos pequenos que caem na água e
que fazem assim, \textit{serororo}! Nós chamamos de \textit{temoa}. Era um
dos netos dele, quando era gente. 

--- Vamos, meu tio! Fique parado! Você vai escutar. Escute os que se
reúnem. Deixe para ir amanhã de novo à procura de comida! --- disse seu
neto. 

--- \textit{Haaa}! --- ele disse com voz rouca. --- \textit{Hai}! \textit{Hai}! Por quê? --- ele não perguntou. --- \textit{Ho}! Está bem! --- disse. Ele logo disse assim. --- \textit{Hëëë}! --- disse ele, da mesma forma que diz hoje. 

Enquanto o neto se dirigia a ele, ele permaneceu parado. 

--- \textit{Ho}! Sim! Vamos esperar! --- disse à sua esposa. 

Ela tem uma voz harmoniosa, como a que se escuta nas cabeceiras dos
igarapés. \textit{Pẽi}! \textit{Pẽi}! \textit{Pẽi}! Os que falam assim, na beira d'água, os
que sempre ficam dentro da água, os \textit{pëipëimɨ}. Era a esposa de
Jacaré. Apesar de ser esposa de Jacaré, era parecida com a perereca que
tem desenhos na coxa. 

Quando o neto disse aquilo, eles pararam, ele e sua esposa. \textit{Hɨ̃ɨɨ}!
Ele sabia de quê se tratava. 

--- Eu não revelarei o fogo que eu possuo! --- pensou logo. 

Ele logo ficou bravo, em vão. Ele foi dormir, para não dizer nada
àqueles que estavam reunidos. \textit{Hɨ̃ɨɨ}! Ele afastou sua rede, virou
seu rosto para o outro lado e adormeceu. Não se mexia. 

Os outros queriam fazer Jacaré rir e se revezavam. Os passarinhos se
revezaram. \textit{Ão, ão, ão, ão, ão, õoooo}! Eles tentaram, mas ele não
se mexeu. Aquele que possuía o fogo, cujas brasas brilhavam, guardou a
boca bem fechada e parecia não ter boca. 

Cada pássaro ia no meio do xapono dançar, se revezando.
Os uirapurus \textit{huimiri}, os \textit{taɨsarakari} e os galos-da-serra iam se deitar
com ele, se revezavam. Os bicos-de-brasa se sentavam no chão, sentavam,
sentavam, mexendo as asas; queriam fazê-lo rir e se revezavam. A esposa,
de costas, disse logo: 

--- Não se mexa! Fique aí quieto! --- disse ela.

Em seguida:

--- Não se mexa! Durma! --- dizia ela do seu canto.

--- Vocês vão fazer assim? --- Jacaré não reclamou, apesar do barulho. 

Bem depois, eles cansaram. Ele não se mexia. Ficaram cansados, os
coitados se mexeram tanto, tiveram tantas dores, que começaram a
sofrer. 

Beija-Flor não se levantou logo. Ele estava no centro, afastado com
Tohomamoriwë, irmão dele. Eram os dois: o mais velho, Beija-Flor, e o
mais novo, Tohomamoriwë, daqueles beija-flores pequeninos. Beija-Flor
agiu primeiro. 

--- Irmão maior, sua vez! Você tenta logo, eu vou olhar primeiro! ---
falou o irmão mais novo. 

Apesar de o irmão mais velho se mexer, Jacaré não reagiu.

--- \textit{Sĩo}! \textit{Sĩo}! \textit{Sĩo}! \textit{Tõu, tõu, tõu}! --- disse.

Mexendo-se à frente de Jacaré, ficando parado, ele colocou peninhas
brancas no seu cu, nas suas patas, as peninhas estavam repartidas
igualmente nos dois lados do cu. A língua de Beija-Flor saía e os
olhos de Jacaré se abriram. Deu uma olhada e adormeceu de novo. 

--- Não com você! --- pensou. 

Beija-Flor se mexeu duas vezes e se cansou como os outros.
O irmão mais velho disse ao mais novo: 

--- Vai, irmão menor, tua vez, depressa! Com você é bem capaz de ele
rir.

Aí o irmão mais novo disse:

--- \textit{Waooo}! Olhem isso! Sou eu mesmo! Olhem para mim! 

Todo mundo olhou. Bico-de-Brasa estava triste, sentado, como um
doente, aguardando o fogo, para pegá-lo logo. Aquele que estava esperando
se levantou e se aproximou. 

--- Bem, olhem só como eu faço: \textit{Sĩo! Sĩo!} --- dizia a certa
distância, de onde se levantou. 

Parecia o som do carapanã \textit{tëërëkë.}

--- \textit{Sĩo}! \textit{Sĩo}! \textit{Sĩo}! \textit{Tëɨ}! \textit{Tëɨ}! \textit{Tëɨ}!

Um som bonito começava a ser ouvido e o dono do fogo arregalou os olhos,
olhou para o que acontecia.

Beija-Flor logo se levantou, para dançar à frente de Jacaré.

--- \textit{Wão}! Você vai ver!

Ele ficou à frente de Jacaré, continuando a dizer:

--- \textit{Sĩo}! \textit{Sĩo}!

Parecia que voaria para sempre. 

--- \textit{Tëɨ, tëɨ, tëɨ}! --- ele dizia. 

Ele parecia pendurado com as pernas abertas. Ele fazia assim: a cauda
dele ficou virada para cima, e os dois irmãos ficaram um ao lado do
outro. Eles faziam como se fosse uma dança. 


Nessa altura, Jacaré se levantou. Ele estava sério, mas se endireitou e
sentou na rede, e deu uma gargalhada. Com as gracinhas de Tohomamoriwë,
ele entregou o fogo. Tohomamoriwë estava na altura dos olhos de Jacaré: 

--- \textit{Ho, ho, ho, ho}! --- Jacaré riu. 

\textit{Prohu}! A brasa pulou. Tou! Bico-de Brasa esperava o fogo e
o apanhou. \textit{Hɨɨɨ, krihi, krihi}! Ele apanhou o fogo, que queimou seu
bico, por isso o bico dele é vermelho, pela queimadura do
fogo. \textit{Hɨ̃}! Como o fogo era pesado, ele não conseguiu voar alto,
quase caiu de volta com o fogo. 

--- Japu! --- ele chamou.

Japu estava esperando, empoleirado mais em cima. Ele foi levar o fogo.
\textit{Wẽooo}! Pois ele é maior. \textit{Tu tu tu tu tu tu tu!} Levou as brasas em cima
da árvore abiorana murcha, ele as colocou na ponta do tronco da árvore. 

A mulher de Jacaré se levantou. Quando Bico-de-Brasa quase apanhava o
fogo, ela jorrou urina. Apesar de a mulher de Jacaré jorrar sua urina,
Japu levou o fogo mais alto e a urina não o alcançou. Enquanto
o fogo estava em cima e não apagava, ela não acertou o fogo, a urina não
alcançou. Ela disse:

--- Vocês pegaram o fogo, então vocês chorarão quando cremarem os seus
mortos, vocês sofrerão e chorarão pelos seus mortos cremados! 

Ela falou verdade, pois quando morremos, nos cremamos; ela disse a
verdade: nós praticamos a cremação e nos cremamos. Ela disse que era
para ser nossa tradição. 

--- Eu vou ao igarapé e ficarei feliz lá com meu marido para sempre ---
disse. --- Vocês sofrerão com o fogo. Ele se tornará eterno. O fogo
derreterá seus olhos! 

É verdade o que ela disse, a esposa disse a verdade. Nós nos cremamos,
nossa carne queima, ela falou certo. Se isso não houvesse acontecido,
não nos cremaríamos. Dito isso, ela e seu marido, \textit{Kruxu! Kopou!} Os dois
foram às águas para nada de ruim lhes acontecer, para eles não ficarem
doentes, não pegarem diarreia, nem dor de cabeça, nem conjuntivite, nem
terem dor nas pernas, nem conhecerem a malária. E assim será. 

Eles têm problemas de dentes como nós? Não têm, não! Não conhecem dor de
dente. 

Esses eventos aconteceram para que seja assim. Eles ainda estão
felizes. 

--- \textit{Pĩri}! \textit{Pĩri}! \textit{Pĩri}! \textit{Pĩri}! \textit{Pĩri}! \textit{Pĩri}! \textit{Pĩri}! \textit{Pĩri}! \textit{Pĩri}! \textit{Pĩri}! \textit{Pĩri}!
--- cantou a esposa, até chegar às águas. 

Os outros pegaram o fogo e os dois ainda moram nas águas.

\chapter{Iwariwë}

\letra{K}{ama} Iwariwë hei pë rë kui, ya pë wãha rë tare, ruwëri pë no prepramoma,
kaɨ wakë tanomihe, riyëriyë yaro pë wamahe. Riyëriyë atayu pë wamahe.
Riyëriyë kaxa a pata yaxaamahe. Kaɨ wakë kuami yaro. Ɨhɨ Kaxana kama e
urihi ha, Iwariwë a rii përɨoma, Kaxana a urihi ha. A urihi rii poma.
Kaxana u rii koama. Urihi ɨ̃hɨ Kaxana a wãha kua. Rɨpɨrɨpɨ ɨ̃ha a waɨ
hãtooma, kaɨ wakë rë pore, kaɨ wakë rë titipore pei kahikɨ hamɨ. Mɨ amo
yai hamɨ, Kaxana a urihi kua, Kaxana u mɨ amoa. 

Hëyëha Iwariwënɨ kama e të pë hei kurenaha kama huya e pë kãi rë
përɨawei makui riyëriyë të pënɨ të pë xi wãrii no preo ayao hei. Kamanɨ
rɨpɨrɨpɨ yaminɨ a wama. Të pë toonomi. Suwë makui riyëriyë të pë ha, të
pë no kãi preaama. Hei të pë rë kuinɨ, yakumɨ të pë puhi ha tao hërɨnɨ,
Iwariwë iha kaɨ wakë he rë haaɨ rë piyërahei, hei mɨ amo ha të kua. 

Mi titi hamɨ harɨkano kë kɨ waɨ rë hãtoawei kahikɨ ĩsitoripɨ wai hõra ha
hĩria he harurehenɨ, ai të rë rãa he haruawei. \textit{Kãrɨ, kãrɨ, kãrɨ!} Kahikɨ
wahato kuɨ makure. 

Ei harɨkano rë kɨ hõra nohi waɨ kui, a nohi karɨkarɨmouwei! E puhi kuɨ
ha kuikunɨ, wãri rë të wakëxi wai praama mare makui, të pë ha rãmanɨ, të
pë mamo kãi ma rë përɨmapouwei. 

Ɨhɨ pei kahikɨ ha, kɨ rë harɨkaɨwei, kɨ wapɨɨ he harukei, hesiopɨ xo.
\textit{Kãrɨ, kãrɨ, kãrɨ!} E kɨ nakɨ kupɨma wahato. 

Ɨhɨ e pë ha rarɨnɨ, pë noã rë wahato taɨwei, taɨwei, taɨwei: 

--- \textit{Hoaaaaaaaa!} Pei këëë! Pë ta taeo! --- e pë kuɨ xoatarioma. 

Ai patama kamiyë pëma kɨ patamanɨ kaɨ wakë ha tararënɨ, wakë rë
pëtarionowei, wakë kupronomi. Kamiyë pëma kɨ no patama preaama. Riyë të
pë ha kamiyë pëma kɨ no patama xi wãriprou no preoma, hawë rãakãi të pë
kuaama. Yaro riyëriyë pënɨ të pë xi wãriprou no preoma. 

Ɨhɨ iha wakë he ha ha piyërehenɨ, Iwariwë a prukaremahe, kaɨ wakë ha,
wakë yua piyëkëapotayomahe. 

Të pë puhi rë owahataohe të pë xĩro õha kunoha, kaɨ wakë mori hanomi,
wakë mori kupronomi. Ɨhɨ a mao ha kunoha, hei riyëriyë pëma të pë waɨ
xoawë. 

Ɨhɨ Iwariwënɨ të pë ha hirakɨnɨ, wakë tapraɨ ha hirakɨnɨ, napë pënɨ wakë
tapraremahe. Të wakë pë kãi hĩihaɨhe, ɨ̃hɨnɨ të hirakema yaro. Pei kahikɨ
ha wake titipoma yaro.

Hapa anamahu wakë rë kui, wakë ha homoprarutunɨ, heinaha hawë maramahe
kure të ha, wakë anamahu pata makeama, ɨ̃hɨ wakë anamahu ha kɨ yaropɨ
rɨpɨpɨo hãtooma, kaxapɨ hãrɨkɨpɨo hãtooma, ɨ̃naha të tama. 

Wakë no xi ɨmapou he parohoma, wakë kãi mori wawënomi, kurenaha ɨ̃hɨ a
napë kõkaprou rë piyërahei, ɨ̃hɨ re e wakë wawëmapehe, a napë kõkaprou
piyërayomahe. 

Ĩxiĩxi a rë iaɨwei hamɨ, hena pë si pomama, pita pë rëmama, maxita pënɨ
të pë paterimama, pita pë ka ha titanɨ, hena pë ĩxi ɨnaha të tama, kama
yami a totihotii yaro, rɨpɨrɨpɨ të pë ha. Hei, iwa kë a, wa rë kuɨwei. 

Kaɨ rë wakë no xi ɨ̃hɨtapoma? Wakë noãpou he parohooma. Ɨhɨ wakënɨ wama
aka hemata tapraɨ. Hei iwa kë a, wa rë kuɨwei, pei aka no watëno pëwëmi,
ɨ̃hɨ kaɨ wakënɨ aka rë haikiarenowei, korokoro të wai hemata hëa. Wakë
hopoma yaro. 

A napë kõkaprou piyërayomahe, e të pë waroo piyëkema, kihamɨ e të pë noã
no waxuoma, të pë rë nakayouwei, të pë kõkamoma. Kama a mohotio tëhë, ai
kaxa a yuaɨ mɨ napë kuo tëhë, a napë kõkaprou heamahe. 

Ɨhɨ wakë rë wawëmare hamɨ, rɨpɨrɨpɨ të pë ha, të pë puhi yai toprarou
haɨtao puhiopë yaro, ĩsitoripɨ të nakaxi wai taa he ha yatirarɨhenɨ. Ɨhɨ
iha wakë rë wawërɨhe, napë, Yanomamɨ pënɨ wakë ha piyëarɨhenɨ,
piyërehenɨ, pëma kɨ iaɨ, rɨpɨ të pë ha. 

Hapa pëma kɨ mori ianomi. Ɨhɨ weti ha pata pënɨ rɨpɨ të pë wamahe? Ĩyë
makui, të pë kãi wamahe? Ɨnaha pëma të mori pruka tama. Horema
korimorewë: \textit{Tɨ̃ɨ̃ɨ̃! Tɨ̃ɨ̃ɨ̃!} Wama pë kuɨ hirii? Ɨhɨ naxomi wama xi kɨ mori
wama, ɨ̃naha pë no preaaɨ kuoma. 

Ei a napë rë kõkapraruhahei, hei kurenaha e të pë kõkaproma. Rë
kõkapraruhe, a wasitaremahe. 

--- Pei! Xoape! --- pë xɨɨ hekamapɨ e kuoma --- \textit{Serororo!} --- të pë wai
rë kuɨwei, ɨ̃hɨ hekamapɨ ree. 

Hawë xãraima ãto pë wai rë kure mau u pë ha, të pë wai keo ha:
\textit{Serorororo!} Të pë wai rë kuɨwei. Temoa, hawë reha pë rë kure naha, të pë
rë kure, hekamapɨ Yanomamɨ e kuoma, hei kurenaha e kuoma. 

--- Pei! Xoape! A ta yanɨkɨtaru! Ai wa të pë ã hirii. Hei të pë rë
kõkamorɨhe, të pë ã wayou ta hiri! Henaha yai, henaha wa kɨ napë yai huɨ
kõopë --- hekamapɨ e kuma. 

--- \textit{Haaa!} --- e kutarioma, pë a marë wahëi 

--- \textit{Hai! Hai!} Exi të ha? --- e kunomi. --- \textit{Ho!} --- e kutarioma. 

Ɨnaha kama a kuɨ xoawë:

--- \textit{Hëëë…!} --- a rë kuɨwei, pei a wã haɨ. 

Hei a noã taɨ ha, a yanɨkɨtarioma. 

--- \textit{Ho!} Awei! Pei pëhë kɨ no tatou! --- hesiopɨ̃ mau u pë hawaro hamɨ të
pë ã rë karëhouwei. 

\textit{Pẽi, pẽi, pẽi!} Mau u pe he tatoopë ha të pë rë kuɨwei, pẽipẽimɨ, ɨ̃hɨ
hesiopɨ e kuoma, Iwariwë hesiopɨ. Ɨhɨ hesiopɨ e makui, hawë moka e wai
kuwë. Waku kɨ wai oni, ɨ̃hɨ e kuɨ ha, e yai yɨkɨtarioma. 

Ai të pë hapa rë kuprore makui: \textit{Hɨ̃ɨɨ!} Ɨha a puhi kua yaro 

--- Ma, hei kuikë ipa ya wakë rë tapore, ya wakë wawëmaɨ kuami --- a
kutou xoarayoma. 

A xomi huxutou nokarayoma. A ha miikunɨ, të pë rë mɨre, a kuɨ maopë.
Kama pë praopë ha: \textit{Hɨ̃ɨɨ! }Të pë mohekɨ mɨ marë kutaowei. Pëkɨ yawëtëa
xoaparioma. A kãi karihipronomi. Ɨhɨ a rë miore, hei të pë noka rĩya rë
ĩkamouwei, hei e të pë wai yaiataroma. Kĩritari e pë wai yaiataroma. \textit{Ão,
ão, ão, ão, õoo!} Të pë xomi kuma makui, a kãi kupronomi. 

Ɨhɨ ëyëha rë wakë rë titipore, hëyëhë të wakë anamahu pata ma
watawatamope, ɨ̃naha husi kua xoakema ĩkari, hawë kõmikõmi. Ɨnaha kahikɨ
kua xoa parɨoma. 

A rë kure tëhë, të pë pëɨxokɨa hërɨɨ, yaiatarou, huimiri e yaiatarou,
tãɨsarakari e pë yaiatarou,ẽhoamɨri e pë yakaaprarou, yõreketerari e pë
roaroaaprarou, roroaprarotiii, roroaprarotiii, e pë kãi
yahuyahuapraroma, a ka ĩkamaɨ xi totihitaoapehe, e pë yaiataroma.
Hesiopɨ e rë kui kihi e yaipë rëa kure: 

--- A kupro tihë! Miha a kuaaɨ kuparuhë! 

E kuɨ nokamoma: 

--- Kuaa tihë! Miparu hërɨ! --- e kuɨ nokamoma. A mɨ hururanɨ! Të pë ã
ma teteo tëhë:

--- Ɨnaha rë wama kɨ kuaaɨ kupe --- e kunomihe. 

Yakumɨ, të pë ha motarɨnɨ, të karihiproimi ha, të pë ha waximirɨnɨ, pei
të pë wai kuaaɨnɨ, të pë wai ha ninirɨnɨ, të pë no preaatii yaro. 

Ah! Kihi Tẽxori a makui, e kãi hokëprou haɨonomi. Kɨpɨ yai, hei kɨpɨ mɨ
amoa. Hei kɨ yawëtëpɨa kupiyei. Tohomamoriwë a yai, kihinɨ ɨ̃hɨ kihi pata
e rë kui, kihi Tẽxoriwë, ɨ̃hɨ a rë kui pata kee. Tohomamowë oxe kee, të
pë wai rë sinapii, ĩsitoripii, ɨ̃hɨnɨ kihi e kuaama, pata e rë kui e
kuaama: 

--- Pei! Apa, kahë! Wa wapaɨ xomao, kamiyë ya mamo yëo parɨo --- e
kuma. 

Pata e kuaaɨ makure:

--- \textit{Sĩo! Sĩo! Sĩo! Tõu, tõu, tõu!} --- e kuaama. 

Kuaaɨ makure, pei mɨ tarɨ ha, hëyëha e katioma, Tẽxo e husi ha
horoiprarunɨ, e mamikɨ kãi ha horoiprarunɨ, e posi wai horoi
wauhuapraroma. E aka kãi nianiamoma makui, e mamokasi homotarioma, a mɨɨ
ha, miaa kõkema. 

--- Kahë iha mai! --- e puhi kutarioma. E kuaaɨ ha horohoprarunɨ, hei të
pë rë hɨtɨtɨpraruhe, të pë waximi yaro. 

--- Pei! --- pata e ã hama, oxe iha --- Pei! Õasi! Kahë! Haɨpraru! --- e
kuma. 

--- Kahë iha pei të yai xi kirihiprario --- kuɨ ha: 

--- \textit{Waooo!} Pë mamo ta yëparu! Ɨhɨ rë! Wamare a ta mɨ! Kamiyë yai! --- e
kuma. 

E të pë mɨ puruparioma. Yõreketerariwë hei të no pretaa kure. Hawë e
rãakãi e kutaoma. Ɨhɨ kaɨ rë wakë no tapou kure. Wakë rĩya ha nokaanɨ.
Hei e no rë tare, e hokëtarioma. 

--- Pei, ɨhɨ rë, pë mamo ta yëparu! Ɨnaha ipa të kua. \textit{Sĩo! Sĩo!} --- e
kuma, ɨ̃hɨ kiha ree kuɨ kure. 

--- \textit{Sĩo! Sĩo! Sĩo! Tëɨ! Tëɨ! Tëɨ!} 

Hawë të krë e ã taa xoamakema. 

Ɨnaha a kuɨ tëhë, ɨ̃naha e të kuɨ totihiatarou hapa hërɨa ha, hei tënɨ
kaɨ wakë rë tapore, mamokasi homoprou nokarayoma. E mamo homotarioma.
Ɨhɨ të kupë hamɨ, e mamo xatiprakema, e hokëtou nokarayoma hokëprarioma:

--- \textit{Wão!} --- hawë e yë hërɨpë --- Sĩo! Sĩo! 

E ku hërɨma pei mɨ tarɨ ha e wai kutou xoarayoma. 

--- \textit{Tëɨ, Tëɨ, Tëɨ!} --- e kuaprarou ha. 

Hawë e yaua e wai kupario ha, e wai ha wayakarɨnɨ, e kuma. E të texinakɨ
mɨ wai ha yaprekerɨnɨ, e kɨpɨ mɨ wai hetutupɨkema. Hawë hekuramou e kɨ
wai tikutikupɨapraroma. 

Kuaaɨ ha Iwariwë a hokëprarioma. A puhi yopraa makupe, a ha hokëprarunɨ,
a ha tipëtarunɨ, ĩkawã wã no rarɨɨ xoaopë. Tohomamoriwënɨ kaɨ wakë no xi
ɨhɨtarema yai, heinaha e të mɨ wai tarɨaprarou kupe, hei, hei, mɨ wai
taɨaprarou, mɨ wai tarɨaprarou tëhë: 

--- \textit{Ho, ho, ho, ho!} --- e kurayoma. 

\textit{Prohu!} Kaɨ wakë anamahu pata yutuparioma --- Tou! --- Hei të no rë tare,
Yõreketerariwënɨ, e wakë nokarema. \textit{Ɨɨɨ, krihi, krihi} --- pei husinɨ,
husi no rë watënowei, ɨ̃hɨ husi marë wakëɨi, ɨ̃hɨ kaɨ rë wakë unosi, e wakë
nokarema. --- \textit{Hɨ̃!} --- wakë pata hute yaro heinaha e wakë kãi pata
kuapraroma. Wakë mori kãi pata kei mɨ yapaoma. Hei kihi a no rë tare. 

--- Xĩapo! --- të pë rë kuɨwei, kihi e paoma, kihinɨ: \textit{Wooo!} Ɨhɨnɨ wakë
pata nokare hërɨma, pë ma rë prei, ɨ̃hɨ. \textit{Tu tu tu tu tu tu tu!} Apia hi
pata hëwë ha, pei hi nanoka pata ha, wakë pata anamahu makeketayoma,
hesiopɨ e hokëprarioma. 

Hei Yõreketerariwënɨ e wakë mori mapramaɨ tëhë, naasi hirekoma. Hesiopɨ
e naasi hirekei makure, Xĩaporitawënɨ wakë kãi tirerayoma, naasi
hawërayoma. Ɨnaha e kuma, wakë kãi tirerayou ha, wakë misi rupramanomi
yaro, wakë tanomi yaro, naasi notikema yaro. 

--- Ɨnaha kë wama wakë ma tarenowei, ɨ̃naha kë wama wakë ma tarenowei,
wama wakë imi preaɨ ha kë, wama kɨ ã no rĩya preo hëo, wama wakë imi
rĩya preati hëo! --- hesiopɨ e kuma. 

Peheti a kuma, të pë nomaɨ, kamiyë pëma kɨ yaayopë, ɨ̃hɨ pëma kɨ pehi
wãha hiraprarema, kamiyë pëma kɨ ĩximayou hëopë, pëma kɨ pehi wãha
hiraprarema. 

--- Hawaro ha kë, xoati hawaro ha kë, ipa kë ya të ã kãi rii rë
topraowei kë, ya rii makui ha! --- e kuma --- Wakë imi preaɨ ta përahe,
wakë imi parimi preaɨ ta përahe, wama kɨ mamo rĩya protomotou kë! --- e
kuma. 

Peheti e të takemahe. Hesiopɨ peheti a kuma. Pëma kɨ yaayou, pëma kɨ
yãhi kɨ ĩxipë, të pë pehi wãha hirapraɨ katitirayoma. Kutaenɨ pëma kɨ
yaayou. Ɨnaha të kuprou mao ha kunoha, pëma kɨ yaayoimi. A kuma. A ha
kunɨ, kama, hẽaropɨ xo: \textit{Kruxu! Kopou!} Mau u ha kɨ rii kupɨ kiriopë, kɨ
kupɨprou rë mai! Ai kɨ kãi pëpɨɨ, kɨ kãi kriipɨprou, kɨ he kãi
hayupɨprou, mamorinɨ kɨ kãi yupɨaimi, kɨpɨ matakɨ kãi nini no preaaimi,
hurapɨɨ taimi, kɨ kupɨopë. 

Hei kamiyë pëma kɨ nakɨ rë kurenaha ɨ̃hɨ pë nakɨ kuwë? Kuwëmi. Nakɨ kãi
niniaɨ taomi yaro. 

Ɨnaha të kuopë, të kuprarioma. Wã totihitawë kɨ marë kupɨa xoare. Kupɨa
xoaa. 

--- \textit{Pĩri! Pĩri! Pĩri! Pĩri! Pĩri! Pĩri! Pĩri! Pĩri! Pĩri! Pĩri! Pĩri!}
--- e kuɨ morokaɨ hëparioma. 

Wakë yureihe ha! Ɨharë kɨpɨ, përɨpɨa xoaa. 

\chapter{O surgimento do cupim}

\letra{A}{í tem} cupim! --- nós dizemos. 
Por causa do rio no qual as pessoas se afogaram, outros subiram nas
árvores por medo, ensinando-nos. Ensinaram-nos a subir. Subiram,
subiram. Conforme o rio subia, eles também subiram, sempre mais, até a
copa das árvores e, em seguida, se transformaram em cupinzeiros, mesmo
não existindo cupinzeiros nesse tempo. 

Os cupins grudaram nos troncos. Quem era gente se tornou cupim. A casa
dos cupins fica sentada em cima dos galhos. As casas sentadas são a
imagem daquela que existiu, oriunda dos Yanomami. 

Eles se sentam na forquilha das árvores. Os Yanomami ficaram sentados
nas forquilhas das árvores por medo. Apesar de quererem fugir, eles não
conseguiram. Depois da transformação dos cupins por causa do dilúvio,
aqueles que se afogaram boiaram à deriva na água e se tornaram
jacaré-açu.\footnote{O dilúvio é tema da história do surgimento dos \textit{napë}, que está no volume \textit{Os comedores de terra}, desta mesma série.} Alguns se transformaram em jacaré, outros em peixe, outros em capivara. Caíram na água. Transformaram-se
assim pela água. Não foi obra de ninguém! 

Quem os teria feito? Os cupinzeiros eram gente. Desde a
transformação, feito isso o cupim está sentado e grudado às árvores nas
beiras de rio.

Uma casa gruda, outra está pendurada, outra está enfiada lá em cima. As
casas de cupim ficaram na posição na qual as pessoas estavam.
É assim. 

São do tamanho de uma criança; uns quase ficaram na terra seca, uns
caíram na água por medo. Os que eram um pouco maiores caíram na água por
medo e se transformaram em cabas\footnote{Vespas ou marimbondos.} \textit{xaxa}. Imediatamente ganharam esse nome. Assim se transformaram. Após a transformação,
sua imagem se espalhou. Ocuparam todas as regiões onde moravam os
Yanomami. 

\chapter[Yanomamɨ pë rë ãrepopronowei]{Yanomamɨ pë rë\break ãrepopronowei}

\letra{K}{ihi} ãrepo kë ko --- pëma kɨ kuɨ. Ɨhɨ ko pë rë kui, Motu unɨ pë mixi ha tuo kuikunɨ, ɨ̃hɨ hei suwë ya wãha
yuaɨ rë taprarɨhe hamɨ, ɨ̃hɨ u nɨ të pë kiriri toreroma, pëma kɨ hiraɨhe
ha. Pëma kɨ tuopë, tuo hiraɨ ha. Të pë tuoma, të u pata tirei ha, të pë
kiriri ĩhetoma, ãrepo ko pë maoma makui, ɨ̃hɨ u pata ha rërërɨnɨ, ãrepo
ko pë kua hërarioma. 

Ãrepo ko pë yërëkoma, yërërarioma. Ɨhɨ Yanomamɨ pë rë kuonowei, ɨ̃hɨ ko
pë. Ko pë tikëprawë, hii hi pë poko hamɨ, ɨ̃hɨ të pë rë kuonowei, të pë
no uhutipɨ, tikëkëwë, ãrepo ko pë rë kui. 

Ko pë rë yakaroprai, hii hi pë hakarakɨ hamɨ, ɨ̃hɨ Yanomamɨ të pë kiriri
yakaropramoma. Të pë tokuu puhio makui, pë tokunomi. Ɨhɨ unɨ të pë xi
wãrihou ha kuikunɨ, ai pë mixi rë tuoprou he rë yatianowei, pë rĩya ha
pokëpronɨ, pë xomi rë niaaprarou huxomionowei, pë kãi ãopanaprarioma. Ai
a kãi iwaprarioma, ai a kãi yuriprarioma, ai a kãi kayuriprarioma, të pë
keparioma, Yanomamɨ të pë. Mau unɨ pë xi rë wãrihonowei, ɨnaha pë
kuprarioma. Taprano mai! 

Wetinɨ pë ha tapraɨ kuikunɨ të pë? Yanomamɨ kuoma makui, pë xi re
wãrihonowei, pë yai. Ɨnaha të pë kuaama. Të pë kuaaɨ ha kuikunɨ, hii hi
pë hamɨ, pata u kɨ hamɨ, ãrepo ko pë tikëkëwë, ko pë sutiprawë. 

Ai ko pë kãi yauprawë, ai ko pë yakaroa, kiha ai të ko wai hĩiatayoa,
Yanomamɨ të pë kuaanowei naha, ko pë kuwë. 

Ai të pë ihiru rë kurenaha, të pë mori haxɨrɨoma, të pë kãi keparioma,
kiriri. Hei kurenaha të pë kiriri keoproma, Xãxa na pë kuprarioma. Ɨnaha
të pë kuprarioma. Të pë kuprou ha kuikunɨ, të pë no uhutipɨ
praukurayoma. Yanomamɨ kutarenaha të pë urihipɨ hamɨ të pë kurarioma.

\chapter{Os levados pelo rio}

\letra{H}{avia} um pajé chamado Xiritowë, que veio a se chamar Keopëteri. Os
descendentes moram lá ainda. Os Xiritowëteri, depois, se chamaram
Keopëteri. 

Qual era o nome do rio onde eles afundaram, aquele que deu o nome de
Keopëteri? Esse rio se chama Xitipapɨwei. Eles bebiam dessa água, que
cobriu tudo. Caíram nas águas do Xitipapɨwei. Eles não morreram. Ainda
existem ali. 

Um dia, Xiritowë mandou seu genro buscar os visitantes
num xapono\footnote{É o nome da casa coletiva circular onde vivem os Yanomami. Cada casa
dessas corresponde a uma comunidade ou assentamento.} amigo para dançar durante a festa, e assim
nos ensinou a nos chamarmos e a nos convidarmos mutuamente. 

O genro de Xiritowë foi convidar os parentes e amigos deles, os
Anahupɨweiteri, para dançar durante a festa, assim como fazemos até hoje.
Ele correu até o xapono dos Anahupɨweiteri. 

O pajé Xiritowë chamou seus conterrâneos para a festa sem imaginar que o
rio inundaria o xapono; ele pensava que todos iriam morar lá para
sempre. 

Como se chamava esse rio {antigo}? Esse rio se chamava
Xitipapɨwei. Aqueles que as águas cobriram, antes bebiam dessa água. 

Xiritowë tornou a se chamar Keopëteri --- os afundados. Depois de eles se
afundarem nas águas desse rio, tornaram a se chamar Keopëteri, porque
caíram nas águas desse rio que os levou. Assim, ainda moram lá. Esse rio
se chama Xitipa; os Keopëteri caíram nas águas do Xitipapɨwei. Eles não
morreram. Existem ainda. 

A filha de Keopëteri e seus parentes não se afogaram, apesar de estarem
no fundo do rio, o rio os levou. Eles vivem sempre lá. Tornaram-se
eternos. Tornaram-se esses monstros que nunca morrem; eles afundaram.

\chapter{Keopëteri}

\letra{X}{iritowëteri} a yai hekura përɨoma, kamiyë kurenaha, ɨ̃hɨ a wãha
Xiritowëteri. Kama Keopëteri pë wãha kukema kutaenɨ, kama Keopëteri ɨ̃ha
pë hiraa xoaa. Xiritowëteri Keopëteri pë wãha kuprarioma.

Motu u pata wãha rë kuonowei, weti naha u wãha kuoma? Motu unɨ pë rë
kepenowei, Keopëteri a wãha kuprarioma. Pë kepema yaro, pë yurema yaro.
Motu Xitipapɨwei u pata wãha kua. Ɨhɨ kama e u pata rë makepenowehei,
kama pënɨ u rë koanowehei, u makui, Xitipa kama e u wãha pata yua
xoaopemahe, Keopëteri a kepema. A nomanomi. Ɨha a kua xoaa. Ai pë wãha
xoaa. 

Siohapɨ e matoto rë ayonowei, ɨ̃hɨ weti naha kuwë pë xoama? Ɨhɨ hei kama
nohi e pë rë kuonowei, ɨ̃hɨ pë yahipɨ hetikëkëoma, pë rĩya ha praɨmahenɨ,
hei pëma kɨ nakayou hiraɨ ha, pëma kɨ xoayou hiraɨhe ha, pë nakayoma.
Xiritowëteri a hekuranɨ pë reahumou tëhë pë rë nakanowei. Pë kãi hakë
përɨotii tarei. A keo hëomai tao, të pë puhi ha kunɨ, matoto a ayoma,
kamiyë ipa reahu të kurenaha të kua yaro, a rërërayo hërɨma, kama pë rë
përɨonowei, kama maxi norimɨpɨ Anahupɨweiteri pë praɨmamahe.
Anahupɨweiteri kama norimɨ e pë yai kuomahe. Ɨhɨ pë xĩro hëtarioma. 

Ɨhɨ ei Keopëteri pë tëëpɨ rë kui, hei Motu unɨ pë mixi tuo taonomi, pëma
kohomoo waikio tëhë, kama unɨ pë rii yurema, pë rii kepema. Xoati ɨ̃ha
kama pë kua. Pë ma rë kure, parimi pë kupropë. Yai të pë nomaɨ rë mai pë
kupropë, pë kepema. 

\chapter{A queda do céu}

\letra{A}{história} do céu: há a história da derrubada do céu. No início, ele
estava lá em cima. Este solo é o céu caído, portanto, os primeiros
habitantes foram esmagados. Esta terra os esmagou. Tornaram-se, então,
os Amahiri, que moram em grupos como nós, mas moram lá em baixo. 

Aquele céu caiu. Os nossos antepassados saíram bem no meio; se tivessem
sido lerdos como os que foram amassados, não estaríamos aqui nesta
floresta. 

Depois daquele primeiro céu eterno, este segundo sobreviveu. Este novo
céu sobreviveu. Aquele primeiro, que estava em cima, envelheceu e caiu. 

Depois de sua queda, nós surgimos nesta terra mesmo, pois nossos
antepassados se multiplicaram nesta floresta. Apenas os habitantes de um
xapono sobreviveram para que pudéssemos existir, mas quase que eles não
existiram. Em cima desta terra, nossos antepassados se reproduziram e
nasceram, e depois deles, os seus descendentes. 

O céu não exterminou todos os nossos antepassados, ele esmagou apenas alguns. Os
Amahiri se agruparam lá em baixo da terra. Os que foram esmagados se
chamam Amahiri. 

Apesar de estarmos nas alturas, nós existimos. Assim, se essas montanhas
não houvessem existido e se não se erguessem, nossos antepassados não
haveriam saído, não haveriam existido. 

Eles ficaram espertos por causa dessas cavernas.

Quando começou a estrondar lá em cima, quando o evento se
aproximava, ele avisou seu povo. Os outros não sabiam:

--- Vamos, meus jovens, nós da região central vamos escapar por essas
montanhas, pois eu não sou tolo! Vocês não serão esmagados. Nós apenas
sobreviveremos nesta região e, se for preciso, passaremos para o outro lado
do céu. Limpem essa montanha! --- disse. --- Abram o caminho! Limpem! 

Dito isso, eles limparam ao redor da montanha para se abrigar. Quando o
céu ia cair, quando esse momento se aproximava, a montanha já estava
limpa. Na hora de cair, o céu arrebentou, porque ele estava velho.
Enquanto se arrebentava, os que escaparam entraram na caverna. 

--- Vamos, enquanto o céu ainda está alto! Venham! Depois de amanhã,
depois de amanhã, o céu vai descer até o chão! 

Depois de ele dizer isso, o céu caiu. Esmagou os Esmagados. Moravam
nesse lugar. Quando o céu caiu, esmagou os que ficaram, e os que estavam
na caverna não foram esmagados. O céu ficou por cima da caverna. Assim,
passaram a se chamar Derrubadores de Céu; era o nome deles. Queriam
derrubá-lo, por isso se chamaram assim, com o mesmo nome. Apesar de o
céu quase os amassar, eles escaparam. 

Enquanto caía, o filho mais novo e o cunhado pularam e, assim, se
prepararam. O pai mandou que enfrentassem, mandou arrebentarem o céu. Fez
que o arrebentassem. Apesar de o céu parecer indestrutível, mesmo assim
os dois o arrebentaram. 

Aquele que arrebentou o céu, aquele que tinha esse nome, arrebentou
mesmo: ele se chamava Hutukarariwë. Atacou logo, sofrendo por causa do
sangue, se cortando com os pedaços do céu, cortado perto dos
olhos. \textit{Kreti}! \textit{Kreti}! \textit{Kreti}! Fazia assim. 

Ele penou. Chamava-se assim, Hutukarariwë. Escaparam por onde ele
arrebentou o céu; por essa abertura, só o grupo dele escapou. Os que
sobraram escaparam e saíram. Ninguém mais saiu de outro lugar. Nossos
ancestrais se reproduziram a partir daqueles que conseguiram escapar.
Imediatamente continuaram a se reproduzir. Assim aconteceu. 

--- Enquanto o céu está vivo ainda --- disse o pai aos dois filhos ---
vocês vão juntos! Vão embravecer! Não cortem em silêncio! 

A partir do momento em que nossos antepassados se reproduziram, surgiram
também os Waika. Eles se dividiram.

Antigamente havia também outros grupos. Outro nome importante dessa
época é o espírito Hemarewë. Todos esses nomes são nomes de espíritos.
Hemarewë também vivia nessa época como pajé, ele foi um dos primeiros
habitantes da região.

\textls[-15]{Mas não é o nome dos nossos antepassados. A história do nosso grupo
Parahiteri se encaixa no meio da história dos Yanomami. Alguns grupos
foram se extinguindo e de gerações posteriores foi que apareceram nossos
antepassados.}\looseness=-1

Nossos antepassados surgiram na região central. Nós ficamos nessa
região central, onde surgiu a primeira mulher,\footnote{História contada no volume \textit{Os comedores de terra}, nesta mesma série.} pois nossos antepassados
moravam lá. Ela nasceu nessa região central, ficava lá. Esses moradores
foram chamados de habitantes da região central. Foi assim que nós
surgimos.

\chapter{Hutukarariwë}
 
\letra{H}{etu} misi të ã: hetu misi rë tuyënowei, të ã kãi kua. Hapa a kuoma hei
heaka hamɨ, hei keno kë a. Kutaenɨ ei yëtu hamɨ të pë rë kuonowei,
hëyëmɨ pata të pë hɨ̃kɨwë. Hei a patanɨ pë xëprarema. Pë ha xëya hërɨnɨ,
Amahiri pë kuprarioma. Ɨhɨ misinɨ Amahiri pë xëye hërɨma. Hei pëma kɨ rë
kurenaha kuwë pë hiraa. Temɨtemɨ të pë kohomowë.

Kuwë yaro hei a kerayoma. Ɨhamɨ kamiyë pëma kɨ no patama rë hare, mɨ amo
hamɨ, ai të pë rë kui ɨ̃hɨ xëamorewë xoaonowei pë rë kui, pë puhi rë
mohotionowei rë kuonowei të pë kuo ha kunoha, pëma kɨ kuami, hei a urihi
ha.

Kihi a rë kui, kihi parimi a rë kui, a hëtarioma makui, kihi a
hëtaritayoma. Tukutuku a pata hëtariotayoma. Hei a rë kui a ha
horepɨonɨ, a kerayoma.

Ɨhamɨ pëma kɨ rë pëtore, të pë pararayoma. Hei a urihi ha, hei a rë kui
ha, mahu të pë rë hëtarionowei. Mahu a yahi pata hëtarioma. Ɨhɨ pëma kɨ
kupropë, ɨ̃naha pë mori kuonomi. Hei a rë kui hamɨ, pëma kɨ no patama
pararayoma, wawërayoma. Ha parahërɨnɨ, kamiyë pëma kɨ no hekama
kuprarioma.

Hei a rë kuinɨ, kamiyë pëma kɨ no patama xëaɨ haikionomi. Ai të pë rë
kui, të pë xëyë hërɨma. Kama pë rii Amahiri hiraopë, hëyëha. Amahiri pë
wãha kua, pë rë xëparenowei!

Tirewë makui, pëma kɨ kua. Hei kurenaha, makayo kɨ kuo mao ha kunoha, kɨ
pata ĩtao mao ha kunoha, pëma kɨ no patama kuami, mori hanomi. Ɨhɨ kɨ
kua yaro, të pë moyaweoma.

Moyawë yaro, kihamɨ të pata kimotayou tëhë, të si wëtɨkɨpraaɨ ha, pë noã
waxurema. Kihamɨ ai të pë mohoti no prepramoma.

--- Pei, huya pë, ipa wama kɨ rë kui, hei makayo kɨ rë kui, pëma kɨ
tokupë, mɨ amo yai hamɨ pëma kɨ rë kui, ya puhi mohotimi. Wama kɨ xëaɨ
maopë, ɨ̃hɨ pëma kɨ mahu rë hëtore, a urihi hamɨ, pëma kɨ no mɨhɨpropë.
Hei kɨ rë kui, kɨ ta wawëahe! --- e kumahe --- Kɨ no mayo ta takiohe! Yo
ta hõkɨhe!

E ha kuikuhenɨ, makayo kɨ të wawëaremahe, pë tokuu puhiopë yaro. E kei
kuketayomahei, hei a aheteprou tëhë kɨ kope kuo waikioma. A kei
aheteprou tëhë, të pata ha hëtɨrutunɨ, kama a rohote yaro, a hëtɨrayoma.
A ha hëtɨrɨnɨ, a hëtɨɨ tëhë, pë rë tokure, pë rukërayoma.

--- Pei! Haɨmohe! Ai të henaha, ai të henaha! A pata pitaɨ kurakiri,
ɨnaha a pata tireo kuo tëhë --- e kuma.

A ha kunɨ, a pata kerayoma. Xëapëteri pë rë kui, pë xëparema. Ɨhɨ pë
përɨoma. Xëapëteri pë xëparema. Kama pë wãha rë kuonowei. A ha kerɨnɨ,
pë rë hëpraruhe, pë xëyë hërɨma, makayo kɨ ha pë kua yaro, pë hɨ̃kɨanomi.
E pata hietiye kiriomahe, kuaaɨ tëhë, kama a wãha rë yehiponowehei, ɨ̃hɨ
hei a rë tuyënowehei, kama pë wãha kuoma. Ɨhɨ a wãha yehipomahe. A tuyëɨ
puhiohe yaro, pë wãha kuoma, kama a wãha rë kurenaha, pë wãha kure, pë
xëparioma.

A kei tëhë, a napë praɨpraɨmopɨmahe, ihirupɨ oxe pe heri ɨ̃naha a napë
kuaamahe. A napë ximɨkema, pë hɨɨnɨ. A no hëtɨpɨma mai ha, kurenaha a
pata hëtɨpɨmarema. Kama a yai rë hëtɨmarenowei, ɨ̃hɨ kama a wãha rë
yehiporenɨ, a yai hëtɨmarema. Ɨhɨ a wãha Hũtukarariwë a wãha kuoma. A napë kea xoakema, a ĩyë no preaaɨ makui, a rë hanɨɨwei, pei mamo kasi kɨ
hamɨ. Kreti! Kreti! Kreti! --- a tama.

A no preaama. Ɨhɨ a wãha Hutukarariwë kuoma. Ɨhɨ a rë hëtɨmare hamɨ, pë
harayoma, hei a ha. Ɨhɨ mahu a rë hëprore hamɨ, pë harayoma. Pë ha
harɨnɨ, ai të hare ha kunomai! Pë mahu ha harɨnɨ, kamiyë pëma kɨ no
patama raroma. Rarou xoao hërɨma. Ɨnaha të kuprarioma. 

--- Ɨnaha të pata yoprao kuo tëhë --- pë hɨɨnɨ pë noã tama. Kɨ noã
tapɨma, pë hɨɨnɨ. --- A napë ta patotoa xoaikuhe xë! Pei a napë ta
ɨramorɨhe xë!Mamikãi kãi tuyëatihehë xë!

Ɨhɨ tëhë, pëma kɨ no patama ha pararɨnɨ, Waika pë kãi kuprarioma. Pë no
patama xereroma. 

Kamiyë pëma kɨ mɨ amo hamɨ pëma kɨ no patama harayoma. Mɨ amo hamɨ, a
suwë rë kepraruhe a hamɨ, kamiyë pëma kɨ mɨ amoprarioma yaro, pëma kɨ no
patama mɨ amoprarioma. Mɨ amo hamɨ, a keprarioma, mɨ amoprou xoarayoma,
mɨ amo hamɨ pëma kɨ wãha hirapehe yaro. Ɨnaha pëma kɨ kuprarioma. 

Ɨhɨ pë wãha rë kui, kama pë xĩro rë përɨonowei, ai pë xoaa. Kamiyë pëma
kɨ no patama, napë pë no patama kãi, waiha opi pë pëtou, waiha të
warokei. Ɨhɨ te he tikë ha, hei yahë pë wãha rë waikare, ɨ̃hɨ të maxi ha,
mɨ amo ha, Hemarewë a hekura përɨoma. Pata të wãha, hapa Hemarewë.
Hekura kɨpɨ wãha. A hekura rë përɨonowei, pei kë a wãha Hemarewë kuoma,
hapa yai. Ɨhɨ a hekura përɨo mɨ hetuoma kutaenɨ pei a wãha Hemarewë pata
totihiwë. Urihi a kãi hapa përɨaɨ mɨ hetuoma kutaenɨ, a wãha yuamou,
pata. 

Hei kama të pë no patama rii rë pëtouwei të pë wãha. Hei kamiyë pëma kɨ
mai! Të ã yaia, kamiyë pëma kɨ no patama rii rë kuprarionowei. Kama pë
no hekama rii rë përɨhɨi, të pë wãha rii. 

Hei kamiyë mɨ amo hamɨ të ã kua. Ɨhɨ te he tikë hamɨ, hei ai xapono. Të
pë përɨhɨmou mao tëhë, kama të pë rë pëtarionowei. Të pë rë përɨonowei,
ai a xoaa. Ɨhɨ pë wãha. Ɨhɨ të rë kure hamɨ ai xapono, pë rë përɨonowei,
ɨ̃naha të pë kuoma. Hei ai pë waikou waikioma, hei pë rë hëaaimati, pë ha
maprarunɨ, hëyëha kamiyë pëma kɨ no patama rë kuprarionowei, ɨ̃hɨ tëhë mɨ
amo ha të yai kuprarioma. 

\chapter{O sangue de Lua}
 
\letra{N}{o início}, os dois que flecharam Lua também existiam, antes de nossos
antepassados Yanomami se misturarem. Eram espíritos. O irmão mais novo,
Uhutimari, morava com seus irmãos. Eram somente eles, junto ao seu irmão
mais velho, Escorpião.\footnote{\textit{Uhutimari} também designa um tipo de escorpião} Eram três. O mais velho tinha o nome daquele
inseto que faz doer muito, o escorpião. Por isso, chamava-se Escorpião.
O do meio tinha o nome da árvore paricá.\footnote{A árvore paricá fornece as sementes e a casca com as quais os Yanomami
produzem um pó alucinógeno utilizado em diversos rituais.} 

Quem realmente flechou Lua, o verdadeiro flecheiro de Lua, foi
Escorpião. 

Por que o flechou? Naquela época, Lua ficava baixo, sentado na
terra. Sendo muito faminto de carne, devorava sempre as crianças.
Devorou o filho de Paricá. 

Era alta, assim como hoje? Não! Perto da casa de Escorpião, erguia-se um
jatobá reto, onde Lua se empoleirou desajeitadamente. Ele se sentou em
uma forquilha baixa. Paricá e seu grupo moravam junto com os dois
outros, Escorpião e Uhutimari. Lua devorava as filhas quase formadas,
os filhos quase crescidos, as filhas quase moças. Comia as crianças
dessa faixa etária. 

Como as queimava, chamando assim o ódio de todos, os dois irmãos
mais novos quase o flecharam quando ele desceu para
atacar. Paricá se deslocou de \textit{wayumɨ} e ensinou aos
demais a se deslocarem de \textit{wayumɨ}.\footnote{Longas estadias coletivas na floresta. Em geral são motivadas pela falta de comida no xapono. A comunidade pode se dividir em vários grupos quando se trata de um xapono populoso, e se desloca num vasto círculo, fazendo acampamentos sucessivos.} É por isso que hoje os Yanomami
vão de \textit{wayumɨ} até onde há o \textit{bacabal} para se alimentar. Os irmãos iam
de \textit{wayumɨ} e, assim, nos ensinaram. Foram por lá.

O xapono deles era como o nosso. Ele chorou como nós choramos, ficou
muito abalado. Depois da cremação do corpo do segundo filho, que Lua
comeu em seguida, Paricá cobriu as cinzas no meio do xapono. Saíram
de \textit{wayumɨ}. 

Em determinado momento, um dos integrantes teve de voltar correndo,
tendo esquecido os dentes de cutia, outros ficaram sentados a certa
distância. Quando chegou à entrada do xapono, ele viu Lua comendo as
cinzas no meio do xapono, que a gente sempre mantém limpo. A massa de
Lua se mexia.

--- \textit{Ũũũũũ} --- fazia um ronco assim. 

Ele comia até o carvão, devorava as cinzas com gula. 

\textit{Hɨ}! Ele ficou com medo e rapidamente recuou: 

--- Será que o monstro grande está fazendo isso? Ele está comendo? O
monstro está comendo as cinzas! \textit{Sãrai}! --- disse, recuando de
medo. Ele foi buscar e avisar o pai da criança morta, Escorpião, que
fará Lua sofrer as consequências. Ele buscou o pai. 

--- O monstro grande está comendo lá! Ele está comendo o que te deixou de
luto. Ele está devorando os restos, ele está comendo as cinzas
do seu filho. 

--- Ele está comendo as cinzas do meu filho!? --- perguntou o pai, desolado e
chorando. 

--- Vamos! Vamos! Vamos, meu irmão! --- disseram os dois irmãos mais
novos, apesar de eles não serem bons flecheiros. --- Já aprontamos as pontas de nossas flechas. 

Escorpião observava os dois flechando em vão, pensando que eles não
conseguiriam, apesar de Lua não estar muito alto, pois os dois eram
péssimos flecheiros. 

Lua se empoleirava e olhava para si mesmo, porque tinha comido o
menino. Lua estava mole, digerindo mal. 

\textit{Taɨ xiri, taɨ xiri, taɨ xiri}! Os dois estavam flechando, mas suas
flechas, infelizmente, não acertavam o alvo. Fizeram Lua subir,
espantaram-no. Fizeram-no subir, de tantas flechas que atiraram. 

Ele ficou altíssimo, rodando e subindo, e os dois insistindo. \textit{Tai,
tai, tai, tai!} Lua fez as flechas se tornarem espíritos. Por fim,
Escorpião, o pai da criança morta, conseguiu vingá-la. 

Nossos antepassados não sabiam fazer guerra, foi ele quem nos ensinou. 

Lua subia em direção à sua casa, sua rede estava lá, lá em cima. A sua
casa e a sua região estavam escondidas. 

Quando Lua, que era diferente desta, passou pela porta, ele o flechou.
Quando entrou, estava cansado e deitou-se lentamente na sua rede. 

Escorpião se moveu, erguendo-se e ao mesmo tempo apontando a flecha
para cima. 

Quando estava pronto para entrar, quando Lua ia se sentar na sua
rede: \textit{prãoo, kroxooo}! Ele não falhou: apesar da altura, ele
acertou completamente. Apesar de o vento sempre soprar muito nessa região,
a flecha não desviou, a flecha voou direto através do vento. Quando Lua se inclinava para se deitar, a flecha se fincou entre as duas
escápulas. Escorpião o fez balançar. O sangue jorrou. \textit{Ho, ho, ho,
ho, ho, ho, ooooooo}! Lua! \textit{Taka, taka, taka, taka, taka,
taka}! \textit{Ha}! Ele o fez tremer. 

O sangue caído, as gotas de sangue caindo de lá para cá não se
estragaram. O sangue caía se transformando logo em gente, mas em
gente feroz. O sangue se transformou em Yanomami, que imediatamente
flechavam. As gotinhas de sangue voaram sem se espalhar bem. O sangue
desceu flechando e não se esgotou. Os Yanomami, formados a partir
do sangue de Lua, mataram os habitantes do xapono do flecheiro de Lua. 

Ninguém sobreviveu, nem Escorpião, que se tornou espírito. Os dois que
conviviam com Paricá tampouco sobreviveram. O sangue de Lua queria se
tornar Yanomami; queria se tornar Yanomami ferozes. Queria se tornar
matadores de Yanomami. Aconteceu, assim é. 

Foi então que surgiram nossos antepassados, a partir do sangue de Lua. 

Assim aconteceu: o Sangue de Lua exterminou todos os que moravam em
baixo. Somente os espíritos sobreviveram, apesar de eles viverem como
nós. Os Yanomami Sangue de Lua não guerrearam com os espíritos.

Pouparam Kasimi e seu neto. 

\chapter{Pẽripo ĩyë}

\letra{P}{ẽripo} a rë nianowei, ɨ̃hɨ hapa kɨ kãi përɨpɨoma. Kamiyë Yanomamɨ pëma kɨ
yai rë kui no patama pë koyokoo mao tëhë, Pẽripo a rë niapɨnowei, ɨ̃hɨ kɨ
kãi kupɨoma, përɨpɨoma. Hekura kuoma. Kama oxe Uhutimari e pë kãi
hiraoma. Kama pë xĩro hiraoma. Suhiriwë pata, oxe ɨ̃naha e pë kuoma. Të
pë no nini he rë parohoi, ɨ̃hɨ patare e kuoma, Suhiriwë. Ɨhɨ rë a wãha
Suhirina kuoma. Ɨhɨ rë të pë wãha rë yehiponowei, ɨ̃hɨ rë a wãha Yakuana
kuoma. Suhirina, Uhutimariwë oxe, ai a, ai a, ai a, ɨnaha e pë kuoma,
kama pata. Kɨpɨ rë përɨpɨonowei, ɨ̃naha të kuoma. 

Pẽripo a rë niaɨwei hei a yai, Pẽripo niaprarewë, Suhirina. 

Exi të ha a niama? Kama Pẽriporiwë a yahatotoo tëhë, kama e pë waɨ yaro,
Yakuana a wãha rë kuonowei ihirupɨ wama, Pẽriponɨ. A naiki yaro.
Yahatoto ha kihi naha pita ha e të pata roo parɨoma yaro. 

Kihi a rë kui ɨnaha a kuoma? Hɨ̃ɨɨ! Kuonomi. Kama yahipɨ ahete ha,
pukature hamɨ të rë kurenaha, motua e ãhi wõroropë ha, e të pata
wahehetaoma. Yahatoto ha e të pata hãkioma. Ɨhɨ kɨpɨ kãi përɨpɨoma. Ɨhɨ
Yakuananɨ pë kãi rii përɨoma. Ɨhɨ pë tëë, suhe mo kɨ, ihirupɨ e mori
pataɨ, pë tëë mori mokou, ɨ̃naha e pë kuprou tëhë, e pë waɨ nokamoma. 

E pë iximama kutaenɨ, kama Pẽriporiwë a napë ha rurorɨnɨ, oxe kɨpɨnɨ a
mori niapraɨ makure, Yakuana a wayumɨ hokërayoma, të pë wayumɨ huɨ hiraɨ
ha, të pë wayumɨ huɨ, kihamɨ të pë iaɨ, hoko ma pë kuo pë hamɨ, të pë
wayumɨ hu hërɨɨ, kurenaha të hiraɨ ha, pë kuaama. Ɨhamɨ pë ukukema. 

Ɨhɨ hei ipa xapono kurenaha, pë xaponopɨ kuoma. Kamiyë pëma kɨ ɨ̃kɨɨ rë
kurenaha, a ha ɨ̃kɨrɨnɨ, puhi kãi no preaama. Wakë ha yëarɨnɨ, ihirupɨ a
rë nokaaɨ kõowei ha, ĩxino heha yohopa hërɨnɨ, a hokërayoma. 

Kihi naha ai tëka hikari rë prare naha kama e tëka praa, kihi naha të pë
kuke herayoma, roa kupe herayoma. Ai të rë rërëɨwei hei kurenaha, tomɨ
nakɨ nohi rë mohotuaɨwehei, të rërëimama, rërëimama, rërëimama. Pei yo
ka ha a mori kutou tëhë, hei kure naha, të rë kure, të rërëimama. Xapono
mɨ amo a yai ha, ĩxino a wai ha, ɨ̃hɨ, einɨ Pẽriporiwë a iaɨ tararema,
ĩxino a ha. Xapono ha të pë ma rë wawëtoi. Kihi yo ka rë kurenaha, yoka
kua yaro. Pẽriporiwë a iama, ĩxinoma a ha, xapono a mɨ amo ha të pata
hamorimoranɨ: 

--- \textit{Ũũũũũ…} --- të pata kuma. 

Ẽxehẽroma kɨ makui, kɨ wama. Të pata wëhërɨwëhërɨmoma. Kuwë ha:

\textit{Hɨ!} A kirirarioma, a kiriri he tatoprarioma. 

--- Yai rë të pata iaɨ ta yaipiyei, ei rë wa të no pata kiriaɨ nosi ë!
Yai të pata, ĩxinoma a ha, yai të pata rë iapiyei! \textit{Sãrai!} --- e kua mɨ
yaparayoma. Ɨhɨ kama pë hɨɨ a kõaɨ ha, a yɨmɨkamapë. Suhiriwë a
yɨmɨkamapë. Ɨhɨ Suhirina iha, ɨ̃hɨ ihirupɨ, wama yaro kutaenɨ, a noa
prearema, ihirupɨ. A kõrema.

--- Pei! Yai të pata ia harayou. Hei wa wã no rë preohe, të pata iaɨ. Të
pata wëhërɨwëhërɨmou. Ɨhɨ ĩxinohë a waɨ --- e kuma. 

--- Ipa ĩxino a wa harayou --- e kuma, mɨa kãi. 

--- Pei, pei, pei, oxei, haɨmo! Pei, pei oxe, yai kë ũũũ! --- oxe kɨpɨ
kupɨma, e kɨpɨ nɨhɨtepɨmi makui. 

--- Pei! Ipa ya kɨ hĩihaɨ waikire! --- e pë xomi kuma. 

A no tapɨma mai! Oxe e kɨpɨ ma kupɨnowei, e kãi mori no mɨhɨpɨanomi,
kɨpɨ ninipɨo he parohooma. 

Hei kurenaha Pẽriporiwë a wahehetaoma, a mɨprou ha, ihirupɨ e warema
yaro. A ëpëhëoma kutaenɨ, a xi wãrima. E kɨpɨ nini prepɨamoma. Hei naha
të pata kuu makui:

--- \textit{Tai xiri, tai xiri, tai xiri!} --- heinaha të pë xerekapɨ pata
hãrokaɨ kuaaɨ tikooma. Ɨhɨ të rë kui a tiremarema a ha yaxuprapërɨnɨ. 

A pata tirepɨmarema yakumɨ, xereka pënɨ hei a pata rë yaxupɨre, të pata
rë hamorimoimati, kuaa hërɨɨ tëhë. \textit{Tai, tai, tai, tai!} Xereka rë pë ma
kuonowei, hekura rë e pë pehi pata hãrokoa xoarayoma. Hei a rë kuinɨ, pë
hɨɨ a yainɨ, a no yurema. 

Kamiyë Yanomamɨ pëma kɨ no patama niayonomi makui, ɨ̃hɨ të pë hirama. 

Pẽripo a tirewë makui, hei kama pëkɨ kihamɨ pë kɨ kuotayoma, ei yoka ma
rë kurati, yoka kua, kama yahipɨ, urihipɨ pesia yaro. 

Hei yoka kurenaha, inaha e yoka kua yaro. Ɨhɨ yoka kua xoaa. Yoka tapoma
makui, hei a rë kui a kuo mao tëhë, a niama. A hatayou tëhë, a
waximirayoma, kama pëkɨ hamɨ, a përɨo ka kuaa hërɨma. 

Hei e rë hokëtore, e rë hokëtou nokare, ɨ̃hɨ kama pëkɨ ha a përɨaɨ tëhë a
nokaaɨ puhiopë yaro, a pehi mɨ nonomarema. 

Kama e yoka ha të pata ha horeikutunɨ kama pëkɨ hamɨ, të pata tipëatayou
tëhë: \textit{Prãoo, kroxooo!} Yai hamɨ e kãi morokõnomi, tirewë makui, xereka e
kãi hawënomi. Watori hiakawë a kua makurati, xerekapɨ kãi yarënomi. Ɨhɨ
watori a pata huxomi hamɨ xereka e morokoa katitirayo hërɨma. Pei pëkɨ
ha të pata wëkëatayou tëhë, pei a përɨatayou tëhë, mapuupëka pata
nokaretayoma. A pehi kãi pata wahehiamatayoma. Ĩyë kɨ pata rë
hirekerati. \textit{Ho, ho, ho, ho, ho! Ooooooo!} Pẽripo! \textit{Taka, taka, taka, taka,
taka, taka! Ha!} A pata porepɨ tamama. 

Ɨhɨ rë ĩyë kɨ rë kerati, mokure kɨ rë kui, ĩyë kɨ wãrimou kateheo maopë,
ĩyë kɨ kei kurati. Mokure të ĩyë pata kerayoma, hei a pita hamɨ. Ɨha ĩyë
ketayou yaro, ɨha ĩyë pata niayou rë xoarati, niayou kuketayoma. A ĩyë
rë keimati hamɨ, Yanomamɨ kurenaha, kama ĩyë yanomamɨprou xoarayoma.
Pẽripo ĩyë rë kui ĩyë nakaxi, remaxi yëɨ kateheonomi. Ɨhɨ ĩyë kãi niayou
itorayo tayoma. Ĩyë kãi mapronomi. Ɨhɨ ĩyë kɨ kepë ha, hei a rë niare,
kama yahipɨ haikiarema. 

Ai a hëpranomi. Kama Suhiriwë a kãi temɨ kutonomi, pei a no uhutipɨ
hekura hurayoma. Pei a no uhutipɨ xĩro rë hõriprariowei, a hekura
kuprarioma. Kama kɨpɨ rë kui Yakuananɨ kɨ kãi rë përɨpɨawei, e kãi
hëtonomi. A temɨ kãi përɨo hëonomi. Ĩyëpë yanomamɨprou puhio yaro.
Pẽripo ĩyë Yanomamɨ xëtimi pë kuprou puhio yaro. Të kuprarioma. Ɨnaha të
yai kua.

Kamiyë yama kɨ ĩyëpë ha ta kuio! Të pë puhi kuu maoma. Kamiyë pëma kɨ no
patama, ɨ̃hamɨ pë rë pëtouwei të kuprarioma, Pẽripo ĩyë hamɨ. 

Ɨnaha të kua: Pẽripo ĩyënɨ pepi hamɨ pë rë përɨonowei, pë haikiaremahe.
Hekura pë xĩro hëprarioma. Hei kurenaha pë xĩro përɨoma, pë kua makure,
hekura pë niaonomi. 

Ɨhɨ pë kãi hëpraremahe. Suhiriwë yesipɨ, hekamapɨ, ɨ̃hɨ e kɨ wai
hëpɨpraremahe, Yakuana nɨɨpɨ, hekamapɨ.

\chapter{Kasimi e o seu neto}

\letra{K}{asimi} e seu neto foram até os espíritos para não viverem como viviam
os Yanomami. Os espíritos viviam escondidos no mato. Todos foram
exterminados pelo Sangue da Lua.

Ou seja, a avó e o neto conseguiram alcançar os espíritos. Ela está lá ainda. É assim, não morreu. Aonde vão os espíritos, ela também vai, ela os
alcançou. Ela chegou ao lugar dos espíritos para se tornar eterna. 

Kasimi alcançou o xapono dos primeiros espíritos, os Ihiruwëteri, nome
de espíritos. O primeiro xapono dos espíritos era esse, o das
Crianças, Kasimi chegou a eles, ela chegou aos
Oxemawëteri, os Jovens, que é outro nome deles. 

Kasimi alcançou a moradia dos Parawari Yokënamari, todos solteiros, e que
estavam dançando. Ela chegou durante uma festa, e, assim, nos ensinou a
fazer festa. Kasimi era o nome da mãe daquele que flechou a Lua.

Os antepassados, que moravam espalhados, não resistiram aos Yanomami
oriundos do Sangue da Lua, de onde nós viemos. Isso se fez para nós
brigarmos, para nós guerrearmos. 

Quando começaram esses eventos, os antepassados logo ficaram espertos,
que antes não eram. Quando brigavam, era como uma dança, e
simplesmente não paravam de rir. Quando havia guerra, não sabiam reagir
e {só} faziam pajelança. Antes, eles não se vingavam.
Durante a guerra, eles festejavam; era isso que eles faziam! 

No início, havia os que ensinaram os Yanomami a morar. Os espíritos
existiam e eram parecidos com os Yanomami. Os espíritos não foram obra
de ninguém. Eram assim, como os Yanomami. O xapono deles era tão limpo
como o meu, não era fechado. Moravam juntos. 

Dizem que moravam assim sem ninguém os ter ensinado. Eles viviam em um xapono
igual ao meu. Eles andavam sempre no limpo. Eles faziam amizade,
conversavam e se visitavam. Se não agissem assim, teriam nos exterminado
há muito tempo, pois eles nos comem. Se eles morassem ainda no limpo,
todos vocês, rapazes, cantariam: \textit{ea, ea, ea}! Todos vocês seriam
pajés. 

Hoje, os espíritos não são mais visíveis, pois não moram mais no limpo.
Eles dançavam, faziam festas no limpo, como os Yanomami. Eles dançavam
como dançam os Yanomami, no limpo. 

Faziam os rituais de \textit{himou} e de \textit{wayamou}, cantavam como os
pajés cantam.\footnote{O \textit{himou} é uma modalidade de diálogo cerimonial usada para trazer notícias, ou fazer um convite para uma festa. O \textit{wayamou} é um diálogo cerimonial realizado à noite por um hóspede e um anfitrião por ocasião de uma visita, destinado reforçar ou restabelecer relações pacíficas entre dois xaponos.} Eles cantavam assim. Eles também brincavam de roubar
esposas, como fazemos durante as festas. 

Moravam na planície, em terra plana, não moravam naquele tipo de
montanha. Depois, eles foram morar lá nas montanhas, foram logo assim. 

Como eram todos gente que morava no limpo, a imagem da minha avó, apesar
de ser espírito, ainda alcança os pajés, porque ela era Yanomami. A mãe
daquele que flechou o monstro chegou onde moravam os matadores de
monstros, que moravam no limpo. Ela deve ter chegado enquanto eles ainda
eram visíveis. 

Aquela que fechou a casa dos espíritos se chamava Kasimi, mesmo. O segundo nome dela era Maxikomi. 

Ela carregava um grande cesto. Não existia porta grande como essa. Ela
carregava esse cesto andando no caminho dos espíritos. 

Enquanto eles olhavam para Kasimi, o cesto apareceu. Ela não tinha
o cesto até esse momento. Depois de o cesto ficar visível, apesar de
haver uma grande entrada, o cesto não passava pela entrada. 

Ela tentava entrar com o cesto, que bloqueava a entrada, por isso os
espíritos deram uma gargalhada. Enquanto estavam rindo, ela se mexia
para conseguir entrar. A casa dos espíritos estava se fechando devagar,
fechando devagar e o xapono acabou fechando totalmente. 

O xapono onde moravam os espíritos e cuja entrada ficou fechada tinha
nome: Yoararopɨwei. Esse xapono se chamava Yoararopɨwei. Era muito
bonito. Gostavam muito dele. Apesar de ser um xapono, ele era muito
brilhante, como um espelho, possuía uma luz própria. Colocaram o nome de
Yoararopɨwei.

--- É o xapono de Yoararopɨwei --- diziam. 

Kasimi chegou lá. 

--- Meus queridos! Meus queridos! Estou chegando com meu netinho de um
grande sofrimento. Esperem-me! Esperem-me aí! Abram a entrada! --- dizia ela,
vindo.

--- Ó! É a voz de quem? Quem é, será?

--- Meus queridos, abram a entrada! Estou chegando com meu netinho!---
disse, vindo. --- Eu estou chegando e sofrendo de fome! Agradem a meu
netinho! O único que restou, agradem-no! Ele é meu único!

--- Quem é você? De quem é essa voz? 

Eles queriam que ela pronunciasse seu nome.\footnote{Os Yanomami, tradicionalmente, não podem chamar uns aos outros por seus nomes próprios, o que lhes causa constrangimento, e por isso usam termos de parentesco. Quando não há consanguinidade, são usados termos de afinidade, como cunhado ou sogro.} 

--- Quem pode ser? Quem é você mesmo? Essa voz de mulher, de quem pode
ser? \textit{Ãaaaaaõooooo}! --- disseram. 

--- Sou Kasimi! Sou Kasimi! Queridinho, não pergunte quem
sou! Sou Kasimi!

Escutava-se o som de seus pulos. Infelizmente a entrada fechou. Ela fez
assim, como quando alguns ficam presos na
cadeia. Foi assim. 

A história dos espíritos foi obra de alguém? Não pensem assim! Não foi
obra de ninguém! Ela aconteceu através de Kasimi. Essa é a verdade! 

Eles também comiam, comiam banana-pacovã, faziam festas no tempo da
pupunha, faziam também beijus, sabiam caçar, comiam anta, quando faziam
festas; era assim que viviam os espíritos, no início. Tiravam lenha,
assim faziam. Plantavam bananeiras, enquanto moravam no limpo,
ensinando-nos, assim, a plantar. Nós continuaremos a plantar os
alimentos como eles os plantavam. 

Atualmente, quase que nós não comeríamos pupunhas. Foram os Japiins que
espalharam as sementes de pupunheira. Não foram os antepassados
dos \textit{napë} que criaram essas pupunheiras. Eles não inventaram as
sementes. 

Omawë plantou pupunheiras, depois de inventar as sementes? Não, não foi
ele quem fez isso!

Quando nós desconhecíamos a pupunha, os japiins se agrupavam no chão, as
pupunheiras se erguiam perto daquele xapono cuja entrada ficou fechada. 

Somente eles faziam festas no tempo da pupunha. As pupunheiras não
existiam lá onde moraram Yoawë, Omawë, Ruwëri e Pore. Foram os Japiins
que criaram as pupunheiras. Os Japiins moravam com seus irmãos mais
novos, os Jaloacas. Foram eles mesmos que deram essas palmeiras. 

Não foram os Yanomami que conseguiram as sementes para podermos comê-las
hoje. Não comemos pupunhas hoje devido a um antepassado
dos \textit{napë} que as tenha feito aparecer. Elas nos foram
dadas. Foi Japiim quem as conseguiu. A pupunha se espalhou. Foi
assim. Na verdade, foi assim. 

Apesar de serem espíritos, foram eles que ensinaram os Yanomami a fazer
festas. Os Yanomami seguem o ensino da festa daqueles dois que faziam
festas, mesmo sozinhos. Eles moravam na região central. A partir daí,
nós faremos festas. 

Ele nos ensinará a fazer a luta de \textit{yaɨmou}.\footnote{\textit{Yaɨmou} é uma luta cerimonial realizada em festas de aliança}. Qual era o nome
desses dois? Será que alguém ensinou a vocês o nome desses dois? Se
um \textit{napë} perguntasse o nome dos dois que moravam juntos, alguém
diria o nome dos dois? Como eles se chamavam? Esses dois
ensinaram a festa e o ritual do \textit{yaɨmou}. 

\chapter{Kasimi}
 
\letra{H}{ekura} pë iha, hei kurenaha pë kuopë ha, kɨ waropɨopë, kɨpɨ
hëpɨprarioma. Waropɨkema, awei, kɨpɨ si rë poaɨwei. Ɨhɨ hekura pë iha, a
rë waroore, Yanomamɨ pë përɨaɨ rë kuonowei naha, a kuprou maopë. 

Hekamapɨ kãi warokema. Ɨhɨ Pẽripo ĩyënɨ pë no rë watëɨwehei, pë yesinɨ
hekamapɨ a kãi warokema, hekura pë iha.Ɨha a kua xoaa. Ɨnaha të kua. A nomanomi. Hekura pë huɨ tëhë, a hupë, kama a warokema. Parimi a rë
kuowei, a warokema. 

Hekura, ɨ̃hɨ hapa pë rë kui, Ihiruwëteri pë xaponopɨ kuopë ha, Kasimi a
warokema, hekura pë wãha. Ihiru të pë yosika rë rararenowehei pë yaro,
pë wãha Ihiruwëteri kua, hekura. Ɨhɨ pë iha Kasimi a warokema. Hekura
Ihiruwëteri, Oxemawëteri pë iha a warokema, Kasimi a warokema. 

Parawari Yokënamari pë yai hiraopë ha, xĩro të pë pata praɨpraɨpraropë
ha, Kasimi a yai warokema. Kamiyë pëma kɨ reahumou hiraɨhe tëhë, pë
reahumou tëhë, e warokemahe. Kasimi a wãha kuoma. Ɨhɨ Pẽripo a rë
nianowei nɨɨpɨ wãha. Pë nɨɨ e wãha Kasimi kuoma. Ɨhɨ a hëprarioma, yami
xĩro, hekamapɨ xo. 

Ɨhɨ ĩyë ha yanomamɨprarunɨ, hei kurenaha, ĩyë ha pëtarunɨ, ai të pë
hëpranomihe, hei pata pë rë përɨhonowei. Pëma kɨ xëyopë, pëma kɨ
niayopë, të tapraremahe. 

Hapa ai të rë mohotimouwei, të kuonomi. Hapa të pë puhi mohotioma. Të pë
xëyou tëhë, praɨɨ kurenaha, të pë ka xĩro ĩkaprou pëotima. Të pë ma
niaɨhe tëhë, të pë imɨkɨ kãi rërëkëapraroma. Hapa të pë no yuayonomi. Të
pë ma niaɨhe tëhë, të pë kãi praɨma. Ɨnaha pë kuaaɨ pëoma. 

Ɨhɨ hei Pẽripo ĩyënɨ të pë rë niare a patamorayoma, Yanomamɨ hei të pë
kupropë.  

Kamiyë Yanomamɨ pëma kɨ përɨaɨ rë hiranowehei, hapa të pë përɨkema. Hapa
hekura pë rë kui Yanomamɨ kurenaha pë kuoma. 

Taprano hekura pë kuami. Kama xoati. Yanomamɨ kurenaha. Hei xapono ipa
kurenaha, wawëtowë yahi kurenaha pë kuoma kãi, kahuhuwë mai, pë
hiraoma. 

Kama pë rë përɨohe hõra, ai tënɨ pë rë hiranowei, hekura pë kuami. Hei
kurenaha ipa xapono hamɨ Yanomamɨ kurenaha pë kuoma. Wawëtowë pë kãi
hutima. Pë nohimoma, pë ã kãi wayoma, pë kãi hama huma, kuaaɨ tëhë, pë
kuoma makui, ɨ̃naha pë kuo ha kunoha, kamiyë yëtu hëmɨ, pëma kɨ
haikiareihe. Pëma kɨ rë waɨwehei. Ɨnaha pë kuo xoao ha kunoha, hei huya
wama kɨ rë kui. \textit{Ea, ea, ea!} Hɨtɨtɨwë wama kɨ kuɨ. 

Pë wawëtowë kua yaro. Hei kuikë pë wawëtoami. Yanomamɨ kurenaha pë kãi
praɨma, praɨama, wawëtowë, pë kãi reahumoma. Yanomamɨ praɨpraɨmou
wawëtowë kurenaha pë kuaama. 

Pë kãi himoma, pë wãyamoma, ɨ̃hɨ pë rë hekuramore, amoa kurenaha pë kuma.
Amoa a rë taɨwehei ɨ̃naha pë kuma. Pë hesiopɨ kãi hãkɨoma, pë kuaama. 

Wawëtowë pë hirao tëhë, yarɨta ha. Kama pë yarɨtaoma yaro, kihi ma pë rë
kui, ɨnaha të pë kuonomi. Ɨhɨ kihi kama pë yahipɨ hamɨ pë përɨhɨrarioma.
Ɨnaha pë kurarioma. 

Hekura taprano pë mai! Wetinɨ pë taprarema? Kama xoati pë Yanomamɨ rë
përɨkenowei pë yaro, kuoma makui, pë wawëtowë kua ha, ɨ̃hɨ yayë a wãha no
uhutipɨ rë hirore, hekura a makui, a waroo xoa. Yanomamɨ a kuoma yaro.
Ɨhɨ yai të rë nianowei nɨɨpɨ ɨ̃hɨ pë xĩro kuopë ha a ha waroa taronɨ,
wawëtowë pë kuopë ha, e kõo kateheo ha maohenɨ. 

Pë pëka rë kahuprarenowei, Yanomamɨ a wãha, ɨ̃hɨ a wãha Kasimi kuoma.
Maxikomi ai a wãha. Pei pëka yai rë kahuprarenowei, Kasimi a wãha yai
kua. Maxikomi oraora a wãha. Korokoro a wãha yai Kasimi kuoma. 

Ɨhɨnɨ yotema a pata ha yehitarɨnɨ, hei yo ka rë kurenaha, yo ka kuami
makure, pë peipɨ yo pata rë haawei hamɨ të pata yehitarema. E pë mɨ
puruama, pë mɨ puruaɨ tëhë, e të pata pëtarioma. A yehiponoma mai! Të
pata ha pëtamarɨnɨ, të yosi ka pata hore kuwë totihiwë makui, e pata
hõkikeyoruma. 

E të pata ha hõkiikionɨ, të kãi pata rë hare, ɨ̃naha a kuaaɨ tëhë, e të
pata hõkikei ha, hekura e pë ka ĩkapraroma, no ka ĩkaprarɨ he, ĩkaprarɨ
he kurenaha e të pata yãɨka hërɨɨ tëhë, hekura pë yahipɨ ka
komaaimatayoma, komaa hërɨma, komaa hërɨma, komaa hërɨma, pë pë ka
kahuprarema. Kihi pei ma pë ma rë kõmii. Pë ka kahuprarema. Tei, tei
huxomi hamɨ pë tapramarema. Hekura pë huxomipramaɨ xoarayoma. Të yai
kuprawë. 

Kama hekura pë pëka rë kahumanowei, pë xaponopɨ rë kuonowei, pei pë
xaponopɨ wãha. Ɨha pë rë përɨonowei, ai të pënɨ pë wawëtowë rë
përɨonowei, pë rë reahumonowei, e wãha xapono yupraɨ taohe? Ɨhɨ xapono
Yoararopɨwei a xapono wãha, Yoararopɨwei. A riëyëhëo he parohoma. A nohi
toaɨ totihiomahe. Xapono a makui të xĩi pata hãtohãtopraramamahe. Mɨre
pë xĩi rë kurenaha e xĩi kuaaɨhe yaro, a wãha Yoararopɨwei tapomahe. 

--- Yoararopɨwei kë a --- pë kuma. Ɨha e warokemahe. Kasimi a warokema. 

--- Oxeiwë pëë! Oxeiwë pëë! Ipa xëtëwë! Ya no kãi wai preaaimi! Wamare
no ta tapa! Miha wamare no ta tapa! Yo ka ta karokɨhe! --- a kuimama. 

--- Õ! Weti kë a wã? Weti wa wã ta tawëëë?

--- Oxei, oxei, yo ka ta karopahe! Ya hekamapɨ wai pararuaɨ kë a kure!
--- e kuimamahe --- Ipa ya të ohiri no kãi wai preaaimi! Ipa të nohi wai
ta toahe! Ipa të wai mahu rë hëprouwei, ipa të nohi wai ta toahe! Ɨhɨ
ipa të wai mahu yaro. 

--- Weti kë wa? Weti kë wã ta tawë? --- e pë kuma. A wãha yupramapehe
--- Weti naha kuwë pei wa të ã yai ta tawë? Weti kë wa wãaaa ëëë? Weti
naha kuwë suwë wa të ã yai ta tawë? \textit{Ãaaaaaõoooooo}! --- e të pë pata
kuma. 

--- Kasimi kë ya wã, Kasimi kë ya wã --- e kurayomahe. --- Oxei weti ma.
Kasimi kë ya wã! 

E të pë mamikɨ pata haruharumopë. Pë pëka kahupraɨ tikorayoma. Napë pë
pëka kahumaɨ hiraɨ ha, pë pëka ma rë kahuowei, ai pë pëka mare kahuowei
kurenaha të taprarema. Ɨnaha të kuprarioma. 

Hekura taprano pë wãha! Pë puhi kuɨ mai! Kama taprano pë mai! Kama
Kasiminɨ ɨnaha pë yai taprarema. Ei të ã yai. 

Pë kãi iama, kurata pë kãi wamahe, raxa a kãi reahuamahe, naxi hĩ pë kãi
ramamahe, pë naiki kãi taoma, xama pë kãi wamahe, pë reahuamamahe,
hekura pë kuaaɨ parɨoma. Kãɨ ãxo pë kãi tamahe, pë tarɨkɨ taoma, ɨnaha
pë kuaama, kurata si pë kãi keamahe, hei kurenaha pë wawëtowë kuo tëhë,
pëma kɨ̃ hiraɨhe ha. Ɨhɨ pënɨ pëma kɨ ni rë keore, pëma të taɨ hëopë. A nii keamahe. 

Kihi kamiyë pëma kɨnɨ kihi raxa pëma kɨ mori waɨ hëonomi. Ɨhɨ pënɨ ɨ̃ha
ãyakorari pë iha mo rë piyëarahei hamɨ kihi të si kɨ, piyërenowei hamɨ.
Napë pë no patapɨnɨ raxa mo tapraɨ taonomihe. Mo kãi tapranomihe. 

Omawënɨ mo ha takɨ hërɨnɨ, mo keke hërɨma? Keanomi. 

Pëma kɨ tao mao tëhë, ãyakorari pita ha pë përɨoma, ɨ̃hɨ pë pëka rë
kahuprarenowei hamɨ, kihi të si pë. 

Ɨhɨ pë mahunɨ raxa a reahuamahe. Kihamɨ, Yoawë, Omawë, Ruwëri, Pore kihi
si kɨ kuonomi. Ãyakorari pë yaia kure. Ãyakoari, Kuyarori oxe pë xo, pë
përɨoma. Ɨhamɨ hei si kɨ rii piyëwa notiwa. 

Yanomamɨ të pënɨ kihi, mo ha piyëprarɨhenɨ, pëma të pë waɨ hëami. Napë
iha napë a patanɨ mo kɨ ha pëtamarɨnɨ, pëma kɨ waɨ hëami. Ɨhɨ piyëwa. Pë
aka praukurayoma. Ɨnaha a kuprarioma. Ɨnaha të kua, hei të yai. 

Hekura pë makuinɨ pëma kɨ reahumou hirakemahe, Yanomamɨ. Reahu a rë
hiranowei, pëma kɨ reahumopë, yami kɨpɨ makui, ɨ̃hɨ të mɨ amo ha, kɨ
përɨpɨoma. 

Ɨhɨ të he tikëa ahete ha, reahumorewë pëma të tapë, yãɨmorewë pëma të
tapë, a rë hiranowei. Ɨhɨ weti naha kɨpɨ wãha kupɨoma, ɨ̃hɨ kɨpɨ wãha
hirapɨkemahe, kama wama kɨ iha? Ai të pënɨ napë pë iha, të wãrihihe ha,
kɨ pata rë përɨpɨo mɨ hetuonowei, kɨ wãha yupɨamahe? Ɨhɨ weti naha kɨ
wãha kupɨa? Reahu të rë hirapɨnowei, yãɨmou të rë hirapɨnowei. 

\chapter{O pássaro siikekeata}
 
\letra{K}{eora}, \textit{kɨrɨɨɨ, keora kɨrɨ}! --- disse assim. Os dois fugiram ensinando a fazer o \textit{yaɨmou}. Apesar de serem somente dois, faziam festa, ensinaram a encher os cestos de fruta
conori. Não sabiam matar anta, porém tinham muita anta moqueada. 

Como se chamavam os dois avós de Siikekeatawë? O neto deles se chamava
Siikekeatawë. 

Siikekeã, \textit{keã, keã}! Vocês escutam esse canto de passarinho? Era o
nome dele. Não era Yoawë. Era outro irmão mais velho. Yoasiwë era o nome
do irmão mais novo. 

As famílias yanomami em geral são numerosas. 

Esses dois apareceram na sequência desta história. Sim, os dois
apareceram. Aquele que eles chamam Yoahiwë, aquele se
tornou \textit{napë}, os dois foram ao rio Tanape. Omawë e Yoahiwë se
tornarão \textit{napë}. Não foram esses dois que ficaram. Não foi Omawë, o
irmão mais velho e bonito. Nem Yoasiwë. Não foi aquele
passarinho \textit{uxuweimɨ} bonito que ficou.\footnote{Referência ao herói Omawë. 
Ver o volume \textit{Os comedores de terra}, nesta série.} 

Trata-se aqui de outros dois. Ficaram somente os de aparência velha e triste. Esses dois ficaram.

Onde fica a foz do rio, cuja parte inferior olhamos? Onde se encontram
os dois rios? Foi nesses dois rios que os dois se dividiram. É assim. É essa 
a história dos dois que se dividiram. 

Esses dois irmãos mais velhos ficaram, ficaram fazendo festa, ensinaram
os descendentes a fazer festa. Terminou a festa, a festa acabou quando
juntaram a comida, colocaram a carne de anta em cima dos conoris.
Enquanto isso, eles cheiraram paricá. Ensinaram o \textit{yaɨmou}, 
apesar de estarem sozinhos. Os dois conversaram, fazendo o \textit{yaɨmou} no meio do xapono. 

--- Sou teu irmão e pergunto, como vamos falar? Nós vamos discutir e,
depois, nos fazer comer mesmo! --- disseram os dois, ensinando. 

Disseram o que dizem os Yanomami quando fazem o ritual de \textit{yaɨmou}.
Os dois deram exemplo aos Yanomami. 

--- Vamos encher a barriga até cair! --- falavam brincando, como se
houvesse muita gente ao redor. 

--- \textit{Aë, aë, aë, aë, aë, aë, heeeee}. 

Enquanto dizia isso, o neto deles saiu levando o arco pequeno \textit{haowa}.
Ouviu a voz forte do pai. O menininho escutou a voz forte do seu pai, a
voz vinha subindo da curva do rio. Apesar de ele não ser de grande
tamanho, apesar de ele ser pequeno, a voz forte espantou os dois até os
expulsar. 

Enquanto flechava passarinhos, o neto correu até certa distância e,
enquanto flechava, a voz forte surgiu. 

Os dois avós faziam \textit{yaɨmou}, se batendo no
corpo. \textit{Hëë, hëë, hëë, haëëë, haëëë, haëëë}! (Eu mesmo sou surrado
pelos meus parceiros ao fazer o \textit{himou}!) 

Enquanto faziam assim, o neto rodeava para matar passarinhos. Nesse
momento, a voz grande surgiu. A voz não era fina. 

--- Siiiikekeã, \textit{kea, kea}, rasgar a pele, rasgar, rasgar! --- dizia
a voz, chegando. 

A floresta estava tremendo com essa voz. 

--- Siiiiikekea, \textit{kea, kea}! dizia a voz, chegando. 

Apesar de ser pequeno, ele estava andando ali, ouvindo essa voz. 

Enquanto os dois avôs discutiam, o neto retornou. Na entrada, soltou o
arco. Soltou-o de medo. 

--- Avô! Avô! Parem com esse barulho! A voz terrível do monstro se
aproxima! Ele vem rasgar a pele de vocês, a voz já está perto, avô,
parem! 

Quando os dois pararam, assustados, pararam de repente, a voz surgiu
naquele instante. Quando ficaram silenciosos, ouviu-se logo a voz
forte: 

--- Siiiiikekea, \textit{kea, kea}! 

--- \textit{Hɨ̃ɨɨ}! --- os dois gemeram, e se levantaram assustados. 

O que lhes aconteceu?

--- Vamos! Vamos, querido! O monstro vai rasgar nossa pele, parente
querido, vamos, vamos, depressa! --- disse o mais velho, cujo nome já
disse. 

Enquanto a voz dizia isso, o irmão mais velho também queria se
transformar. 

--- Vamos, venha, meu irmãozinho, venha, meu netinho! Vamos depressa! O
neto rodava na frente dos dois. 

--- \textit{Keora kɨrɨ}! \textit{Keora kɨrɨ}! \textit{Keora kɨrɨ}, vamos cair na água, lá
embaixo! --- os dois logo disseram. 

Os dois disseram isso, embarcaram e voaram acima da água. Lá, os dois
caíram rio abaixo. Enquanto os dois prosseguiam, a cauda vermelha do
neto saiu. \textit{Prohu}! 

O neto seguiu os avós, aqueles que voam. Por que ele tem esse nome de
Yoasiwë? Quando os passarinhos pousam em galhos fincados na água, a
cauda deles não é vermelha? O neto se transformou nesse passarinho. Onde
ele se transformou, onde foram os dois avós, a imagem do neto ficou, se
multiplicou e ficou voando acima das águas. Os dois avós caíram, levando
à frente deles seu verdadeiro neto, de quem restou somente a imagem.
Foi assim. 

Onde os dois avós caíram e onde a sua imagem ficou, se soterra o fim do
rio. É lá que os dois moram, eles não morreram e ainda moram lá. Eles
não morrem de doença. 

Os dois ensinaram o \textit{yaɨmou}, da mesma forma que faziam; é por isso que nós, Yanomami, fazemos festas. Nós perpetuamos os rituais. Não foram outros que nos ensinaram a fazer festa.

\chapter{Siikekeatawë}
 
\letra{K}{eora} \textit{kɨrɨɨɨ, keora kɨrɨ!} --- a ha kurɨnɨ, kɨ rë tokupɨrayonowei të
tamahe, ɨ̃hɨ kɨpɨnɨ yãɨmou të yai hirapɨkema, yami kɨpɨ makui kɨpɨ
reahumoupɨɨ ha kuponɨ, momo, a yami makui të pë yorehi pata taɨ hirapɨɨ
ha, kɨ no xamapɨ mi makui, xama të pë pata rë tapɨamanowei. 

Ɨhɨ Siikekeatawë xɨɨ e kɨ rë kupɨonowei, weti naha kɨ wãha kupɨa?
Hekamapɨ e wãha. Siikekeariwë e wãha kuoma. 

Siikekeã, \textit{keã, keã!} Wama të pë kiritamɨ kuɨ hiriaɨ tao? Ɨhɨ e wãha
kuoma. Ɨhɨ hei Yoawë, ɨ̃hɨ mai, ai rii pata yai, ɨ̃hɨ ai a rë kui oxe a
wãha, Yoasiwë oxe ai a wãha, ai pë marë Yanomamɨ tetei, kɨ rë përɨpɨtou
hërayonowei, ɨ̃naha kɨpɨ hëpɨtou kurayoma. 

Yoahiwë, të pë rë kuɨwei, ɨ̃hɨ kihamɨ hei të pë napëpropë, hëyëmɨ pata u
hamɨ, kɨpɨ hupɨnomi. Ai kɨ rii hupɨrayoma. Napë të pë kupropë. Hei kɨpɨ
rë hëpɨtore, Yoahiwë, Omawë pata a yai hei katehe a yai rë kui ɨ̃hɨ mai!
Ɨhɨ Uxuweimɨ si pë rë riëhëi ɨ̃hɨ mai! 

Të si pë puhi rë ahii, të si pë wai tapramou rë rohotei, ɨ̃hɨ hei kɨpɨ
përɨpɨa hëkema. Ɨhɨ pata e hëkema. Ei ai kɨpɨ rë kui, kihamɨ ai a
kurayoma, kihamɨ Totewë a kurayoma. 

Weti ha pei u kɨ he tikëpɨa kure yai? Pëma u pë koro rë mɨɨwei. Kama u
kɨpɨ weti ha u kɨpɨ he tikëa?Ɨhɨ u kɨpɨ hamɨ kɨpɨ rë xereronowei. Ɨnaha
të kua. 

Kɨ yai rë xererepɨprarionowei ɨ̃hɨ të rii. Ɨhɨ të rë kui hei pata e kɨ rë
hëpɨre, kɨpɨ reahumopɨɨ hëoma, notiwa të pë reahumou hiraɨ ha. Kɨpɨ
waikipɨprou, reahumou tiraprou, nii pë pata kõkapɨrarei, momo pë pata
heaka hamɨ xama pë pata makekepɨrei, kupɨaɨ tëhë, ẽpena a kopɨama. Të pë
yãɨmou hiraɨ ha, yami kɨpɨ makui. Kɨpɨ noã tayou, mɨ amo ha kɨpɨ
yãɨmapɨyoma, pata a makui xo. 

--- Hei aihë ya rë kui weti naha yahë kɨ kupë? Yahë kɨ noã tayou ta
kuparunɨ, yahë kɨ iamayou ta yaio këëë --- kɨ kupɨma, të hirapɨɨ ha. 

--- Weti naha Yanomamɨ të pë kuɨ, të pë yãɨmou tëhë? --- Të pë kuma? Ɨhɨ
e kɨpɨ kumahe. 

--- Pëhë e kɨ pëtɨrɨ praparei! --- kɨpɨ xomi kuma, ai të pë kuama mai
ha. Pë ĩtapɨpraapë pë kopemapɨma. 

--- \textit{Aë, aë, aë, aë, aë, aë, heeeee} --- e hore kuma. 

Kuɨ tëhë, hei hekamapɨ e horepɨkema. Haowa e ha hayurëma. Ɨhɨ ya e wãha
rë yuprarɨhe, e horekema. E wãha rë kuonowei. Ɨhɨnɨ kama pë hɨɨ a wã
pata hĩriapë. Ɨhɨ pë hɨɨ e rë kuonowei, ihirupɨ wainɨ e ã pata hirirema.
Pei e të ã mɨ pata yamou kuimi, e të ã pata ãyoriaɨ kuimi. Ɨnaha të kasi
pata kuwëmi makui, të wai ma ihirupɨwei, të huxomi ha preonɨ, kɨ rë
yaxupɨre të kupropë. 

Hekamapɨ e rërëpɨpe kirioma, kiritamɨ pë wai niaɨ tëhë, pë wai tɨhɨyëmaɨ
tëhë, e të ã pata pëtarioma, yaɨmopɨɨ tëhë, pë xɨɨ kɨ ã karëhopɨɨ tëhë,
kɨ si rë paipɨyouwei: 

--- \textit{Hëë, hëë, hëë, haëëë, haëëë, haëëë}, kɨ kupɨɨ tëhë, kiritamɨ e pë
napë tiporema. (Ware si marë yoaɨwehei, ware a himomaɨhe tëhë.) 

Kuɨ tëhë, të ã pata pëtarioma. Wã okaroonomi. 

--- Siiiiiikekea, \textit{kea, kea, keããããã!} --- e të pata kutoimama. 

Urihi kë a pata haruharumotamama. 

--- Siiiiikekea, \textit{kea, kea!} --- e të pata kutoimama. 

Të wai ma ihirupɨwei rë, kiharë të wai ma hure. 

Kuɨ tëhë, pë xɨɨ kɨpɨ noã tayou tëhë, a he tatoprarioma. Pei yo ka ha e
haowapɨ huheprakema. E kiriri huheprakema. 

--- Xoape! Xoape! Kɨ ã ta mapɨiku! Yai të ã no pata kiriaimi, wahë kɨ si
rĩya ikekaimi, të ã pata ahetea waikire, xoape, kɨ ã ta mapɨpo! --- e
kupɨma. E kɨ ã mapɨprapema kiriri, mapɨprao tëhë, e ã pata pëtou
nokarayoma. Kɨ mata waipɨprao tëhë, të ã pata nokakeyoruma. 

--- Siiiikekea, \textit{kea, kea!} --- e të pata kupɨtarioma. 

--- \textit{Hɨ̃ɨ!} --- e kɨ kupɨma, e kɨ kiriri hokëkëpɨprou ha xoaronɨ, ɨ̃hɨ weti
naha e kɨ kupɨprarioma?

--- Pei kë! Pei kë! Yai tënɨ pëhë kɨ si rĩya rë ikekaimi ë! Pusi pei
këë! Pei këëë! Pei a ta haɨpraru xë! Pata e kuma, ya e wãha rë
yuprarɨhe.

--- Haɨmo pusi, xei, hapoxë pei kë, haɨmoxë, hekamapɨ e mɨ wai
tarɨapraroma. 

--- \textit{Keora kɨrɨ! Keora kɨrɨ! Keora kɨrɨ!} --- e kɨpɨ kuɨ xoaoma. 

Ɨhɨ rë e kure, porakapɨ kɨ rërëpɨa xoarayo hërɨma, pei u heaka hamɨ.
Kihamɨ kɨ kepɨ kiriopë. Kɨ kupɨa hërɨɨ tëhë, hekamapɨ ya e wãha rë
takɨhe, ɨ̃hɨ kɨ kupɨa hërɨpë hamɨ, e texina ĩyë. \textit{Prohu!} 

Ɨhamɨ e pë nosi yaurayo hërɨma. Ɨhɨ rë pë yëɨ. Ɨhɨ exi të pë Yoasiwë?
Mau u pë hetu hamɨ, të pë texina rë ĩyëi? Ɨhɨ e kuprario hërɨma. Ɨhɨ xɨɨ
e ku hërɨpë hamɨ, a no uhutipɨ rë hëaanowei, mau u pë hetu hamɨ pë yëɨ.
Ɨnaha pë kuprarioma. Pei a yai rë kui, a kãi kepɨye kirioma. No uhutipɨ
a hëpɨmake hërɨma. Ɨhɨ rë a pararupɨre hërɨma. Ɨnaha të kuprarioma. 

Ɨhɨ kɨpɨ rë kepɨora kiri, kɨpɨ no uhutipɨ rë përɨpëye kirionowei hamɨ,
pei u koro titia. Ɨhamɨ kɨ përɨpɨa. Kɨ rë nomapɨno rë mai, ɨ̃hɨ kɨpɨ
xoaa. Ai kɨ ha wãritipɨprarunɨ, kɨ nomapɨnomi. Kɨ ha xawarapɨprarunɨ, kɨ
kãi nomapɨ taimi. 

Ɨhɨ kɨpɨnɨ yãɨmou, të rë hirapɨnowei, ɨ̃naha të tapɨma yaro, Yanomamɨ
pëma kɨ reahumou. Pëma të pou hëa. Ai të pënɨ reahumou tararenowehei, të
rë hiranowehei, kuami. Ɨnaha të kua.
