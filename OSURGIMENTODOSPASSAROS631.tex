\chapterspecial{O {surgimento} {dos} {pássaros}}{}{}
 

\letra{O}{grande tuxaua} que dividiu a terra onde nós moramos era um verdadeiro
líder, e morava além da parte central da terra dos
Këpropë.\footnote{  Trata"-se de outro nome dos próprios Parahiteri, narradores da história.}  Esse nome, um nome importante, também é o nome
da região onde ele nos ensinou a morar; é o nome da região onde eles
moravam antigamente.

Nessa terra, ele fazia o \emph{kawaamou} e transmitiu esse ritual; ele os
mandava se reunirem e nos ensinou, assim, a nos
reunirmos em nosso xapono.\footnote{   Os xaponos são as casas coletivas circulares onde moram os Yanomami. Cada casa corresponde a uma comunidade; em geral não se fazem duas casas numa mesma localidade.} 

Como se chamava aquele que dividiu a terra, quando, no início, os
Yanomami ainda não praticavam esse ritual; que, em seguida, se tornou
líder e deu a terra? Ele tem nome. Aquele que mostrou
o \emph{kawaamou} se chamava Gavião.

Ele existiu desde o início, esse verdadeiro líder que, depois, chegou
para morar com os poucos líderes que sobreviveram ao dilúvio. Gavião
morava em um xapono e o cunhado dele era o líder do grupo dos
Wãhaawëteri em outro xapono vizinho. Wãhaawë era o nome
dele. Os dois sobreviveram.

A região que ocupou se chamava Wãha, por isso o nome dele era
Wãhaawë. Foram os ancestrais mesmo. Foi ele que dividiu e deu a terra
para nossos antepassados morarem. Aquele que deu a terra aos nossos
antepassados se chamava Wãhaawë. Os moços iguais aos nossos, que moravam
com ele, se chamavam também Wãhaawë. 

Moravam os dois, Gavião e Irara. Chamava"-se Irara aquele que ensinou os
Yanomami a atirar com zarabatana.

Nessa terra, quando não havia Yanomami, moravam somente três pessoas.
Era o xapono de Gavião, que bebia a água do rio Porena. O xapono dele se
localizava na beira do rio Porena e bebia sua água. É o nome do rio onde
os dois moravam. Tomavam banho nessa água. A região de Porena era a região
deles. Moravam lá, perto daquele rio. 

Nós, Yanomami, quase que não aprendemos a soprar veneno nos outros. Já
mencionei esses nomes, Irara e Gavião, nomes terríveis que vocês já
ouviram. 

Agora é a história daquele que experimentou soprar. Foram esses dois que
ensinaram àquele feio; Irara já conhecia o veneno, pois ele o possuía. 

Depois de ter observado Irara soprar veneno, o feio aprendeu e logo
soprou em Mel. Experimentou nele; aquele feio experimentou em Mel pela
primeira vez, quando o ato de matar não existia e quando ninguém
morria. 

Aquele antepassado chamado Irara recebeu o nome de Nokahorateri, o que
entende de veneno; o grupo dele também passou a se chamar assim. 

O feio quis atirar logo em Mel, pois pediu veneno a Irara. Ele o pegou,
quebrou um pedaço de semente para experimentá"-lo imediatamente. Esta é a
história do feio que soprou pela primeira vez o veneno. Não são os
nossos antepassados, são outros! 

%Aqui é preciso um intervalo entre parágrafos. 

Havia um líder muito bonito chamado Mel. Era nobre e muito cheiroso:
onde ele passava, deixava seu perfume, pois o cheiro dele era como o dos
cabelos das mulheres que os Yanomami acham tão bonitas que desejam
ir aos xaponos delas para se casarem.

Duas mulheres chegaram e viram Mel roçando, as duas ficaram sentadas. Ele
mostrou como roçar, ele abriu a roça. Saíram do xapono delas e andaram uma
atrás da outra, até o xapono dele, só para vê"-lo. 

--- Olha só, como esse homem é bonito! 

As duas pensavam isso por causa dos seus cabelos bem macios e
cheirosos. Hɨ̃a!, fizeram as duas cheirando. Toda a floresta
estava cheirosa, o perfume dele se espalhava. 

No mesmo xapono, morava Mucura, o feio, que não cheirava tão bem.\footnote{  Mucura é um nome amazônico para o gambá.} Ao avistar as duas mulheres, cheias de desejo e se
dirigindo apressadas à casa de Mel, Mucura se zangou. Ele era mesquinho
e, assim, ensinará os feios a matar os bonitos. Indiretamente, as
duas mulheres entregaram Mel à morte. 

Quando apareceram as duas mulheres, passando na frente da casa do feio,
ele estava deitado no chão. Xiri! Kuxuha! Krɨhɨ! Mucura bateu o chão com o pé para chamar a atenção das duas e, quando elas olharam, o feio fingiu cair no chão para elas cuidarem dele.

Apesar de ser fedorento, de estar sempre com olhos purgando e cheio de
feridas, quando ele viu as duas mulheres se abraçando, ele bestamente
pensou:

--- Elas chegaram para mim, minhas duas mulheres apareceram, gosto
delas. 

Ao perceber que o olhar das duas mulheres se dirigia à rede bonita de
Mel, ele se zangou mais ainda. 

--- Estou aqui! --- disse ele, mas sem efeito, pois elas não vieram por
sua causa.

A rede estava amarrada lá do outro lado, em um espaço livre, e as
franjas de sua rede vermelha, cor de sangue, se balançavam levemente, a
rede estava logo ali. 

--- É essa daí, lá mesmo, lá está a rede dele! --- disse uma das duas
mulheres, em pé de frente para a casa de Mucura. 

Enquanto isso, Mucura fingia fazer o \emph{wayamou}, apesar de não saber
fazê"-lo:\footnote{  O \emph{wayamou} é um diálogo cerimonial realizado à noite por um hóspede e um anfitrião por ocasião de uma visita, destinado reforçar ou restabelecer relações pacíficas entre dois xaponos. Mucura não está fazendo o \emph{wayamou} do modo correto.} 

--- Aë, aë, aë! Minha mãe, a comida que eu guardo no moqueador,
pegue a carne e dê às minhas duas esposas que chegaram, dê a elas
agora. 

Apesar de não serem esposas dele, ele falou assim. 

Ele deu um pedaço de carne da bunda dele às duas mulheres, que olharam
de soslaio e fugiram, por causa do fedor. 

Kuxu!, elas fizeram, cuspindo, kuxu!

O fedor afastou as duas, que foram em direção da rede de Mel, que estava um pouco mais adiante. Elas foram lá, deitar"-se juntas na rede
dele. Ao chegar em casa, Mel pensou, hɨ̃ɨ!, e perguntou:

--- O que vocês duas estão fazendo? Não havia ninguém como vocês antes,
como vocês duas chegaram?

--- Viemos por sua causa! Venha! Venha! 

Agarraram"-no e o deitaram em cima delas. Deitaram uma de cada lado para
cheirá"-lo. O feio, que fazia o \emph{wayamou}, viu isso e
chorou. 

Ele se abaixou e as moscas seguiam seu fedor. Ele estava cheio de
ura,\footnote{  Isto é, bernes.} as pálpebras estavam cheias de ovos de mosca.
Quase cego pela ira, ele rapidamente saiu, pois aqueles que guardavam o
veneno, Irara e Gavião, moravam ali perto. Ele pediu: 

--- Me dêem o veneno! Vou experimentar! Quero magoar as duas mulheres
que chegaram e me fizeram sofrer. É verdade o que estou dizendo! Vou usar agora! Dêem um pedaço! --- disse ele. 

Eles logo responderam:

--- Então pegue o meu! Experimente com este! Experimente! Tente! 

--- Mas com quem? 

--- Não digam que não sabem! Vocês já conhecem aquele feio, Mel! Sabem
como ele não presta! É o nome daquele que não presta! 

Apesar de ser menos que ordinário, de ser careca, apesar de ter um rosto
feíssimo, ele disse isso! O nariz dele era horrível, e mesmo assim ele
disse isso. O olhar dele era horroroso, e mesmo assim, ele disse isso. 

--- Verdade, mesmo? 

--- Sim! Verdade! Vou experimentar hoje!

--- Se você diz isso, então tome! 

Quando voltou, ele rapidamente se preparou. 

Querendo impedi"-lo, a mãe de Mucura falou: 

--- Querido! O que você está querendo fazer? Deixe para lá! Deixe para
lá! --- disse ela. --- Cuidado, querido! O que você está querendo? O que
você está pretendendo fazer? Cuidado! Não vai, não, não vai! Fique
quieto! --- disse ela. --- Não vá lá de novo, com Irara e Gavião! As
mulheres são de Mel! São mulheres dele! Não pense que elas vieram para você!
--- disse a mãe. --- Elas fecharam o nariz por causa do seu fedor!
--- disse.

Pois seu fedor chegava longe.

--- Não me pergunte o porquê! Eu vou! 

Mel havia voltado para derrubar as árvores grandes. Mucura já queria matar
Mel. 

Mucura correu e logo foi, enquanto zoava a queda das grandes
árvores. Kou! Kou! Kou! Kou! 

--- Ãaaaaaõoooo! Aë, aë, aë, aëëëëëëëëëë! --- estrondava o eco da voz bonita de Mel. 

Para experimentar a força do veneno, Mucura o experimentou primeiro em
Lagartixa. As lagartixas, que ficam grudadas aos jutaís, aquelas que
sempre sobem. Mucura foi em direção de Lagartixa, enquanto Mel
derrubava. 

Onde se erguia um jutaí, Lagartixa subia, fazendo tararararara! 

--- Kuxu! Kuxu! Kuxu! Vou soprar naquele mesmo! Eu vou experimentar a força do veneno! --- disse Mucura. 

Primeiro ele soprou Lagartixa. Soprou. Pegou de raspão na garganta de
Lagartixa. Paha! E uma última vez, paha! Paiii!

Com isso, ele o feriu, e onde arrancou um pedaço da garganta, deixou
vermelhas as gargantas das lagartixas. Mucura ficou olhando, para ver
Lagartixa passar. 

O veneno não agiu. O veneno era fraco. Somente as folhas de jutaí,
tontas de veneno, caíram. Lagartixa, que Mucura havia soprado, subiu
mais acima. Mesmo tonto, ele não caiu. Ele se recuperou. 

--- Puxa! Não faz isso comigo! 

Mucura não gostou de ele ter resistido ao veneno. Lagartixa sumiu, já
estava em outro lugar. Mucura, depois de soprar o veneno que tocou
levemente a garganta de Lagartixa, foi em direção ao bonito Mel, que
trabalhava e não aguentaria a força do veneno: o fôlego dele não seria
tão forte. 

Mel estava virado em cima do andaime. Paha! Paha! Ele se
desequilibrou. Aquele que foi soprado antes pegou um atalho e foi falar
a Mel: 

--- Tome cuidado! Ele me soprou! Aquele feio fedorento quis primeiro
experimentar a força do veneno comigo! Cuidado, fique atento! Ele vem em
sua direção! --- disse ele. 

--- Hɨ̃ɨɨ! A força do veneno não me destruirá! O feio não me
pegará! --- Mel disse. 

O feio já estava perto.

--- Ele está me desafiando! --- disse Mucura. 

Mel virou, enquanto derrubava com o machado. Paha! Paha! Paha!

Mucura simplesmente soprou; você não aguentaria muito tempo a força
desse veneno, não dá para aguentar. 

Mucura foi logo embora e Mel caiu, tonto. Ele caiu. Corre, corre! Aquele
que tinha caído voltou para suas duas mulheres. Quando ele chegou, não
demorou…

 --- Estou com muito frio! Estou morrendo de frio! --- chegou em casa,
delirando, tonto. --- Faça fogo para mim! --- disse ele. 

Ele já estava morrendo. Enquanto as duas cuidavam do marido,
esfregando"-o, ele caiu morto. 

Ensinando o choro aos Yanomami, elas duas logo choraram, choraram,
pensando nele. As duas se abraçavam, chorando, se segurando pela mão,
chorando. Onde estava o marido morto, as duas mulheres avançavam e
recuavam.\footnote{  Trata"-se de um movimento da dança feita em ocasiões fúnebres.}  Aí o feio se juntou, aquele que havia
soprado fingia chorar, de medo fingia chorar, e disse assim: 

--- Meu grande e querido sobrinho! Meu sobrinho, mesmo! Mataram meu
sobrinho! Mataram meu irmão! Puxa vida! Reapareça para elas! ---
fingia. 

Enquanto falava assim, as duas mulheres fugiram, não queriam escutar.
Ele as seguiu. Enquanto ambas as mulheres rodavam ao redor do xapono,
chorando, ele falava, fingindo chorar; rodeava ao redor do
xapono, dissimulando, e as duas mulheres choravam, fugiam dele de novo,
pois estavam bem zangadas. Apesar de ele chorar, as lágrimas não saíam,
ele simplesmente fingia. 

Ele fugiu. As duas mulheres cremaram o corpo de Mel e, enquanto o
cremavam, o feio fugiu para se esconder, como se fosse escapar: fugiu
pensando poder se esconder. 

Depois da fuga dele, e ensinando os Yanomami como se mata, como se segue os
rastros, eles seguiram logo os rastros. Procuraram os rastros.
Procuraram os rastros. Procuraram. Fizeram isso logo, e não perderam os
rastros. 

--- Foi o feio que o matou! --- diziam. --- Foi mesmo o feio, o feio
fugiu! Agora é a vez dele! --- esbravejavam. --- Nunca sobreviverá,
criminoso! Vamos dar o troco! Acabou com nosso líder, então vamos
matá"-lo! --- disseram, e seguiram os rastros. 

As formigas Tokonari, as formigas Xĩriana, as
formigas Mamisipreima e as saúvas se mexeram, seguindo os
rastros. Os passarinhos também seguiram os rastros. 

Eles chamaram {Resimaritawë}, que seguiu os rastros como cachorro,
pelo cheiro; eles passaram levando Resimaritawë.\footnote{  No dicionário de Lizot consta tratar-se do ancestral do minhocuçu, um anelídeo grande, \emph{heresima}. Como o termo não coincide exatamente, mantemos o nome original.} Não conseguiam encontrar os rastros onde havia pedras, onde havia a
pedra \emph{maharixitoma}, porque os rastros já tinham sumido. 

Naquele lugar, onde Mucura havia matado Mel, os seus rastros estavam
dando voltas, sumindo no meio de um pântano. Ele fugiu e estava a uma
distância igual àquela que nos separa do rio Maupuuwei, aonde
vamos {caçar em grupo}. 

Mucura se escondia na montanha, ele foi lá em cima, porque queria
escapar. Ele subiu em uma árvore. A montanha era redonda como um jutaí,
ele entrou lá, onde a montanha tinha uma fenda. Pretendia se trancar
ali. Eles derrubariam a montanha para pegá"-lo. Mucura não entrou no
jutaí. Ele entrou nessa montanha, porque queria fugir. 

Como criminoso, ele se arranhava, mas não com as
unhas.\footnote{  A pessoa que assassina alguém costuma se arranhar com as unhas,
castigando"-se. Mucura é mau, então não chega a se machucar.}  Adormeceu. Criminoso. Ele se admirava, porque
matou. Ele ensinou a ser criminoso. Ele perpetuou o crime. Ele estava
dormindo. 

Resimaritawë seguiu o fio de Mucura, e ficou escutando. 

--- Aqui! Ele está dormindo --- pensou.

Como criminoso, Mucura guardava um tipo de canudo no braço, guardava um
em cima da orelha. Õooo, õooo, õooo! Ouvia"-se a respiração de
Mucura e Resimaritawë escutava:

--- Õooo, õooo, õooo! Mel! Mel! Você viu como é bom? Foi isso que
fiz para você! --- roncava Mucura. 

Apesar de estar roncando, ele se gabava. 

--- Está vendo? Foi o que fiz pra você! Mel, foi o que fiz para você!
--- dizia, roncando.

Ouvindo isso, Resimaritawë logo se assustou e gritou: 

--- Hɨ̃aaaaaaaaaëë, aaaaaaëë, o Mucura fedorento, aqui, o feio em
pessoa está se gabando bem na minha frente, aaaaaë! --- disse ele. 

Esse abrigo estava na base da montanha, que vocês não conseguiriam
derrubar rapidamente, mas eles trabalharam como loucos e causaram um
impacto incrível no cume da montanha. Não tinham terçados, mas mesmo
assim conseguiram matar Mucura. Apesar de não possuírem terçados como os
dos \emph{napë,} eles conseguiram matá"-lo. 

Chamaram os do grupo dos tucanos Parawari, porque o grupo das
Maitacas não conseguiam. Os Araris estavam tendo dificuldades com seus
machados de pedra, que se destruíam. Apesar dos machadinhos
dos Tokorari, todos sofriam por causa das ferramentas, que se
quebravam em pedaços e não entravam na pedra. Chamaram os do grupo
Parawari, que viviam agrupados na baixada. 

Devido ao que fez o feio ao seu sogro, o bonito Mel, seu genro Parawari
disse:

--- Vamos! Vão chamar meus pais, que moram com meus avós: eles moram bem
perto e têm verdadeiros facões! 

--- Vai, querido, corra! Vai você! Vai buscar! Vai buscar! --- diziam
assim. 

Eles os buscaram. Eles chegaram e atacaram a montanha. Não queriam
destruir o facão deles. Como outros facões haviam sido destruídos, os
pedaços das lâminas estavam espalhados no chão. Para poupar esforços
inúteis, eles amoleceram a parte interna da montanha, como se fosse uma
árvore, com a força do pensamento. 

Kraxi! Kraxi! Kraxi! Depois de amolecer a pedra, derrubaram uma
parte. Kraxi, kraxi, kraxi! Atacaram logo de todos os lados para
fazerem voar lascas de pedra. A montanha era do tamanho de uma sumaúma.
Fizeram outro buraco grande, para poderem continuar com a destruição da
pedra. O buraco no tronco se aprofundava; fizeram uma espécie de cratera
na pedra. 

--- Vejam essa montanha! --- disseram os Parawari. 

Conseguiram fazer esse buraco porque os Parawari têm esse bico mais
comprido, e que será mais comprido para sempre. Os terçados, sendo
mais compridos, começaram a derrubar a montanha. O bico do tucano
empoleirado é, na verdade, seu terçado. Ele não nos morde, quando olhamos
para ele? Kraxi, kraxi, kraxi! Foi nesse momento que começaram a
fazer a montanha balançar. Não demorou. Kru tu tu tu tu tu!
Pedacinhos de pedra voavam e pulavam; os pedaços se espalharam no
chão. Huãaaaaaa! Eram muitos, e estavam tristes por estragar seus
terçados, pois o bico deles é curto. Era isso mesmo: o trabalho começava
a ser realizado. Kru tu tu tu tu! Continuavam trabalhando.

Só faltava o coração da montanha. Os terçados sendo curtos, esse
pedacinho ainda resistia. Apesar de pequeno, o tronco da montanha não
quebrou rapidamente.

--- Pei kë! Aaaaaaaaooooo! Vamos! 

Apesar de o tronco estar quase torado, a montanha não estava se
mexendo. 

--- Aaaaaëëëë! Vai você! A ponta da montanha vai nessa direção
--- disseram a Preguiça. 

Queriam que ele puxasse a ponta da montanha, pois a montanha não caía.
Preguiça esticou um fio flexível e puxava a montanha. Os terçados dos
Parawari, cortando a pedra, não atingiam o núcleo. 

Preguiça esticou uma espécie de fio. Ele nem pensou: --- Essa montanha vai
me machucar! --- Ele não tem costume de ficar com medo. Com o fio,
parecido com linha de pesca, ele puxou o cume da montanha. Não havia
nada para conter a pedra. Preguiça estava sozinho, sem apoio, mas também
sem medo. Ku tu tu tu tu tuuuuuuu! A pedra começou a estourar,
fazendo um barulho enorme; parecia cair um pedaço grande de céu. 

Mucura estava preocupado e chorava, olhando para baixo com lágrimas
escorrendo. Ele estava desesperado. 

Tuuuuuuuuuuuu, tẽẽẽrërërërë! Preguiça caiu mais à frente, para
não acabar embaixo da pedra, que caía. Ãaaaaaa! O fio o
impulsionou e ele foi cair bem longe, ele se engatou lá longe com suas
garras. A pedra, quebrando e levando árvores, não alcançou a árvore onde
Preguiça se agarrou; se ficasse pendurado mais perto, ele se
machucaria. 

Eles destruíram Mucura. 

Todos os animais, as araras, os tucanos, os urus, os inambus, os mutuns,
os jacus, os urumutuns, os mutuns"-de"-traseiro"-vermelho e os jacamins
eram gente. 

Todos juntos, pegaram aquele que foi destruído. 

Chegaram até o sangue de Mucura derramado no chão para se pintarem. As
pequenas maitacas amarelas se pintaram com seu sangue cinzento. Todos os
passarinhos são diferentes: uns são vermelhos, outros cinzentos, outros
têm pálpebras cinzentas. As cores dos pássaros vêm daquele momento,
quando se transformaram em animais, naquele mesmo lugar, com o sangue
derramado. Transformaram"-se onde havia o sangue derramado. 

Aqueles que foram destruir Mucura não voltaram para seus xaponos, onde
havia roças. Não voltaram para recuperá"-las. De tanto se pintarem com o
sangue derramado, ele acabou e, depois de terminarem, logo voaram. Logo
se transformaram. Eles ocuparam a floresta toda, não restou nenhum
espaço. 

Outros pintaram de preto o seu peito, passavam um pouco de sangue na
garganta, outros pintaram as pálpebras, outros, os cabelos, outros
pintaram o cume da cabeça; outros se pintaram de cor cinzenta,
derramaram os miolos e os excrementos com os quais se pintaram. Assim
fizeram. 

Preguiça, que puxou a pedra, passou no seu corpo os excrementos, por
isso ele tem cor cinzenta, é a cor dos excrementos de Mucura. Ele também
passou sangue na bunda, levemente. 

Assim, depois de esgotarem o sangue de Mucura, eles logo voaram, e todos
sumiram. 

Termina assim essa história, porque eles voaram, se transformaram e
partiram. Onde aconteceu essa história, outra segue. Contamos o que
aconteceu a Preguiça e Mucura, cujo esconderijo foi derrubado.

A região onde eles moravam tem um nome. Aquele homem bonito, que foi
morto, morava à frente da serra Moyenapɨwei. Ele nasceu, morava à frente
da serra Moyenapɨwei.  Essa serra se chamava Moyena.
Ao pé dessa serra, Mel abriu roças. Por isso, se chamaram assim, pois
ocupavam a região desse nome. 

Apesar de ser uma serra, chamava"-se assim. Tinha esse nome, pois era uma
região bonita; quando as palmeiras \emph{moyena} floresciam, as exalações
tomavam toda a floresta. Ainda existe o perfume onde ficava sua moradia,
e seus descendentes ainda moram lá. Chamam"-se Moyenapɨweiteri. Ficaram
morando lá, pois os antepassados se chamavam assim. 

Nessa região central, morava também Jacaré. Jacaré morava nessa região
central, onde Mucura matou Mel.
% Este parágrafo puxa a história de Jacaré (A proliferação do fogo).
