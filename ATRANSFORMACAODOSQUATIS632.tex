\chapterspecial{A {transformação} {dos} {quatis}}{}{}
 

\letra{O}{s animais} moravam em xapono; os quatis, as cutias e as antas, as
queixadas, os cuatás, os beija"-flores, os passarinhos moravam em grupos
como nós.

Naquela época, a mesma transformação ocorreu com todos esses animais,
exceto o caititu. Ele não andava como nós; não morava como nós, ele
sempre andou como ele ainda anda hoje. É um animal, e sempre foi. Sempre
andou como animal, assim como os cuxiús, os inambus, as cutias
vermelhas, os veados roxos.

Havia cinco espécies de animais. 

Já existiam todos os animais que há hoje na floresta? Não, somente
esses. Os jabutis não existiam, não andavam, não existiam como animais e
nem como gente. Nem os tatus-galinha. 

Vocês comem o quati, apesar de ele ser Yanomami. Os Yanomami se
transformaram em quatis. Tornaram"-se animais no tempo de Horonamɨ. 

Como eram Waika, eles se transformaram. Eram os ancestrais dos Waika.\footnote{O par \emph{waika}/\emph{xamatari} parece ter sido usado originalmente para designar outros grupos yanomami vivendo em região geográfica diversa de quem fala, os primeiros ao norte e oeste, e os segundos ao sul, reconhecendo-se neles conjuntos de características que os particularizam. Os termos foram atribuídos em diferentes momentos pelos brancos para designar grupos específicos de forma estável e, no caso de \emph{xamatari}, para designar a própria língua do tronco yanomami usada pelos Parahiteri que fizeram este livro.} Quando saíram de \emph{wayumɨ}, todos se transformaram. \footnote{   \emph{Wayumɨ} são longas estadias coletivas na floresta. Em geral são motivadas pela falta de comida no xapono. A comunidade pode se dividir em vários grupos quando se trata de um xapono populoso, e se desloca num vasto círculo, fazendo acampamentos sucessivos.} 

--- Querido! Meu nariz se rasgou! 

--- Meu nariz se rasgou assim também!

--- Õãaa! Xiri! Meu nariz arrebitou! 

Foi assim com todas as crianças. 

--- Õãaa! Xiri! Avô! Meu nariz também arrebitou! 

--- Õãa! Xiri! --- disseram todas as crianças ao avô delas. 

Transformaram"-se enquanto estavam de \emph{wayumɨ}. Os ancestrais Waika
se metamorfosearam. São os primeiros moradores; eles se transformaram.
Ocuparam toda a floresta. Não sobrou nenhum xapono em torno do qual não
vivam quatis. 

Esse rio grande, rio abaixo, do qual vocês comem muitos peixes, na sua
parte média, apesar de ser rio abaixo, dá para avistar a pelagem muito
vermelha dos quatis. Eles andam por lá. 

--- Fĩfĩfĩ! --- eles dizem. 

Os quatis são Waika. Foram os antepassados dos Waika, que se
transformaram indo de \emph{wayumɨ}. O nariz quebrou e se arrebitou.
Transformaram"-se no meio da floresta. Nunca mais voltaram a morar em
xapono. Eles se transformaram. A imagem deles se alastrou por toda a
floresta, como também a imagem dos jabutis.

% Aqui deveria vir a história do jabuti, que acabo de ver que não está em nenhum livro.