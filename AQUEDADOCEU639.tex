\chapterspecial{A {queda} {do} {céu}}{}{}
 

\letra{A}{ história do céu:} há a história da derrubada do céu. No início, ele
estava lá em cima. Este solo é o céu caído, portanto, os primeiros
habitantes foram esmagados. Esta terra os esmagou. Tornaram"-se, então,
os Amahiri, que moram em grupos como nós, mas moram lá em baixo. 

Aquele céu caiu. Os nossos antepassados saíram bem no meio; se tivessem
sido lerdos como os que foram amassados, não estaríamos aqui nesta
floresta. 

Depois daquele primeiro céu eterno, este segundo sobreviveu. Este novo
céu sobreviveu. Aquele primeiro, que estava em cima, envelheceu e caiu. 

Depois de sua queda, nós surgimos nesta terra mesmo, pois nossos
antepassados se multiplicaram nesta floresta. Apenas os habitantes de um
xapono sobreviveram para que pudéssemos existir, mas quase que eles não
existiram. Em cima desta terra, nossos antepassados se reproduziram e
nasceram, e depois deles, os seus descendentes. 

O céu não exterminou todos os nossos antepassados, ele esmagou apenas alguns. Os
Amahiri se agruparam lá em baixo da terra. Os que foram esmagados se
chamam Amahiri. 

Apesar de estarmos nas alturas, nós existimos. Assim, se essas montanhas
não houvessem existido e se não se erguessem, nossos antepassados não
haveriam saído, não haveriam existido. 

Eles ficaram espertos por causa dessas cavernas.

Quando começou a estrondar lá em cima, quando o evento se
aproximava, ele avisou seu povo. Os outros não sabiam:

--- Vamos, meus jovens, nós da região central vamos escapar por essas
montanhas, pois eu não sou tolo! Vocês não serão esmagados. Nós apenas
sobreviveremos nesta região, e se for preciso, passaremos para o outro lado
do céu. Limpem essa montanha! --- disse--- Abram o caminho! Limpem! 

Dito isso, eles limparam ao redor da montanha para se abrigar. Quando o
céu ia cair, quando esse momento se aproximava, a montanha já estava
limpa. Na hora de cair, o céu arrebentou, porque ele estava velho.
Enquanto se arrebentava, os que escaparam entraram na caverna. 

--- Vamos, enquanto o céu ainda está alto! Venham! Depois de amanhã,
depois de amanhã, o céu vai descer até o chão! 

Depois de ele dizer isso, o céu caiu. Esmagou os Esmagados. Moravam
nesse lugar. Quando o céu caiu, esmagou os que ficaram, e os que estavam
na caverna não foram esmagados. O céu ficou por cima da caverna. Assim,
passaram a se chamar Derrubadores de Céu; era o nome deles. Queriam
derrubá"-lo, por isso se chamaram assim, com o mesmo nome. Apesar de o
céu quase os amassar, eles escaparam. 

Enquanto caía, o filho mais novo e o cunhado pularam e, assim, se
prepararam. O pai mandou que enfrentassem, mandou arrebentarem o céu. Fez
que o arrebentassem. Apesar de o céu parecer indestrutível, mesmo assim
os dois o arrebentaram. 

Aquele que arrebentou o céu, aquele que tinha esse nome, arrebentou
mesmo: ele se chamava Hutukarariwë. Atacou logo, sofrendo por causa do
sangue, se cortando com os pedaços do céu, cortado perto dos
olhos. Kreti! Kreti! Kreti! Fazia assim. 

Ele penou. Chamava"-se assim, Hutukarariwë. Escaparam por onde ele
arrebentou o céu; por essa abertura, só o grupo dele escapou. Os que
sobraram escaparam e saíram. Ninguém mais saiu de outro lugar. Nossos
ancestrais se reproduziram a partir daqueles que conseguiram escapar.
Imediatamente continuaram a se reproduzir. Assim aconteceu. 

--- Enquanto o céu está vivo ainda --- disse o pai aos dois filhos---
vocês vão juntos! Vão embravecer! Não cortem em silêncio! 

A partir do momento em que nossos antepassados se reproduziram, surgiram
também os Waika. Eles se dividiram.

Antigamente havia também outros grupos. Outro nome importante dessa
época é o espírito Hemarewë. Todos esses nomes são nomes de espíritos.
Hemarewë também vivia nessa época como pajé, ele foi um dos primeiros
habitantes da região.

Mas não é o nome dos nossos antepassados. A história do nosso grupo
Parahiteri se encaixa no meio da história dos Yanomami. Alguns grupos
foram se extinguindo e de gerações posteriores foi que apareceram nossos
antepassados. 

Nossos antepassados surgiram na região central. Nós ficamos nessa
região central, onde surgiu a primeira mulher,\footnote{  História contada no volume \emph{Os Comedores de Terra, nesta mesma série.} pois nossos antepassados
moravam lá. Ela nasceu nessa região central, ficava lá. Esses moradores
foram chamados de habitantes da região central. Foi assim que nós
surgimos.