\chapterspecial{A \versal{onça} e a \versal{centopéia}}{}{}
 

\letra{N}{essa época} as onças não comiam gente, não andavam, não existiam. Não
havia onça na floresta. Segue essa história. Não andava onça para nos
matar e nos comer. Hu, Hu! Hu! A onça não dizia isso. 

Quem encontrou a primeira onça? Sozinha, ela sofria de fome, sequinha,
sua barriga gritava de fome, pois ela não tinha dente. Onça tinha apenas
gengivas, ela não mastigava, ela andava magra no meio dessa região do
Xererei, ela andava sozinha, andarilha de fome. Como ela comia quase
nada, ela chorava. Ela chorava por fome de carne. 

Quem a encontrou? Onça chegou aonde estava Centopéia, onde morava
sozinho um Yanomami, Onça chegou à casa de Centopéia. Ela apareceu, elas
se encontraram, ela ia de encontro. Com fome, andava como se fosse cega,
sem olhos, sofria mesmo, fazia muito barulho, tropeçava de fome. 

É uma centopéia! Vocês conhecem esse nome? Era Yanomami, aquele que anda
sem fazer barulho. Krihi! Ninguém diz isso, andando em cima de um pau.
Foi ela quem ensinou primeiro. 

Ela emprestou seus pés para Onça não mais fazer barulho; ela ensinou a
andar discretamente. Depois do ensinamento de Centopéia, Onça andou, ela
foi lá, chegou à terra plana e desceu.

 
