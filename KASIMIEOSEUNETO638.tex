\chapterspecial{{Kasimi} e o {seu} {neto}}{}{}
 

\letra{K}{asimi e seu neto} foram até os espíritos para não viverem como viviam
os {Yanomami}. Os espíritos viviam escondidos no mato. Todos foram
exterminados pelo Sangue da Lua.

Ou seja, a avó e o neto conseguiram alcançar os espíritos. Ela está lá ainda. É assim, não morreu. Aonde vão os espíritos, ela também vai, ela os
alcançou. Ela chegou ao lugar dos espíritos para se tornar eterna. 

Kasimi alcançou o xapono dos primeiros espíritos, os Ihiruwëteri, nome
de espíritos. O primeiro xapono dos espíritos era esse, o das
Crianças, Kasimi chegou a eles, ela chegou aos
Oxemawëteri, os Jovens, que é outro nome deles. 

Kasimi alcançou a moradia dos Parawari Yokënamari, todos solteiros, e que
estavam dançando. Ela chegou durante uma festa, e, assim, nos ensinou a
fazer festa. Kasimi era o nome da mãe daquele que flechou a Lua.

Os antepassados, que moravam espalhados, não resistiram aos Yanomami
oriundos do Sangue da Lua, de onde nós viemos. Isso se fez para nós
brigarmos, para nós guerrearmos. 

Quando começaram esses eventos, os antepassados logo ficaram espertos,
que antes não eram. Quando brigavam, era como uma dança, e
simplesmente não paravam de rir. Quando havia guerra, não sabiam reagir
e {só} faziam pajelança. Antes, eles não se vingavam.
Durante a guerra, eles festejavam; era isso que eles faziam! 

No início, havia os que ensinaram os Yanomami a morar. Os espíritos
existiam e eram parecidos com os Yanomami. Os espíritos não foram obra
de ninguém. Eram assim, como os Yanomami. O xapono deles era tão limpo
como o meu, não era fechado. Moravam juntos. 

Dizem que moravam assim sem ninguém os ter ensinado. Eles viviam em um xapono
igual ao meu. Eles andavam sempre no limpo. Eles faziam amizade,
conversavam e se visitavam. Se não agissem assim, teriam nos exterminado
há muito tempo, pois eles nos comem. Se eles morassem ainda no limpo,
todos vocês, rapazes, cantariam: ea, ea, ea! Todos vocês seriam
pajés. 

Hoje, os espíritos não são mais visíveis, pois não moram mais no limpo.
Eles dançavam, faziam festas no limpo, como os Yanomami. Eles dançavam
como dançam os Yanomami, no limpo. 

Faziam os rituais de \emph{himou} e de \emph{wayamou}, cantavam como os
pajés cantam.\footnote{  O \emph{himou} é uma modalidade de diálogo cerimonial usada para trazer notícias, ou fazer um convite para uma festa. O \emph{wayamou} é um diálogo cerimonial realizado à noite por um hóspede e um anfitrião por ocasião de uma visita, destinado reforçar ou restabelecer relações pacíficas entre dois xaponos.}  Eles cantavam assim. Eles também brincavam de roubar
esposas, como fazemos durante as festas. 

Moravam na planície, em terra plana, não moravam naquele tipo de
montanha. Depois, eles foram morar lá nas montanhas, foram logo assim. 

Como eram todos gente que morava no limpo, a imagem da minha avó, apesar
de ser espírito, ainda alcança os pajés, porque ela era Yanomami. A mãe
daquele que flechou o monstro chegou onde moravam os matadores de
monstros, que moravam no limpo. Ela deve ter chegado enquanto eles ainda
eram visíveis. 

Aquela que fechou a casa dos espíritos se chamava Kasimi, mesmo. O segundo nome dela era Maxikomi. 

Ela carregava um grande cesto. Não existia porta grande como essa. Ela
carregava esse cesto andando no caminho dos espíritos. 

Enquanto eles olhavam para Kasimi, o cesto apareceu. Ela não tinha
o cesto até esse momento. Depois de o cesto ficar visível, apesar de
haver uma grande entrada, o cesto não passava pela entrada. 

Ela tentava entrar com o cesto, que bloqueava a entrada, por isso os
espíritos deram uma gargalhada. Enquanto estavam rindo, ela se mexia
para conseguir entrar. A casa dos espíritos estava se fechando devagar,
fechando devagar e o xapono acabou fechando totalmente. 

O xapono onde moravam os espíritos e cuja entrada ficou fechada tinha
nome: Yoararopɨwei. Esse xapono se chamava Yoararopɨwei. Era muito
bonito. Gostavam muito dele. Apesar de ser um xapono, ele era muito
brilhante, como um espelho, possuía uma luz própria. Colocaram o nome de
Yoararopɨwei.

--- É o xapono de Yoararopɨwei --- diziam. 

Kasimi chegou lá. 

--- Meus queridos! Meus queridos! Estou chegando com meu netinho de um
grande sofrimento. Esperem"-me! Esperem"-me aí! Abram a entrada! --- dizia ela,
vindo.

--- Õ! É a voz de quem? Quem é, será?

--- Meus queridos, abram a entrada! Estou chegando com meu netinho!---
disse, vindo. --- Eu estou chegando e sofrendo de fome! Agradem a meu
netinho! O único que restou, agradem"-no! Ele é meu único!

--- Quem é você? De quem é essa voz? 

Eles queriam que ela pronunciasse seu nome.\footnote{  Os Yanomami, tradicionalmente, não podem chamar uns aos outros por seus nomes próprios, o que lhes causa constrangimento, e por isso usam termos de parentesco. Quando não há consanguinidade, são usados termos de afinidade, como cunhado ou sogro.} 

--- Quem pode ser? Quem é você mesmo? Essa voz de mulher, de quem pode
ser? Ãaaaaaõooooo! --- disseram. 

--- Sou Kasimi! Sou Kasimi! Queridinho, não pergunte quem
sou! Sou Kasimi!

Escutava"-se o som de seus pulos. Infelizmente a entrada fechou. Ela fez
assim, como quando alguns ficam presos na
cadeia. Foi assim. 

A história dos espíritos foi obra de alguém? Não pensem assim! Não foi
obra de ninguém! Ela aconteceu através de Kasimi. Essa é a verdade! 

Eles também comiam, comiam banana"-pacovã, faziam festas no tempo da
pupunha, faziam também beijus, sabiam caçar, comiam anta, quando faziam
festas; era assim que viviam os espíritos, no início. Tiravam lenha,
assim faziam. Plantavam bananeiras, enquanto moravam no limpo,
ensinando"-nos, assim, a plantar. Nós continuaremos a plantar os
alimentos como eles os plantavam. 

Atualmente, quase que nós não comeríamos pupunhas. Foram os Japiins que
espalharam as sementes de pupunheira. Não foram os antepassados
dos \emph{napë} que criaram essas pupunheiras. Eles não inventaram as
sementes. 

Omawë plantou pupunheiras, depois de inventar as sementes? Não, não foi
ele quem fez isso!

Quando nós desconhecíamos a pupunha, os japiins se agrupavam no chão, as
pupunheiras se erguiam perto daquele xapono cuja entrada ficou fechada. 

Somente eles faziam festas no tempo da pupunha. As pupunheiras não
existiam lá onde moraram Yoawë, Omawë, Ruwëri e Pore. Foram os Japiins
que criaram as pupunheiras. Os Japiins moravam com seus irmãos mais
novos, os Jaloacas. Foram eles mesmos que deram essas palmeiras. 

Não foram os Yanomami que conseguiram as sementes para podermos comê"-las
hoje. Não comemos pupunhas hoje devido a um antepassado
dos \emph{napë} que as tenha feito aparecer. Elas nos foram
dadas. Foi Japiim quem as conseguiu. A pupunha se espalhou. Foi
assim. Na verdade, foi assim. 

Apesar de serem espíritos, foram eles que ensinaram os Yanomami a fazer
festas. Os Yanomami seguem o ensino da festa daqueles dois que faziam
festas, mesmo sozinhos. Eles moravam na região central. A partir daí,
nós faremos festas. 

Ele nos ensinará a fazer a luta de \emph{yaɨmou}.\footnote{  \emph{Yaɨmou} é uma luta cerimonial realizada em festas de aliança}. Qual era o nome
desses dois? Será que alguém ensinou a vocês o nome desses dois? Se
um \emph{napë} perguntasse o nome dos dois que moravam juntos, alguém
diria o nome dos dois? Como eles se chamavam? Esses dois
ensinaram a festa e o ritual do \emph{yaɨmou}. 